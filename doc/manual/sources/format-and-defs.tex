% -*- mode: latex; fill-column: 79; mode: auto-fill; mode: flyspell; buffer-file-coding-system: utf-8 -*-
\documentclass[a4]{book}

\usepackage{minitoc} % for \addstarredchapter
\usepackage{xspace}

\newcommand{\vgi}[0]{\textsc{VGI}\xspace}
\newcommand{\vcsn}[0]{\textsc{Vaucanson}\xspace}
\newcommand{\vcsnversion}[0]{2.0.0f}

% %%%%%%%%%%%%%%%%%%%%%%%%%%%%%%%%%%%%%%%%% Adapted from nbconvert output: begin
\usepackage{graphicx} % Used to insert images
\usepackage{adjustbox} % Used to constrain images to a maximum size 
\usepackage{color} % Allow colors to be defined
\usepackage{enumerate} % Needed for markdown enumerations to work
\usepackage{geometry} % Used to adjust the document margins
\usepackage{amsmath} % Equations
\usepackage{amssymb} % Equations
\usepackage[mathletters]{ucs} % Extended unicode (utf-8) support
\usepackage[utf8x]{inputenc} % Allow utf-8 characters in the tex document
\usepackage{fancyvrb} % verbatim replacement that allows latex
\usepackage{grffile} % extends the file name processing of package graphics 
% to support a larger range
% The hyperref package gives us a pdf with properly built
% internal navigation ('pdf bookmarks' for the table of contents,
% internal cross-reference links, web links for URLs, etc.)
\usepackage{hyperref}
\usepackage{longtable} % longtable support required by pandoc >1.10
\usepackage{booktabs}  % table support for pandoc > 1.12.2

    \definecolor{orange}{cmyk}{0,0.4,0.8,0.2}
    \definecolor{darkorange}{rgb}{.71,0.21,0.01}
    \definecolor{darkgreen}{rgb}{.12,.54,.11}
    \definecolor{myteal}{rgb}{.26, .44, .56}
    \definecolor{gray}{gray}{0.45}
    \definecolor{lightgray}{gray}{.95}
    \definecolor{mediumgray}{gray}{.8}
    \definecolor{inputbackground}{rgb}{.95, .95, .85}
    \definecolor{outputbackground}{rgb}{.95, .95, .95}
    \definecolor{traceback}{rgb}{1, .95, .95}
    % ansi colors
    \definecolor{red}{rgb}{.6,0,0}
    \definecolor{green}{rgb}{0,.65,0}
    \definecolor{brown}{rgb}{0.6,0.6,0}
    \definecolor{blue}{rgb}{0,.145,.698}
    \definecolor{purple}{rgb}{.698,.145,.698}
    \definecolor{cyan}{rgb}{0,.698,.698}
    \definecolor{lightgray}{gray}{0.5}
    
    % bright ansi colors
    \definecolor{darkgray}{gray}{0.25}
    \definecolor{lightred}{rgb}{1.0,0.39,0.28}
    \definecolor{lightgreen}{rgb}{0.48,0.99,0.0}
    \definecolor{lightblue}{rgb}{0.53,0.81,0.92}
    \definecolor{lightpurple}{rgb}{0.87,0.63,0.87}
    \definecolor{lightcyan}{rgb}{0.5,1.0,0.83}

    % commands and environments needed by pandoc snippets
    % extracted from the output of `pandoc -s`
    \DefineVerbatimEnvironment{Highlighting}{Verbatim}{commandchars=\\\{\}}
    % Add ',fontsize=\small' for more characters per line
    \newenvironment{Shaded}{}{}
    \newcommand{\KeywordTok}[1]{\textcolor[rgb]{0.00,0.44,0.13}{\textbf{{#1}}}}
    \newcommand{\DataTypeTok}[1]{\textcolor[rgb]{0.56,0.13,0.00}{{#1}}}
    \newcommand{\DecValTok}[1]{\textcolor[rgb]{0.25,0.63,0.44}{{#1}}}
    \newcommand{\BaseNTok}[1]{\textcolor[rgb]{0.25,0.63,0.44}{{#1}}}
    \newcommand{\FloatTok}[1]{\textcolor[rgb]{0.25,0.63,0.44}{{#1}}}
    \newcommand{\CharTok}[1]{\textcolor[rgb]{0.25,0.44,0.63}{{#1}}}
    \newcommand{\StringTok}[1]{\textcolor[rgb]{0.25,0.44,0.63}{{#1}}}
    \newcommand{\CommentTok}[1]{\textcolor[rgb]{0.38,0.63,0.69}{\textit{{#1}}}}
    \newcommand{\OtherTok}[1]{\textcolor[rgb]{0.00,0.44,0.13}{{#1}}}
    \newcommand{\AlertTok}[1]{\textcolor[rgb]{1.00,0.00,0.00}{\textbf{{#1}}}}
    \newcommand{\FunctionTok}[1]{\textcolor[rgb]{0.02,0.16,0.49}{{#1}}}
    \newcommand{\RegionMarkerTok}[1]{{#1}}
    \newcommand{\ErrorTok}[1]{\textcolor[rgb]{1.00,0.00,0.00}{\textbf{{#1}}}}
    \newcommand{\NormalTok}[1]{{#1}}
    
    % Define a nice break command that doesn't care if a line doesn't already
    % exist.
    \def\br{\hspace*{\fill} \\* }
    % Math Jax compatability definitions
    \def\gt{>}
    \def\lt{<}

    % Pygments definitions
    
\makeatletter
\def\PY@reset{\let\PY@it=\relax \let\PY@bf=\relax%
    \let\PY@ul=\relax \let\PY@tc=\relax%
    \let\PY@bc=\relax \let\PY@ff=\relax}
\def\PY@tok#1{\csname PY@tok@#1\endcsname}
\def\PY@toks#1+{\ifx\relax#1\empty\else%
    \PY@tok{#1}\expandafter\PY@toks\fi}
\def\PY@do#1{\PY@bc{\PY@tc{\PY@ul{%
    \PY@it{\PY@bf{\PY@ff{#1}}}}}}}
\def\PY#1#2{\PY@reset\PY@toks#1+\relax+\PY@do{#2}}

\expandafter\def\csname PY@tok@gd\endcsname{\def\PY@tc##1{\textcolor[rgb]{0.63,0.00,0.00}{##1}}}
\expandafter\def\csname PY@tok@gu\endcsname{\let\PY@bf=\textbf\def\PY@tc##1{\textcolor[rgb]{0.50,0.00,0.50}{##1}}}
\expandafter\def\csname PY@tok@gt\endcsname{\def\PY@tc##1{\textcolor[rgb]{0.00,0.27,0.87}{##1}}}
\expandafter\def\csname PY@tok@gs\endcsname{\let\PY@bf=\textbf}
\expandafter\def\csname PY@tok@gr\endcsname{\def\PY@tc##1{\textcolor[rgb]{1.00,0.00,0.00}{##1}}}
\expandafter\def\csname PY@tok@cm\endcsname{\let\PY@it=\textit\def\PY@tc##1{\textcolor[rgb]{0.25,0.50,0.50}{##1}}}
\expandafter\def\csname PY@tok@vg\endcsname{\def\PY@tc##1{\textcolor[rgb]{0.10,0.09,0.49}{##1}}}
\expandafter\def\csname PY@tok@m\endcsname{\def\PY@tc##1{\textcolor[rgb]{0.40,0.40,0.40}{##1}}}
\expandafter\def\csname PY@tok@mh\endcsname{\def\PY@tc##1{\textcolor[rgb]{0.40,0.40,0.40}{##1}}}
\expandafter\def\csname PY@tok@go\endcsname{\def\PY@tc##1{\textcolor[rgb]{0.53,0.53,0.53}{##1}}}
\expandafter\def\csname PY@tok@ge\endcsname{\let\PY@it=\textit}
\expandafter\def\csname PY@tok@vc\endcsname{\def\PY@tc##1{\textcolor[rgb]{0.10,0.09,0.49}{##1}}}
\expandafter\def\csname PY@tok@il\endcsname{\def\PY@tc##1{\textcolor[rgb]{0.40,0.40,0.40}{##1}}}
\expandafter\def\csname PY@tok@cs\endcsname{\let\PY@it=\textit\def\PY@tc##1{\textcolor[rgb]{0.25,0.50,0.50}{##1}}}
\expandafter\def\csname PY@tok@cp\endcsname{\def\PY@tc##1{\textcolor[rgb]{0.74,0.48,0.00}{##1}}}
\expandafter\def\csname PY@tok@gi\endcsname{\def\PY@tc##1{\textcolor[rgb]{0.00,0.63,0.00}{##1}}}
\expandafter\def\csname PY@tok@gh\endcsname{\let\PY@bf=\textbf\def\PY@tc##1{\textcolor[rgb]{0.00,0.00,0.50}{##1}}}
\expandafter\def\csname PY@tok@ni\endcsname{\let\PY@bf=\textbf\def\PY@tc##1{\textcolor[rgb]{0.60,0.60,0.60}{##1}}}
\expandafter\def\csname PY@tok@nl\endcsname{\def\PY@tc##1{\textcolor[rgb]{0.63,0.63,0.00}{##1}}}
\expandafter\def\csname PY@tok@nn\endcsname{\let\PY@bf=\textbf\def\PY@tc##1{\textcolor[rgb]{0.00,0.00,1.00}{##1}}}
\expandafter\def\csname PY@tok@no\endcsname{\def\PY@tc##1{\textcolor[rgb]{0.53,0.00,0.00}{##1}}}
\expandafter\def\csname PY@tok@na\endcsname{\def\PY@tc##1{\textcolor[rgb]{0.49,0.56,0.16}{##1}}}
\expandafter\def\csname PY@tok@nb\endcsname{\def\PY@tc##1{\textcolor[rgb]{0.00,0.50,0.00}{##1}}}
\expandafter\def\csname PY@tok@nc\endcsname{\let\PY@bf=\textbf\def\PY@tc##1{\textcolor[rgb]{0.00,0.00,1.00}{##1}}}
\expandafter\def\csname PY@tok@nd\endcsname{\def\PY@tc##1{\textcolor[rgb]{0.67,0.13,1.00}{##1}}}
\expandafter\def\csname PY@tok@ne\endcsname{\let\PY@bf=\textbf\def\PY@tc##1{\textcolor[rgb]{0.82,0.25,0.23}{##1}}}
\expandafter\def\csname PY@tok@nf\endcsname{\def\PY@tc##1{\textcolor[rgb]{0.00,0.00,1.00}{##1}}}
\expandafter\def\csname PY@tok@si\endcsname{\let\PY@bf=\textbf\def\PY@tc##1{\textcolor[rgb]{0.73,0.40,0.53}{##1}}}
\expandafter\def\csname PY@tok@s2\endcsname{\def\PY@tc##1{\textcolor[rgb]{0.73,0.13,0.13}{##1}}}
\expandafter\def\csname PY@tok@vi\endcsname{\def\PY@tc##1{\textcolor[rgb]{0.10,0.09,0.49}{##1}}}
\expandafter\def\csname PY@tok@nt\endcsname{\let\PY@bf=\textbf\def\PY@tc##1{\textcolor[rgb]{0.00,0.50,0.00}{##1}}}
\expandafter\def\csname PY@tok@nv\endcsname{\def\PY@tc##1{\textcolor[rgb]{0.10,0.09,0.49}{##1}}}
\expandafter\def\csname PY@tok@s1\endcsname{\def\PY@tc##1{\textcolor[rgb]{0.73,0.13,0.13}{##1}}}
\expandafter\def\csname PY@tok@sh\endcsname{\def\PY@tc##1{\textcolor[rgb]{0.73,0.13,0.13}{##1}}}
\expandafter\def\csname PY@tok@sc\endcsname{\def\PY@tc##1{\textcolor[rgb]{0.73,0.13,0.13}{##1}}}
\expandafter\def\csname PY@tok@sx\endcsname{\def\PY@tc##1{\textcolor[rgb]{0.00,0.50,0.00}{##1}}}
\expandafter\def\csname PY@tok@bp\endcsname{\def\PY@tc##1{\textcolor[rgb]{0.00,0.50,0.00}{##1}}}
\expandafter\def\csname PY@tok@c1\endcsname{\let\PY@it=\textit\def\PY@tc##1{\textcolor[rgb]{0.25,0.50,0.50}{##1}}}
\expandafter\def\csname PY@tok@kc\endcsname{\let\PY@bf=\textbf\def\PY@tc##1{\textcolor[rgb]{0.00,0.50,0.00}{##1}}}
\expandafter\def\csname PY@tok@c\endcsname{\let\PY@it=\textit\def\PY@tc##1{\textcolor[rgb]{0.25,0.50,0.50}{##1}}}
\expandafter\def\csname PY@tok@mf\endcsname{\def\PY@tc##1{\textcolor[rgb]{0.40,0.40,0.40}{##1}}}
\expandafter\def\csname PY@tok@err\endcsname{\def\PY@bc##1{\setlength{\fboxsep}{0pt}\fcolorbox[rgb]{1.00,0.00,0.00}{1,1,1}{\strut ##1}}}
\expandafter\def\csname PY@tok@kd\endcsname{\let\PY@bf=\textbf\def\PY@tc##1{\textcolor[rgb]{0.00,0.50,0.00}{##1}}}
\expandafter\def\csname PY@tok@ss\endcsname{\def\PY@tc##1{\textcolor[rgb]{0.10,0.09,0.49}{##1}}}
\expandafter\def\csname PY@tok@sr\endcsname{\def\PY@tc##1{\textcolor[rgb]{0.73,0.40,0.53}{##1}}}
\expandafter\def\csname PY@tok@mo\endcsname{\def\PY@tc##1{\textcolor[rgb]{0.40,0.40,0.40}{##1}}}
\expandafter\def\csname PY@tok@kn\endcsname{\let\PY@bf=\textbf\def\PY@tc##1{\textcolor[rgb]{0.00,0.50,0.00}{##1}}}
\expandafter\def\csname PY@tok@mi\endcsname{\def\PY@tc##1{\textcolor[rgb]{0.40,0.40,0.40}{##1}}}
\expandafter\def\csname PY@tok@gp\endcsname{\let\PY@bf=\textbf\def\PY@tc##1{\textcolor[rgb]{0.00,0.00,0.50}{##1}}}
\expandafter\def\csname PY@tok@o\endcsname{\def\PY@tc##1{\textcolor[rgb]{0.40,0.40,0.40}{##1}}}
\expandafter\def\csname PY@tok@kr\endcsname{\let\PY@bf=\textbf\def\PY@tc##1{\textcolor[rgb]{0.00,0.50,0.00}{##1}}}
\expandafter\def\csname PY@tok@s\endcsname{\def\PY@tc##1{\textcolor[rgb]{0.73,0.13,0.13}{##1}}}
\expandafter\def\csname PY@tok@kp\endcsname{\def\PY@tc##1{\textcolor[rgb]{0.00,0.50,0.00}{##1}}}
\expandafter\def\csname PY@tok@w\endcsname{\def\PY@tc##1{\textcolor[rgb]{0.73,0.73,0.73}{##1}}}
\expandafter\def\csname PY@tok@kt\endcsname{\def\PY@tc##1{\textcolor[rgb]{0.69,0.00,0.25}{##1}}}
\expandafter\def\csname PY@tok@ow\endcsname{\let\PY@bf=\textbf\def\PY@tc##1{\textcolor[rgb]{0.67,0.13,1.00}{##1}}}
\expandafter\def\csname PY@tok@sb\endcsname{\def\PY@tc##1{\textcolor[rgb]{0.73,0.13,0.13}{##1}}}
\expandafter\def\csname PY@tok@k\endcsname{\let\PY@bf=\textbf\def\PY@tc##1{\textcolor[rgb]{0.00,0.50,0.00}{##1}}}
\expandafter\def\csname PY@tok@se\endcsname{\let\PY@bf=\textbf\def\PY@tc##1{\textcolor[rgb]{0.73,0.40,0.13}{##1}}}
\expandafter\def\csname PY@tok@sd\endcsname{\let\PY@it=\textit\def\PY@tc##1{\textcolor[rgb]{0.73,0.13,0.13}{##1}}}

\def\PYZbs{\char`\\}
\def\PYZus{\char`\_}
\def\PYZob{\char`\{}
\def\PYZcb{\char`\}}
\def\PYZca{\char`\^}
\def\PYZam{\char`\&}
\def\PYZlt{\char`\<}
\def\PYZgt{\char`\>}
\def\PYZsh{\char`\#}
\def\PYZpc{\char`\%}
\def\PYZdl{\char`\$}
\def\PYZhy{\char`\-}
\def\PYZsq{\char`\'}
\def\PYZdq{\char`\"}
\def\PYZti{\char`\~}
% for compatibility with earlier versions
\def\PYZat{@}
\def\PYZlb{[}
\def\PYZrb{]}
\makeatother


    % Exact colors from NB
    \definecolor{incolor}{rgb}{0.0, 0.0, 0.5}
    \definecolor{outcolor}{rgb}{0.545, 0.0, 0.0}



    
    % Prevent overflowing lines due to hard-to-break entities
    \sloppy
    % Setup hyperref package
    \hypersetup{
      breaklinks=true,  % so long urls are correctly broken across lines
      colorlinks=true,
      urlcolor=blue,
      linkcolor=darkorange,
      citecolor=darkgreen,
      }
    % Slightly bigger margins than the latex defaults
    
    \geometry{verbose,tmargin=1in,bmargin=1in,lmargin=1in,rmargin=1in}
% %%%%%%%%%%%%%%%%%%%%%%%%%%%%%%%%%%%%%%%%% Adapted from nbconvert output: end

\def\TEST{\textcolor{darkgreen}}
\newcommand{\NOTE}[1]{\small {\textcolor{red}{[}}{\textcolor{blue}{#1}}{\textcolor{red}{]}}}
\newcommand{\urlsmall}[1]{{\scriptsize\url{#1}}}
\newcommand{\REMOVE}[1]{{\textcolor{brown}{[{\bf Remove}:~{\em #1}]}}}
\newcommand{\SUGGESTIONS}[1]{{\textcolor{brown}{[{\bf I accept suggestions}:~{#1}]}}}
\newcommand{\TOOMUCH}[1]{{\textcolor{brown}{[{\bf Is this too much?}~{#1}]}}}
\newcommand{\REREAD}[1]{{\textcolor{brown}{[{\bf Re-read}:~{#1}]}}}
\newcommand{\TODO}[1]{{\textcolor{red}{[{\bf To do}:~{#1}]}}}
\newcommand{\TODOF}[1]{\footnote{\TODO{#1}}}
\newcommand{\NEW}[1]{\textcolor{darkgreen}{\textbf{[New:} {#1}\textbf{]}}}
\newcommand{\REPHRASED}[1]{\textcolor{darkgreen}{\textbf{[Rephrased:} {#1}\textbf{]}}}
\newcommand{\RATIONALEF}[1]{\footnote{\RATIONALE{#1}}}
\newcommand{\TODOQ}[1]{{\textcolor{red}{#1}}}
\newcommand{\PREMISEWHICHCOULDBEPROVEN}[1]{}%{{\textcolor{purple}{#1}}}
\newcommand{\DONE}[1]{{\textcolor{darkgreen}{[{\bf Done}:~{#1}]}}}
\newcommand{\DONEQ}[1]{\textcolor{darkgreen}{#1}}
\newcommand{\Q}[1]{\textcolor{red}{[\textit{#1}]}}
\newcommand{\STRONG}[1]{{\textcolor{red}{[{\bf Too strong}:~{#1}]}}}
\newcommand{\STRONGQ}[1]{{\textcolor{red}{#1}}}
\newcommand{\VIOLENT}[1]{{\textcolor{red}{[{\bf Too violent}:~{#1}]}}}
\newcommand{\MAYBEVIOLENT}[1]{{\textcolor{red}{{\bf [Violent?]}~{#1}}}}
\newcommand{\VIOLENTQ}[1]{\STRONGQ{#1}}
\newcommand{\USELESS}[1]{{\textcolor{brown}{[{\bf Useless?}:~{#1}]}}}
\newcommand{\REFORMULATE}[1]{{\textcolor{brown}{[{\bf Reformulate}:~{#1}]}}}
\newcommand{\MOVE}[1]{{\textcolor{blue}{[{\bf Move}:~{#1}]}}}
\newcommand{\MOVEQ}[1]{{\textcolor{blue}{#1}}}
\newcommand{\MAYBEMOVE}[1]{{\textcolor{blue}{[{\bf Move?}~{#1}]}}}
\newcommand{\MAYBE}[1]{{\textcolor{darkgreen}{{\bf ?}{#1}{\bf ?}}}}
\newcommand{\MAYBEQ}[1]{{\textcolor{darkgreen}{#1}}}
\newcommand{\IMPORTANT}[1]{{\textcolor{blue}{[{\bf Important}:~{\em #1}]}}}
\newcommand{\REMINDER}[1]{{\textcolor{purple}{[{\bf Reminder}:~{\em #1}]}}}
\newcommand{\RATIONALE}[1]{{\textcolor{purple}{[{\bf Rationale}:~{\em #1}]}}}
\newcommand{\IMP}[1]{\IMPORTANT{#1}}
\newcommand{\CHECKINTHEEND}[1]{{\textcolor{brown}{[{\bf Check at the end}:~{\em #1}]}}}
\newcommand{\CHECK}[1]{{\textcolor{brown}{[{\bf Check}:~{\em #1}]}}}
\newcommand{\MYOR}[2]{\textcolor{red}{{\bf [}{{\textcolor{darkgreen}{#1}}{\bf ~OR~}{{\textcolor{darkgreen}{#2}}{\bf ]}}}}}
\newcommand{\ORTWO}[2]{\textcolor{red}{{\bf [}{{\textcolor{darkgreen}{#1}}{\bf ~OR~}{{\textcolor{darkgreen}{#2}}{\bf ]}}}}}
\newcommand{\ORTHREE}[3]{\textcolor{red}{{\bf [}{{\textcolor{darkgreen}{#1}}{\bf ~OR~}{{\textcolor{darkgreen}{#2}}{\bf ~OR~}{{\textcolor{darkgreen}{#3}}{\bf ]}}}}}}
\newcommand{\ORFOUR}[4]{\textcolor{red}{{\bf [}{{\textcolor{darkgreen}{#1}}{\bf ~OR~}{{\textcolor{darkgreen}{#2}}{\bf ~OR~}{{\textcolor{darkgreen}{#3}}{\bf ~OR~}{{\textcolor{darkgreen}{#4}}{\bf ]}}}}}}}
\newcommand{\SYNONYM}[1]{\textcolor{blue}{{\bf [Find a synonym}:~{\em #1}{\bf ]}}}
\newcommand{\UNSURE}[1]{\textcolor{red}{{\bf [I'm not sure of this}:~{\em #1}{\bf ]}}}
\newcommand{\WORD}[1]{\textcolor{purple}{{\bf [Is there a better term?}~{\em #1}{\bf ]}}}
\newcommand{\TERM}[1]{\WORD{#1}}
\newcommand{\LANGUAGE}[1]{\WORD{#1}}
\newcommand{\GRAMMAR}[1]{\textcolor{red}{{\bf [Is this correct in English?}~{\em #1}{\bf ]}}}
\newcommand{\GRAMMARQ}[1]{\textcolor{red}{#1}}
\newcommand{\ENGLISH}[1]{\GRAMMAR{#1}}
\newcommand{\ITALIANISM}[1]{\textcolor{red}{{\bf [Is this an italianism?}~{\em #1}{\bf ]}}}
\newcommand{\EXCESSIVE}[1]{\textcolor{red}{{\bf [This may be excessive}:~{\em #1}{\bf ]}}}
\newcommand{\HACKER}[1]{\textcolor{red}{{\bf [This is informal jargon}:~{\em #1}{\bf ]}}}
\newcommand{\JARGON}[1]{\textcolor{red}{{\bf [This is informal jargon}:~{\em #1}{\bf ]}}}
\newcommand{\INFORMAL}[1]{\textcolor{red}{{\bf [This is informal}:~{\em #1}{\bf ]}}}
\newcommand{\UGLY}[1]{\textcolor{red}{{\bf [ugly}:~{\em #1}{\bf ]}}}
\newcommand{\DISLIKE}[1]{\textcolor{red}{{\bf [I don't like this}:~{\em #1}{\bf ]}}}
\newcommand{\DONTLIKE}[1]{\DISLIKE{#1}}
\newcommand{\GENIAL}[1]{{\textcolor{darkgreen}{[{\bf A genial idea}:~{#1}]}}}
\newcommand{\GOOD}[1]{{\textcolor{darkgreen}{[{\bf Good}:~{#1}]}}}
\newcommand{\IGNORE}[1]{}
\newcommand{\OBSOLETE}[1]{{\textcolor{brown}{[{\bf Obsolete}:~{#1}]}}}
\newcommand{\OBSOLETEQ}[1]{{\textcolor{brown}{#1}}}
\newcommand{\SW}[1]{{\textcolor{darkgreen}{[{\bf Somewhere}:~{#1}]}}}
\newcommand{\SOMEWHERE}[1]{\SW{#1}}
\newcommand{\MAYBESOMEWHERE}[1]{\MAYBE{\SOMEWHERE{#1}}}
\newcommand{\SOMEWHEREMAYBE}[1]{\MAYBESOMEWHERE{#1}}
\newcommand{\FILL}[0]{{\textcolor{red}{\bf [Fill this...]}}}
\newcommand{\SOMETHING}[0]{{\textcolor{red}{\bf [Something]}}}
\newcommand{\PROBABLYNOT}[1]{{\textcolor{red}{[{\bf Probably not}:~{#1}]}}}
\newcommand{\NO}[1]{\textcolor{brown}{{\bf [I don't like this}:~{\em #1}{\bf ]}}}
\newcommand{\NOTREALLY}[1]{\textcolor{brown}{{\bf [Not really}:~{\em #1}{\bf ]}}}
\newcommand{\WHYNOT}[1]{\textcolor{red}{{\bf [Why not?}~{\em #1}{\bf ]}}}
\newcommand{\WHYNOTQ}[1]{\textcolor{red}{#1}}
\newcommand{\YES}[1]{\textcolor{red}{{\bf [Yes}:~{\em #1}{\bf ]}}}
\newcommand{\YESQ}[1]{\textcolor{red}{#1}}
\newcommand{\WRONG}[1]{\textcolor{brown}{{\bf [Wrong}:~{\em #1}{\bf ]}}}
\newcommand{\WRONGQ}[1]{\RED{#1}}
\newcommand{\LONG}[1]{\textcolor{red}{{\bf [Only for the long version}:~{\em #1}{\bf ]}}}
\newcommand{\SHORT}[1]{\textcolor{red}{{\bf [Only for the short version}:~{\em #1}{\bf ]}}}
\newcommand{\LONGSHORT}[2]{\LONG{#1}{\SHORT{#2}}}
\newcommand{\SHORTLONG}[2]{\SHORT{#1}{\LONG{#2}}}
\newcommand{\TEMP}[1]{{\textcolor{darkgreen}{[{#1}]}}}
\newcommand{\META}[1]{\textcolor{darkgreen}{[{\em {#1}}]}}
\newcommand{\WR}[1]{{\textcolor{darkgreen}{[would rephrase: {\em {#1}}]}}}
\newcommand{\INVISIBLE}[1]{{\textcolor{white}{{#1}}}}
\newcommand{\SKIPALINE}[0]{{\\\INVISIBLE{.}\\}}
%\newcommand{\NOTHING}[0]{}
\newcommand{\NOTHING}[0]{\INVISIBLE{\em hack}}
\def\TEMPBLOCK{\textcolor{darkgreen}}

\newcommand{\PushLine}{\hbox{}\hfill\hbox{}}

\makeatletter

\def\cleardoublepage{\clearpage\if@twoside \ifodd\c@page\else%
  \hbox{}%
  \thispagestyle{empty}%              % Empty header styles
  \newpage%
  \if@twocolumn\hbox{}\newpage\fi\fi\fi}

\makeatother

\usepackage{color}
\definecolor{linkcol}{rgb}{0,0,0.4} 
\definecolor{citecol}{rgb}{0.5,0,0} 
