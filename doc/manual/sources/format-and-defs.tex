% -*- mode: latex; fill-column: 79; mode: auto-fill; mode: flyspell; buffer-file-coding-system: utf-8 -*-
\documentclass{book}
%\documentclass{book}

\usepackage{minitoc} % for \addstarredchapter
\usepackage{xspace}

\newcommand{\vgi}[0]{\textsc{VGI}\xspace}
\newcommand{\vcsn}[0]{\textsc{Vaucanson}\xspace}
\newcommand{\vcsnversion}[0]{2.0.0f}

\usepackage[paper=a4paper]{geometry}
\usepackage{url}

% %%%%%%%%%%%%%%%%%%%%%%%%%%%%%%%%%%%%%%%%% Adapted from nbconvert output: begin
\usepackage{graphicx} % Used to insert images
\usepackage{adjustbox} % Used to constrain images to a maximum size 
\usepackage{color} % Allow colors to be defined
\usepackage{enumerate} % Needed for markdown enumerations to work
%\usepackage{geometry} % Used to adjust the document margins
\usepackage{amsmath} % Equations
\usepackage{amssymb} % Equations
\usepackage[mathletters]{ucs} % Extended unicode (utf-8) support
\usepackage[utf8x]{inputenc} % Allow utf-8 characters in the tex document
\usepackage{fancyvrb} % verbatim replacement that allows latex
\usepackage{grffile} % extends the file name processing of package graphics 
% to support a larger range

% This has to be loaded before hyperref to make Index entries into hyperlinks.\
\usepackage{imakeidx}

% The hyperref package gives us a pdf with properly built
% internal navigation ('pdf bookmarks' for the table of contents,
% internal cross-reference links, web links for URLs, etc.)
%\usepackage{hyperref}
\usepackage[colorlinks=false,linkcolor=blue,filecolor=blue,citecolor=blue,urlcolor=red,pagebackref,hyperindex=true]{hyperref}
\usepackage{appendix}
\usepackage{longtable} % longtable support required by pandoc >1.10
\usepackage{booktabs}  % table support for pandoc > 1.12.2
\usepackage{listings}

    \definecolor{orange}{cmyk}{0,0.4,0.8,0.2}
    \definecolor{darkorange}{rgb}{.71,0.21,0.01}
    \definecolor{darkgreen}{rgb}{.12,.54,.11}
    \definecolor{myteal}{rgb}{.26, .44, .56}
    \definecolor{gray}{gray}{0.45}
    \definecolor{lightgray}{gray}{.95}
    \definecolor{mediumgray}{gray}{.8}
    \definecolor{inputbackground}{rgb}{.95, .95, .85}
    \definecolor{outputbackground}{rgb}{.95, .95, .95}
    \definecolor{traceback}{rgb}{1, .95, .95}
    % ansi colors
    \definecolor{red}{rgb}{.6,0,0}
    \definecolor{green}{rgb}{0,.65,0}
    \definecolor{brown}{rgb}{0.6,0.6,0}
    \definecolor{blue}{rgb}{0,.145,.698}
    \definecolor{purple}{rgb}{.698,.145,.698}
    \definecolor{cyan}{rgb}{0,.698,.698}
    \definecolor{lightgray}{gray}{0.5}
    
    % bright ansi colors
    \definecolor{darkgray}{gray}{0.25}
    \definecolor{lightred}{rgb}{1.0,0.39,0.28}
    \definecolor{lightgreen}{rgb}{0.48,0.99,0.0}
    \definecolor{lightblue}{rgb}{0.53,0.81,0.92}
    \definecolor{lightpurple}{rgb}{0.87,0.63,0.87}
    \definecolor{lightcyan}{rgb}{0.5,1.0,0.83}

    % commands and environments needed by pandoc snippets
    % extracted from the output of `pandoc -s`
    \DefineVerbatimEnvironment{Highlighting}{Verbatim}{commandchars=\\\{\}}
    % Add ',fontsize=\small' for more characters per line
    \newenvironment{Shaded}{}{}
    \newcommand{\KeywordTok}[1]{\textcolor[rgb]{0.00,0.44,0.13}{\textbf{{#1}}}}
    \newcommand{\DataTypeTok}[1]{\textcolor[rgb]{0.56,0.13,0.00}{{#1}}}
    \newcommand{\DecValTok}[1]{\textcolor[rgb]{0.25,0.63,0.44}{{#1}}}
    \newcommand{\BaseNTok}[1]{\textcolor[rgb]{0.25,0.63,0.44}{{#1}}}
    \newcommand{\FloatTok}[1]{\textcolor[rgb]{0.25,0.63,0.44}{{#1}}}
    \newcommand{\CharTok}[1]{\textcolor[rgb]{0.25,0.44,0.63}{{#1}}}
    \newcommand{\StringTok}[1]{\textcolor[rgb]{0.25,0.44,0.63}{{#1}}}
    \newcommand{\CommentTok}[1]{\textcolor[rgb]{0.38,0.63,0.69}{\textit{{#1}}}}
    \newcommand{\OtherTok}[1]{\textcolor[rgb]{0.00,0.44,0.13}{{#1}}}
    \newcommand{\AlertTok}[1]{\textcolor[rgb]{1.00,0.00,0.00}{\textbf{{#1}}}}
    \newcommand{\FunctionTok}[1]{\textcolor[rgb]{0.02,0.16,0.49}{{#1}}}
    \newcommand{\RegionMarkerTok}[1]{{#1}}
    \newcommand{\ErrorTok}[1]{\textcolor[rgb]{1.00,0.00,0.00}{\textbf{{#1}}}}
    \newcommand{\NormalTok}[1]{{#1}}
    
    % Define a nice break command that doesn't care if a line doesn't already
    % exist.
    \def\br{\hspace*{\fill} \\* }
    % Math Jax compatability definitions
    \def\gt{>}
    \def\lt{<}

    % Pygments definitions
    
\makeatletter
\def\PY@reset{\let\PY@it=\relax \let\PY@bf=\relax%
    \let\PY@ul=\relax \let\PY@tc=\relax%
    \let\PY@bc=\relax \let\PY@ff=\relax}
\def\PY@tok#1{\csname PY@tok@#1\endcsname}
\def\PY@toks#1+{\ifx\relax#1\empty\else%
    \PY@tok{#1}\expandafter\PY@toks\fi}
\def\PY@do#1{\PY@bc{\PY@tc{\PY@ul{%
    \PY@it{\PY@bf{\PY@ff{#1}}}}}}}
\def\PY#1#2{\PY@reset\PY@toks#1+\relax+\PY@do{#2}}

\expandafter\def\csname PY@tok@gd\endcsname{\def\PY@tc##1{\textcolor[rgb]{0.63,0.00,0.00}{##1}}}
\expandafter\def\csname PY@tok@gu\endcsname{\let\PY@bf=\textbf\def\PY@tc##1{\textcolor[rgb]{0.50,0.00,0.50}{##1}}}
\expandafter\def\csname PY@tok@gt\endcsname{\def\PY@tc##1{\textcolor[rgb]{0.00,0.27,0.87}{##1}}}
\expandafter\def\csname PY@tok@gs\endcsname{\let\PY@bf=\textbf}
\expandafter\def\csname PY@tok@gr\endcsname{\def\PY@tc##1{\textcolor[rgb]{1.00,0.00,0.00}{##1}}}
\expandafter\def\csname PY@tok@cm\endcsname{\let\PY@it=\textit\def\PY@tc##1{\textcolor[rgb]{0.25,0.50,0.50}{##1}}}
\expandafter\def\csname PY@tok@vg\endcsname{\def\PY@tc##1{\textcolor[rgb]{0.10,0.09,0.49}{##1}}}
\expandafter\def\csname PY@tok@m\endcsname{\def\PY@tc##1{\textcolor[rgb]{0.40,0.40,0.40}{##1}}}
\expandafter\def\csname PY@tok@mh\endcsname{\def\PY@tc##1{\textcolor[rgb]{0.40,0.40,0.40}{##1}}}
\expandafter\def\csname PY@tok@go\endcsname{\def\PY@tc##1{\textcolor[rgb]{0.53,0.53,0.53}{##1}}}
\expandafter\def\csname PY@tok@ge\endcsname{\let\PY@it=\textit}
\expandafter\def\csname PY@tok@vc\endcsname{\def\PY@tc##1{\textcolor[rgb]{0.10,0.09,0.49}{##1}}}
\expandafter\def\csname PY@tok@il\endcsname{\def\PY@tc##1{\textcolor[rgb]{0.40,0.40,0.40}{##1}}}
\expandafter\def\csname PY@tok@cs\endcsname{\let\PY@it=\textit\def\PY@tc##1{\textcolor[rgb]{0.25,0.50,0.50}{##1}}}
\expandafter\def\csname PY@tok@cp\endcsname{\def\PY@tc##1{\textcolor[rgb]{0.74,0.48,0.00}{##1}}}
\expandafter\def\csname PY@tok@gi\endcsname{\def\PY@tc##1{\textcolor[rgb]{0.00,0.63,0.00}{##1}}}
\expandafter\def\csname PY@tok@gh\endcsname{\let\PY@bf=\textbf\def\PY@tc##1{\textcolor[rgb]{0.00,0.00,0.50}{##1}}}
\expandafter\def\csname PY@tok@ni\endcsname{\let\PY@bf=\textbf\def\PY@tc##1{\textcolor[rgb]{0.60,0.60,0.60}{##1}}}
\expandafter\def\csname PY@tok@nl\endcsname{\def\PY@tc##1{\textcolor[rgb]{0.63,0.63,0.00}{##1}}}
\expandafter\def\csname PY@tok@nn\endcsname{\let\PY@bf=\textbf\def\PY@tc##1{\textcolor[rgb]{0.00,0.00,1.00}{##1}}}
\expandafter\def\csname PY@tok@no\endcsname{\def\PY@tc##1{\textcolor[rgb]{0.53,0.00,0.00}{##1}}}
\expandafter\def\csname PY@tok@na\endcsname{\def\PY@tc##1{\textcolor[rgb]{0.49,0.56,0.16}{##1}}}
\expandafter\def\csname PY@tok@nb\endcsname{\def\PY@tc##1{\textcolor[rgb]{0.00,0.50,0.00}{##1}}}
\expandafter\def\csname PY@tok@nc\endcsname{\let\PY@bf=\textbf\def\PY@tc##1{\textcolor[rgb]{0.00,0.00,1.00}{##1}}}
\expandafter\def\csname PY@tok@nd\endcsname{\def\PY@tc##1{\textcolor[rgb]{0.67,0.13,1.00}{##1}}}
\expandafter\def\csname PY@tok@ne\endcsname{\let\PY@bf=\textbf\def\PY@tc##1{\textcolor[rgb]{0.82,0.25,0.23}{##1}}}
\expandafter\def\csname PY@tok@nf\endcsname{\def\PY@tc##1{\textcolor[rgb]{0.00,0.00,1.00}{##1}}}
\expandafter\def\csname PY@tok@si\endcsname{\let\PY@bf=\textbf\def\PY@tc##1{\textcolor[rgb]{0.73,0.40,0.53}{##1}}}
\expandafter\def\csname PY@tok@s2\endcsname{\def\PY@tc##1{\textcolor[rgb]{0.73,0.13,0.13}{##1}}}
\expandafter\def\csname PY@tok@vi\endcsname{\def\PY@tc##1{\textcolor[rgb]{0.10,0.09,0.49}{##1}}}
\expandafter\def\csname PY@tok@nt\endcsname{\let\PY@bf=\textbf\def\PY@tc##1{\textcolor[rgb]{0.00,0.50,0.00}{##1}}}
\expandafter\def\csname PY@tok@nv\endcsname{\def\PY@tc##1{\textcolor[rgb]{0.10,0.09,0.49}{##1}}}
\expandafter\def\csname PY@tok@s1\endcsname{\def\PY@tc##1{\textcolor[rgb]{0.73,0.13,0.13}{##1}}}
\expandafter\def\csname PY@tok@sh\endcsname{\def\PY@tc##1{\textcolor[rgb]{0.73,0.13,0.13}{##1}}}
\expandafter\def\csname PY@tok@sc\endcsname{\def\PY@tc##1{\textcolor[rgb]{0.73,0.13,0.13}{##1}}}
\expandafter\def\csname PY@tok@sx\endcsname{\def\PY@tc##1{\textcolor[rgb]{0.00,0.50,0.00}{##1}}}
\expandafter\def\csname PY@tok@bp\endcsname{\def\PY@tc##1{\textcolor[rgb]{0.00,0.50,0.00}{##1}}}
\expandafter\def\csname PY@tok@c1\endcsname{\let\PY@it=\textit\def\PY@tc##1{\textcolor[rgb]{0.25,0.50,0.50}{##1}}}
\expandafter\def\csname PY@tok@kc\endcsname{\let\PY@bf=\textbf\def\PY@tc##1{\textcolor[rgb]{0.00,0.50,0.00}{##1}}}
\expandafter\def\csname PY@tok@c\endcsname{\let\PY@it=\textit\def\PY@tc##1{\textcolor[rgb]{0.25,0.50,0.50}{##1}}}
\expandafter\def\csname PY@tok@mf\endcsname{\def\PY@tc##1{\textcolor[rgb]{0.40,0.40,0.40}{##1}}}
\expandafter\def\csname PY@tok@err\endcsname{\def\PY@bc##1{\setlength{\fboxsep}{0pt}\fcolorbox[rgb]{1.00,0.00,0.00}{1,1,1}{\strut ##1}}}
\expandafter\def\csname PY@tok@kd\endcsname{\let\PY@bf=\textbf\def\PY@tc##1{\textcolor[rgb]{0.00,0.50,0.00}{##1}}}
\expandafter\def\csname PY@tok@ss\endcsname{\def\PY@tc##1{\textcolor[rgb]{0.10,0.09,0.49}{##1}}}
\expandafter\def\csname PY@tok@sr\endcsname{\def\PY@tc##1{\textcolor[rgb]{0.73,0.40,0.53}{##1}}}
\expandafter\def\csname PY@tok@mo\endcsname{\def\PY@tc##1{\textcolor[rgb]{0.40,0.40,0.40}{##1}}}
\expandafter\def\csname PY@tok@kn\endcsname{\let\PY@bf=\textbf\def\PY@tc##1{\textcolor[rgb]{0.00,0.50,0.00}{##1}}}
\expandafter\def\csname PY@tok@mi\endcsname{\def\PY@tc##1{\textcolor[rgb]{0.40,0.40,0.40}{##1}}}
\expandafter\def\csname PY@tok@gp\endcsname{\let\PY@bf=\textbf\def\PY@tc##1{\textcolor[rgb]{0.00,0.00,0.50}{##1}}}
\expandafter\def\csname PY@tok@o\endcsname{\def\PY@tc##1{\textcolor[rgb]{0.40,0.40,0.40}{##1}}}
\expandafter\def\csname PY@tok@kr\endcsname{\let\PY@bf=\textbf\def\PY@tc##1{\textcolor[rgb]{0.00,0.50,0.00}{##1}}}
\expandafter\def\csname PY@tok@s\endcsname{\def\PY@tc##1{\textcolor[rgb]{0.73,0.13,0.13}{##1}}}
\expandafter\def\csname PY@tok@kp\endcsname{\def\PY@tc##1{\textcolor[rgb]{0.00,0.50,0.00}{##1}}}
\expandafter\def\csname PY@tok@w\endcsname{\def\PY@tc##1{\textcolor[rgb]{0.73,0.73,0.73}{##1}}}
\expandafter\def\csname PY@tok@kt\endcsname{\def\PY@tc##1{\textcolor[rgb]{0.69,0.00,0.25}{##1}}}
\expandafter\def\csname PY@tok@ow\endcsname{\let\PY@bf=\textbf\def\PY@tc##1{\textcolor[rgb]{0.67,0.13,1.00}{##1}}}
\expandafter\def\csname PY@tok@sb\endcsname{\def\PY@tc##1{\textcolor[rgb]{0.73,0.13,0.13}{##1}}}
\expandafter\def\csname PY@tok@k\endcsname{\let\PY@bf=\textbf\def\PY@tc##1{\textcolor[rgb]{0.00,0.50,0.00}{##1}}}
\expandafter\def\csname PY@tok@se\endcsname{\let\PY@bf=\textbf\def\PY@tc##1{\textcolor[rgb]{0.73,0.40,0.13}{##1}}}
\expandafter\def\csname PY@tok@sd\endcsname{\let\PY@it=\textit\def\PY@tc##1{\textcolor[rgb]{0.73,0.13,0.13}{##1}}}

\def\PYZbs{\char`\\}
\def\PYZus{\char`\_}
\def\PYZob{\char`\{}
\def\PYZcb{\char`\}}
\def\PYZca{\char`\^}
\def\PYZam{\char`\&}
\def\PYZlt{\char`\<}
\def\PYZgt{\char`\>}
\def\PYZsh{\char`\#}
\def\PYZpc{\char`\%}
\def\PYZdl{\char`\$}
\def\PYZhy{\char`\-}
\def\PYZsq{\char`\'}
\def\PYZdq{\char`\"}
\def\PYZti{\char`\~}
% for compatibility with earlier versions
\def\PYZat{@}
\def\PYZlb{[}
\def\PYZrb{]}
\makeatother


    % Exact colors from NB
    \definecolor{incolor}{rgb}{0.0, 0.0, 0.5}
    \definecolor{outcolor}{rgb}{0.545, 0.0, 0.0}



    % Prevent overflowing lines due to hard-to-break entities
    \sloppy

    %% % Slightly bigger margins than the latex defaults
    %\geometry{verbose,tmargin=1in,bmargin=1in,lmargin=1in,rmargin=1in}
% %%%%%%%%%%%%%%%%%%%%%%%%%%%%%%%%%%%%%%%%% Adapted from nbconvert output: end

\def\TEST{\textcolor{darkgreen}}
\newcommand{\NOTE}[1]{\small {\textcolor{red}{[}}{\textcolor{blue}{#1}}{\textcolor{red}{]}}}
\newcommand{\urlsmall}[1]{{\scriptsize\url{#1}}}
\newcommand{\REMOVE}[1]{{\textcolor{brown}{[{\bf Remove}:~{\em #1}]}}}
\newcommand{\SUGGESTIONS}[1]{{\textcolor{brown}{[{\bf I accept suggestions}:~{#1}]}}}
\newcommand{\TOOMUCH}[1]{{\textcolor{brown}{[{\bf Is this too much?}~{#1}]}}}
\newcommand{\REREAD}[1]{{\textcolor{brown}{[{\bf Re-read}:~{#1}]}}}
\newcommand{\TODO}[1]{{\textcolor{red}{[{\bf To do}:~{#1}]}}}
\newcommand{\FIXME}[1]{{\textcolor{red}{[{\bf FIXME}:~{#1}]}}}
\newcommand{\TODOF}[1]{\footnote{\TODO{#1}}}
\newcommand{\NEW}[1]{\textcolor{darkgreen}{\textbf{[New:} {#1}\textbf{]}}}
\newcommand{\REPHRASED}[1]{\textcolor{darkgreen}{\textbf{[Rephrased:} {#1}\textbf{]}}}
\newcommand{\RATIONALEF}[1]{\footnote{\RATIONALE{#1}}}
\newcommand{\TODOQ}[1]{{\textcolor{red}{#1}}}
\newcommand{\PREMISEWHICHCOULDBEPROVEN}[1]{}%{{\textcolor{purple}{#1}}}
\newcommand{\DONE}[1]{{\textcolor{darkgreen}{[{\bf Done}:~{#1}]}}}
\newcommand{\DONEQ}[1]{\textcolor{darkgreen}{#1}}
%\newcommand{\Q}[1]{\textcolor{red}{[\textit{#1}]}}
\newcommand{\STRONG}[1]{{\textcolor{red}{[{\bf Too strong}:~{#1}]}}}
\newcommand{\STRONGQ}[1]{{\textcolor{red}{#1}}}
\newcommand{\VIOLENT}[1]{{\textcolor{red}{[{\bf Too violent}:~{#1}]}}}
\newcommand{\MAYBEVIOLENT}[1]{{\textcolor{red}{{\bf [Violent?]}~{#1}}}}
\newcommand{\VIOLENTQ}[1]{\STRONGQ{#1}}
\newcommand{\USELESS}[1]{{\textcolor{brown}{[{\bf Useless?}:~{#1}]}}}
\newcommand{\REFORMULATE}[1]{{\textcolor{brown}{[{\bf Reformulate}:~{#1}]}}}
\newcommand{\MOVE}[1]{{\textcolor{blue}{[{\bf Move}:~{#1}]}}}
\newcommand{\MOVEQ}[1]{{\textcolor{blue}{#1}}}
\newcommand{\MAYBEMOVE}[1]{{\textcolor{blue}{[{\bf Move?}~{#1}]}}}
\newcommand{\MAYBE}[1]{{\textcolor{darkgreen}{{\bf ?}{#1}{\bf ?}}}}
\newcommand{\MAYBEQ}[1]{{\textcolor{darkgreen}{#1}}}
\newcommand{\IMPORTANT}[1]{{\textcolor{blue}{[{\bf Important}:~{\em #1}]}}}
\newcommand{\REMINDER}[1]{{\textcolor{purple}{[{\bf Reminder}:~{\em #1}]}}}
\newcommand{\RATIONALE}[1]{{\textcolor{purple}{[{\bf Rationale}:~{\em #1}]}}}
\newcommand{\IMP}[1]{\IMPORTANT{#1}}
\newcommand{\CHECKINTHEEND}[1]{{\textcolor{brown}{[{\bf Check at the end}:~{\em #1}]}}}
\newcommand{\CHECK}[1]{{\textcolor{brown}{[{\bf Check}:~{\em #1}]}}}
\newcommand{\MYOR}[2]{\textcolor{red}{{\bf [}{{\textcolor{darkgreen}{#1}}{\bf ~OR~}{{\textcolor{darkgreen}{#2}}{\bf ]}}}}}
\newcommand{\ORTWO}[2]{\textcolor{red}{{\bf [}{{\textcolor{darkgreen}{#1}}{\bf ~OR~}{{\textcolor{darkgreen}{#2}}{\bf ]}}}}}
\newcommand{\ORTHREE}[3]{\textcolor{red}{{\bf [}{{\textcolor{darkgreen}{#1}}{\bf ~OR~}{{\textcolor{darkgreen}{#2}}{\bf ~OR~}{{\textcolor{darkgreen}{#3}}{\bf ]}}}}}}
\newcommand{\ORFOUR}[4]{\textcolor{red}{{\bf [}{{\textcolor{darkgreen}{#1}}{\bf ~OR~}{{\textcolor{darkgreen}{#2}}{\bf ~OR~}{{\textcolor{darkgreen}{#3}}{\bf ~OR~}{{\textcolor{darkgreen}{#4}}{\bf ]}}}}}}}
\newcommand{\SYNONYM}[1]{\textcolor{blue}{{\bf [Find a synonym}:~{\em #1}{\bf ]}}}
\newcommand{\UNSURE}[1]{\textcolor{red}{{\bf [I'm not sure of this}:~{\em #1}{\bf ]}}}
\newcommand{\WORD}[1]{\textcolor{purple}{{\bf [Is there a better term?}~{\em #1}{\bf ]}}}
\newcommand{\TERM}[1]{\WORD{#1}}
\newcommand{\LANGUAGE}[1]{\WORD{#1}}
\newcommand{\GRAMMAR}[1]{\textcolor{red}{{\bf [Is this correct in English?}~{\em #1}{\bf ]}}}
\newcommand{\GRAMMARQ}[1]{\textcolor{red}{#1}}
\newcommand{\ENGLISH}[1]{\GRAMMAR{#1}}
\newcommand{\ITALIANISM}[1]{\textcolor{red}{{\bf [Is this an italianism?}~{\em #1}{\bf ]}}}
\newcommand{\EXCESSIVE}[1]{\textcolor{red}{{\bf [This may be excessive}:~{\em #1}{\bf ]}}}
\newcommand{\HACKER}[1]{\textcolor{red}{{\bf [This is informal jargon}:~{\em #1}{\bf ]}}}
\newcommand{\JARGON}[1]{\textcolor{red}{{\bf [This is informal jargon}:~{\em #1}{\bf ]}}}
\newcommand{\INFORMAL}[1]{\textcolor{red}{{\bf [This is informal}:~{\em #1}{\bf ]}}}
\newcommand{\UGLY}[1]{\textcolor{red}{{\bf [ugly}:~{\em #1}{\bf ]}}}
\newcommand{\DISLIKE}[1]{\textcolor{red}{{\bf [I don't like this}:~{\em #1}{\bf ]}}}
\newcommand{\DONTLIKE}[1]{\DISLIKE{#1}}
\newcommand{\GENIAL}[1]{{\textcolor{darkgreen}{[{\bf A genial idea}:~{#1}]}}}
\newcommand{\GOOD}[1]{{\textcolor{darkgreen}{[{\bf Good}:~{#1}]}}}
\newcommand{\IGNORE}[1]{}
\newcommand{\OBSOLETE}[1]{{\textcolor{brown}{[{\bf Obsolete}:~{#1}]}}}
\newcommand{\OBSOLETEQ}[1]{{\textcolor{brown}{#1}}}
\newcommand{\SW}[1]{{\textcolor{darkgreen}{[{\bf Somewhere}:~{#1}]}}}
\newcommand{\SOMEWHERE}[1]{\SW{#1}}
\newcommand{\MAYBESOMEWHERE}[1]{\MAYBE{\SOMEWHERE{#1}}}
\newcommand{\SOMEWHEREMAYBE}[1]{\MAYBESOMEWHERE{#1}}
\newcommand{\FILL}[0]{{\textcolor{red}{\bf [Fill this...]}}}
\newcommand{\SOMETHING}[0]{{\textcolor{red}{\bf [Something]}}}
\newcommand{\PROBABLYNOT}[1]{{\textcolor{red}{[{\bf Probably not}:~{#1}]}}}
\newcommand{\NO}[1]{\textcolor{brown}{{\bf [I don't like this}:~{\em #1}{\bf ]}}}
\newcommand{\NOTREALLY}[1]{\textcolor{brown}{{\bf [Not really}:~{\em #1}{\bf ]}}}
\newcommand{\WHYNOT}[1]{\textcolor{red}{{\bf [Why not?}~{\em #1}{\bf ]}}}
\newcommand{\WHYNOTQ}[1]{\textcolor{red}{#1}}
\newcommand{\YES}[1]{\textcolor{red}{{\bf [Yes}:~{\em #1}{\bf ]}}}
\newcommand{\YESQ}[1]{\textcolor{red}{#1}}
\newcommand{\WRONG}[1]{\textcolor{brown}{{\bf [Wrong}:~{\em #1}{\bf ]}}}
\newcommand{\WRONGQ}[1]{\RED{#1}}
\newcommand{\LONG}[1]{\textcolor{red}{{\bf [Only for the long version}:~{\em #1}{\bf ]}}}
\newcommand{\SHORT}[1]{\textcolor{red}{{\bf [Only for the short version}:~{\em #1}{\bf ]}}}
\newcommand{\LONGSHORT}[2]{\LONG{#1}{\SHORT{#2}}}
\newcommand{\SHORTLONG}[2]{\SHORT{#1}{\LONG{#2}}}
\newcommand{\TEMP}[1]{{\textcolor{darkgreen}{[{#1}]}}}
\newcommand{\META}[1]{\textcolor{darkgreen}{[{\em {#1}}]}}
\newcommand{\WR}[1]{{\textcolor{darkgreen}{[would rephrase: {\em {#1}}]}}}
\newcommand{\INVISIBLE}[1]{{\textcolor{white}{{#1}}}}
\newcommand{\SKIPALINE}[0]{{\\\INVISIBLE{.}\\}}
%\newcommand{\NOTHING}[0]{}
\newcommand{\NOTHING}[0]{\INVISIBLE{\em hack}}
\def\TEMPBLOCK{\textcolor{darkgreen}}

%\newcommand{\PushLine}{\hbox{}\hfill\hbox{}}

\makeatletter

\def\cleardoublepage{\clearpage\if@twoside \ifodd\c@page\else%
  \hbox{}%
  \thispagestyle{empty}%              % Empty header styles
  \newpage%
  \if@twocolumn\hbox{}\newpage\fi\fi\fi}

\makeatother

\usepackage{color}
\definecolor{linkcol}{rgb}{0,0,0.4} 
\definecolor{citecol}{rgb}{0.5,0,0} 

%%%%%%%%%%%% ================

\newcommand{\FATBRACKETS}[1]
  {\mbox{$[\![ #1 ]\!]$}}
\newcommand{\BANANABRACKETS}[1]
  {\mbox{$\llparenthesis {#1} \rrparenthesis$}}
\newcommand{\LFATBRACKETS}[2]{\mbox{$\mathcal{#1}[\![ #2 ]\!]$}}
\newcommand{\PFATBRACKETS}[2]{\mbox{$#1[\![ #2 ]\!]$}}
\newcommand{\TRANSFORM}[2]{\mbox{${#1}[\![ #2 ]\!]$}}
\newcommand{\SSYNTAX}[1]{\mbox{$S[\![ #1 ]\!]$}}

\newcommand{\LIFT}[1]
  {\lfloor{#1}\rfloor}
%  {\lvert{#1}\rvert}
%  {\lceil\lfloor{#1}\rfloor\rceil}

\newcommand{\MACROEXPANDSTO}[0]{\ \ \equiv\ \ }
\newcommand{\EQD}[0]{\triangleq}

%% Reified sementic structures:
\newcommand{\PRIMITIVE}[1]{\mathcal{P}({#1})} % primitive name
\newcommand{\FUNCTION}[3]{\mathcal{T}({#1}, {#2}, {#3})} % environment, formal, code
%\newcommand{\CONTINUATION}[3]{\mathcal{K}({#1}, {#2}, {#3})} % environment, formal, code
\newcommand{\CONTINUATION}[1]{\mathcal{K}({#1})}
\newcommand{\FUTURE}[1]{\mathcal{T}({#1})} % token
\newcommand{\TASK}[1]{\mathcal{T}({#1})} % identifier

% Tasks:
%\def\TASKFONT{\mathsf}
%\def\TASKFONT{\mathfrak}
\newcommand{\BUSYTASK}[2]{{\TASKFONT{B}}({#1},{#2})} % environment, expression
\newcommand{\READYTASK}[1]{{\TASKFONT{R}}({#1})}
\newcommand{\DEADTASK}[0]{\TASKFONT{D}}
%\newcommand{\K}[3]{\mathcal{K}({#1}, {#2}, {#3})} % environment, formal, code

%% Stuff for writing rules:
\newcommand{\tsq}[0]{\vdash}
\newcommand{\ts}[0]{\ \vdash\ }
\newcommand{\tr}[0]{\ \rhd\ }
\newcommand{\arNS}[1]{\to_{#1}}
\newcommand{\ar}[1]{\ \arNS{#1}\ }
\newcommand{\NOAXIOM}[0]{\AxiomC{\NOTHING}}
\newcommand{\NOAXIOMQ}[0]{\AxiomC{}}
\newcommand{\tto}[0]{\ \twoheadrightarrow\ }
%%
\newcommand{\TWORULES}[2]
           {\begin{minipage}[b]{\linewidth}
               \begin{minipage}[b]{0.5\linewidth}
                 \centering {#1}
               \end{minipage}
               \begin{minipage}[b]{0.5\linewidth}
                 \centering {#2}
               \end{minipage}
             \end{minipage}}
\newcommand{\THREERULES}[3]
           {\begin{minipage}[b]{\linewidth}
               \begin{minipage}[b]{0.32\linewidth}
                 \centering {#1}
               \end{minipage}
               \begin{minipage}[b]{0.32\linewidth}
                 \centering {#2}
               \end{minipage}
               \begin{minipage}[b]{0.32\linewidth}
                 \centering {#3}
               \end{minipage}
             \end{minipage}}


\usepackage{aecompl}

% Links in pdf
\usepackage{color}
\definecolor{linkcol}{rgb}{0,0,0.4} 
\definecolor{citecol}{rgb}{0.5,0,0} 

% definitions.
% -------------------

\setcounter{secnumdepth}{3}
\setcounter{tocdepth}{2}

% Some useful commands and shortcut for maths:  partial derivative and stuff

\newcommand{\pd}[2]{\frac{\partial #1}{\partial #2}}
\def\abs{\operatorname{abs}}
\def\argmax{\operatornamewithlimits{arg\,max}}
\def\argmin{\operatornamewithlimits{arg\,min}}
\def\diag{\operatorname{Diag}}
\newcommand{\eqRef}[1]{(\ref{#1})}

\usepackage{rotating}                    % Sideways of figures & tables
%\usepackage{bibunits}
%\usepackage[sectionbib]{chapterbib}          % Cross-reference package (Natural BiB)
%\usepackage{natbib}                  % Put References at the end of each chapter
                                         % Do not put 'sectionbib' option here.
                                         % Sectionbib option in 'natbib' will do.

% \usepackage{txfonts}                     % Public Times New Roman text & math font
  
%\usepackage{algorithm}
%\usepackage[noend]{algorithmic}

%%% Clear Header %%%%%%%%%%%%%%%%%%%%%%%%%%%%%%%%%%%%%%%%%%%%%%%%%%%%%%%%%%%%%%%%%%
% Clear Header Style on the Last Empty Odd pages
\makeatletter

\def\cleardoublepage{\clearpage\if@twoside \ifodd\c@page\else%
  \hbox{}%
  \thispagestyle{empty}%              % Empty header styles
  \newpage%
  \if@twocolumn\hbox{}\newpage\fi\fi\fi}

\makeatother
 
%%%%%%%%%%%%%%%%%%%%%%%%%%%%%%%%%%%%%%%%%%%%%%%%%%%%%%%%%%%%%%%%%%%%%%%%%%%%%%% 
% Prints your review date and 'Draft Version' (From Josullvn, CS, CMU)
\newcommand{\reviewtimetoday}[2]{\special{!userdict begin
    /bop-hook{gsave 20 710 translate 45 rotate 0.8 setgray
      /Times-Roman findfont 12 scalefont setfont 0 0   moveto (#1) show
      0 -12 moveto (#2) show grestore}def end}}
% You can turn on or off this option.
% \reviewtimetoday{\today}{Draft Version}
%%%%%%%%%%%%%%%%%%%%%%%%%%%%%%%%%%%%%%%%%%%%%%%%%%%%%%%%%%%%%%%%%%%%%%%%%%%%%%% 

\newenvironment{maxime}[1]
{
\vspace*{0cm}
\hfill
\begin{minipage}{0.5\textwidth}%
%\rule[0.5ex]{\textwidth}{0.1mm}\\%
\hrulefill $\:$ {\bf #1}\\
%\vspace*{-0.25cm}
\it 
}%
{%

\hrulefill
\vspace*{0.5cm}%
\end{minipage}
}

\usepackage{multirow}
\usepackage{pdflscape}
\usepackage{amsfonts} % This interferes with mathbbol and they are not
                      % commutative, but I don't understand exactly what
                      % the interaction is; anyway I don't use amsfonts


\newenvironment{bulletList}%
{ \begin{list}%
	{$\bullet$}%
	{\setlength{\labelwidth}{25pt}%
	 \setlength{\leftmargin}{30pt}%
	 \setlength{\itemsep}{\parsep}}}%
{ \end{list} }

% centered page environment
\newenvironment{vcenterpage}
{\newpage\vspace*{\fill}\fancyhf{}\renewcommand{\headrulewidth}{0pt}}
{\vspace*{\fill}\par\pagebreak}

% My funny macros
\newcommand{\NEWLINE}[0]
  {\INVISIBLE{\tiny\tiny\tiny.}\\}
%% \lstset{language=Lisp,
%%         basicstyle=\tiny\ttfamily,
%%         numbers=left,
%%         stringstyle=\ttfamily,
%%         keywordstyle=\ttfamily,
%%         commentstyle=\ttfamily,
%%         identifierstyle=\ttfamily
%% }
%        numbers=left,
\lstset{language=Lisp,
        basicstyle=\small\ttfamily,
        numbers=none,
        stringstyle=\ttfamily,
        keywordstyle=\ttfamily,
        commentstyle=\ttfamily,
        identifierstyle=\ttfamily
}
%\def\NANOLISP{\textcolor{red}{nanolisp}}
\def\NANOLISP{nanolisp\xspace}
% My hackish support for grammars:
\newcommand{\SPACER}[0]{\\\INVISIBLE{\small{\tiny{hack!}}}}
%\newcommand{\SPACERF}[0]{\SPACER$\ \ $}
\newcommand{\SPACERP}[0]{\SPACER$|\ $}
\newcommand{\SPACERF}[0]{\SPACER\INVISIBLE{$|\ $}}
%% Example:
%% $E ::=$
%% \SPACERF$v$
%% \SPACERP$x$
%% \SPACERP$\LAMBDA{x}{E}$
%% \SPACERP$\APPLY{E_1}{E_2}$
%% \SPACERP$\IFIN{E_1}{\VALUES}{E_2}{E_3}$
%% \SPACERP$\PROMPT{E}$
%% \SPACERP$\CONTROL{x}{E}$

\def\CITE{\TODOQ{[Cite something]}}
\def\union{\cup}
\newcommand{\DISJOINTUNION}[0]{\uplus}

\newcommand{\EMPTYLIST}[0]{\texttt{()}}
%% \newcommand{\CONS}[2]{{\tt (}$#1${\tt\ .\ }$#2${\tt )}}
%% \newcommand{\SINGLETON}[1]{{\tt (}$#1${\tt )}}
\newcommand{\CONS}[2]{{\texttt{(}}{#1}{\texttt{.}}{#2}{\texttt{)}}}
%\newcommand{\CONS}[2]{${\tt (}${#1}${\tt \ .\ }${#2}${\tt )}$}
\newcommand{\SINGLETON}[1]{{\texttt{(}}#1{\texttt{)}}}
\newcommand{\SLIST}[1]{\texttt{(}#1\texttt{)}}

\newcommand{\ALAMBDA}[2]
  {[\lambda{#1}.{#2}]}
\newcommand{\AAPPLY}[2]
  {[{#1}\ @\ {#2}]}

\newcommand{\BUNDLENAME}[0]
  {\CODE{bundle}\xspace}
\newcommand{\LETNAME}[0]
  {\CODE{let}\xspace}
\newcommand{\PRIMITIVENAME}[0]
  {\CODE{primitive}\xspace}
\newcommand{\CALLNAME}[0]
  {\CODE{call}\xspace}
\newcommand{\CALLINDIRECTNAME}[0]
  {\CODE{call-{\allowbreak}indirect}\xspace}
\newcommand{\IFNAME}[0]
  {\CODE{if}\xspace}
\newcommand{\THENNAME}[0]
  {\CODE{then}\xspace}
\newcommand{\ELSENAME}[0]
  {\CODE{else}\xspace}
\newcommand{\FORKNAME}[0]
  {\CODE{fork}\xspace}
\newcommand{\JOINNAME}[0]
  {\CODE{join}\xspace}
\newcommand{\EXTENDEDNAME}[0]
  {\CODE{join}\xspace}
%% \newcommand{\DEFINENONPROCEDURENAME}[0]
%%   {\CODE{non-procedure}\xspace}%{\CODE{define-value}}
\newcommand{\DEFINEPROCEDURENAME}[0]
  {\CODE{procedure}\xspace}%{\CODE{define-procedure}}

%% \newcommand{\ACONSTANT}[1]
%%   {{#1}}
%% \newcommand{\AVARIABLE}[1]
%%   {{#1}}
%% \newcommand{\ABUNDLE}[1]
%%   {[\CODE{bundle}\ {#1}]}
%% \newcommand{\APRIMITIVE}[2]
%%   {[\pi\ {#1}\ {#2}]}
%% \newcommand{\ACALL}[2]
%%   {[{#1}\ {#2}]}
%% \newcommand{\AIFIN}[4]
%% %  {[{#1} \in {#2} \rightarrow {#3},\ {#4}]}
%%   {[\CODE{if}\ {#1} \in \{{#2}\}\ \CODE{then}\ {#3}\ \CODE{else}\ {#4}]}

%% \newcommand{\ALET}[3]
%%   {[{\tt let}\ {#1}\ {\tt be}\ {#2}\ {\tt in}\ {#3}]}
%% \newcommand{\AFORK}[2]
%%   {[\FORK{#1\ #2}]}
%% \newcommand{\AJOIN}[1]
%%   {[\JOIN{#1}]}
%% \newcommand{\AEXTENDED}[2]
%%   {[\chi\ {#1}\ {#2}]}
%% \newcommand{\ADEFINENONPROCEDURE}[2]
%%   {[\CODE{define-value}\ {#1}\ {#2}]} % \triangleq
%% \newcommand{\ADEFINEPROCEDURE}[3]
%%   {[\CODE{define-procedure}\ {\ACALL{#1}{#2}}\ {#3}]}

%% %% \newcommand{\FORK}[0]
%% %%   {\downpitchfork}
%% %% \newcommand{\JOIN}[0]
%% %%   {\uppitchfork}

%% \newcommand{\AFUTURE}[1]
%%   {[\CODE{fork} {#1}]}
%% \newcommand{\ATOUCH}[1]
%%   {[\CODE{join} {#1}]}

%% \newcommand{\ABOX}[1]
%%   {[box\ {#1}]}
%% \newcommand{\AUNBOX}[1]
%%   {[unbox\ {#1}]}
%% \newcommand{\ASETX}[2]
%%   {[set!\ {#1} {#2}]}
%% \newcommand{\ASET}[2]
%%   {\ASETX{#1}\ {#2}}

%% %\newcommand{\ASEQUENCE}[2]
%% %  {{#1};\ {#2}}

%% \newcommand{\ADEFINE}[2]
%%   {{#1} \triangleq {#2}}
%% \newcommand{\ADEFINEMACRO}[3]
%%   {\CONS{#1}{#2}\ \equiv\ #3}
%% %  {${\tt (}$#1${\tt \ .\ }$#2${\tt )}$\ \equiv\ #3}
%% \newcommand{\VALUEPLUS}[0]
%%   {v^{+}}
%% %% \newcommand{\APPLY}[2]
%% %%   {{#1}@{#2}}
%% \newcommand{\COMPOSE}[2]
%%   {{#1}\circ{#2}}
%% %\newcommand{\SEQUENCE}[2]
%% %  {{#1};{#2}}
%% \newcommand{\ALAMBDATWO}[3]
%%   {\ALAMBDA{#1}{\ALAMBDA{#2}{#3}}}
%% \newcommand{\ALAMBDATHREE}[4]
%%   {\ALAMBDA{#1}{\ALAMBDA{#2}{\ALAMBDA{#3}{#4}}}}
%% \newcommand{\ALAMBDAFOUR}[5]
%%   {\ALAMBDA{#1}{\ALAMBDA{#2}{\ALAMBDA{#3}{\ALAMBDA{#4}{#5}}}}}
%% \newcommand{\ASHIFT}[2]
%%   {[\xi{#1}.{#2}]}
%% \newcommand{\ARESET}[1]
%%   {[\langle{#1}\rangle]}
%% \newcommand{\ACONTROL}[2]
%%   {[{\mathcal{C}}{#1}.{#2}]}
%% \newcommand{\APROMPT}[1]
%%   {[\#{#1}]}

\newcommand{\BLACK}[1]
  {\textcolor{black}{#1}}
\newcommand{\RED}[1]
  {\textcolor{red}{#1}}
\newcommand{\PURPLE}[1]
  {\textcolor{purple}{#1}}
\newcommand{\BROWN}[1]
  {\textcolor{brown}{#1}}
\newcommand{\YELLOW}[1]
  {\textcolor{yellow}{#1}}
\newcommand{\BLUE}[1]
  {\textcolor{blue}{#1}}
\newcommand{\GREEN}[1]
  {\textcolor{darkgreen}{#1}}
\newcommand{\DEF}[0]
  {\stackrel{\text{\tiny def}}{=}}
\newcommand{\BELONGS}[0]
  {\ \epsilon\ }
\newcommand{\BELONGSREVERSE}[0]
  {\ \backepsilon\ }

\newcommand{\SEMANTICE}[1]
  {\LFATBRACKETS{E}{#1}}
\newcommand{\SEMANTICESTAR}[1]
  {\LFATBRACKETS{E^{*}}{#1}}
\newcommand{\SEMANTICS}[1]
  {\LFATBRACKETS{S}{#1}}
\newcommand{\SEMANTICSSTAR}[1]
  {\LFATBRACKETS{S^{*}}{#1}}
\newcommand{\SEMANTICT}[1]
  {\LFATBRACKETS{T}{#1}}
\newcommand{\SEMANTICTSTAR}[1]
  {\LFATBRACKETS{T^{*}}{#1}}
\newcommand{\SEMANTICPRIMITIVE}[2]
  {\LFATBRACKETS{P}{{#1}, {#2}}}
\newcommand{\SEMANTICAPPLY}[2]
  {\LFATBRACKETS{A}{#1, #2}}
\newcommand{\SEMANTICM}[1]
  {\LFATBRACKETS{M}{#1}}
\newcommand{\SEMANTICMADDEND}[2]
  {\mbox{$\mathcal{M}_{#1}[\![ #2 ]\!]$}}
\newcommand{\SEMANTICMACROCALL}[3]
  {\mbox{$\mathcal{MC}[\![ #1, #2, #3 ]\!]$}}
\newcommand{\POSSIBLY}[0]{{\textcolor{red}{[{\bf The thing will {\em possibly} be like this}]}}}
\newcommand{\EPSILON}[0]{$\varepsilon$\xspace}
\newcommand{\EPSILONZERO}[0]{$\varepsilon_{0}$\xspace}
\newcommand{\EPSILONONE}[0]{$\varepsilon_{1}$\xspace}
\newcommand{\EPSILONNOXSPACE}[0]{$\varepsilon$}
\newcommand{\EPSILONZERONOXSPACE}[0]{$\varepsilon_{0}$}
%\newcommand{\EPSILONZERO}[0]{$\varepsilon_{0}$}
\newcommand{\EPSILONZEROHOLE}[0]{$\varepsilon_{0}^{\HOLE}$\xspace}
\newcommand{\EPSILONMINUSONE}[0]{$\varepsilon_{-1}$\xspace}
\newcommand{\LAMBDA}[0]{$\lambda$}
\newcommand{\LAMBDACALCULUS}[0]{\mbox{$\lambda$-calculus}\xspace}
\newcommand{\PICALCULUS}[0]{\mbox{$\pi$-calculus}\xspace}

\newcommand{\HOLEDES}[0]{\ensuremath{\SET{E}_{\HOLE}}}

\newcommand{\iem}[1]{\index{#1}{\em{#1}}}
\newcommand{\textiti}[1]{\index{#1}{\textit{#1}}}
\newcommand{\ind}[1]{\index{#1}{#1}}
%\newcommand{\idef}[1]{\index{#1}{\sl{#1}}}
\newcommand{\idef}[1]{\index{#1}{\textit{#1}}}
\newcommand{\TDEF}[1]{\idef{#1}}

\newcommand{\QUOTATION}[3]
      {\begin{quotation}
        {\em #1}\\
        {\INVISIBLE{}\hfill---~{#2}, {#3}}
      \end{quotation}}
\newcommand{\LATER}[1]{\textcolor{blue}{[{\bf Later:} {#1}]}}
\newcommand{\UPDATE}[1]{\textcolor{purple}{[{\bf Update:} {#1}]}}

\newcommand{\TO}[0]
  {\ \to\ }
\newcommand{\TOENEW}[0]
%  {\ \longrightarrow_{\SET{E}}\ }
  {\Longrightarrow_{\SET{E}}}
\newcommand{\TOE}[0]
%  {\ \longrightarrow_{\SET{E}}\ }
  {\LINEBREAK\longrightarrow_{\SET{E}}\LINEBREAK}
\newcommand{\TOEP}[0]
%  {\ \longrightarrow_{\SET{E}}\ }
  {\LINEBREAK\longrightarrow^{+}_{\SET{E}}\LINEBREAK}
%% \newcommand{\TOET}[0]
%%   {\ \longrightarrow_{\SET{E}}^{+}\ }
%% \newcommand{\TOEST}[0]
%%   {\ \longrightarrow_{\SET{E}}^{\slashed{\parallel}+}\ }
%% \newcommand{\TOESRT}[0]
%%   {\ \longrightarrow_{\SET{E}}^{\slashed{\parallel}*}\ }
%% \newcommand{\TOERT}[0]
%%   {\ \longrightarrow_{\SET{E}}^{*}\ }
%% \newcommand{\TOES}[0]
%%   {\longrightarrow_{\SET{E}}^{\slashed{\parallel}}}
\newcommand{\TOET}[0]
  {\LINEBREAK\longrightarrow_{\SET{E}}^{+}\LINEBREAK}
\newcommand{\TOEST}[0]
  {\LINEBREAK\longrightarrow_{\SET{E}}^{\slashed{\parallel}+}\LINEBREAK}
\newcommand{\TOESRT}[0]
  {\LINEBREAK\longrightarrow_{\SET{E}}^{\slashed{\parallel}*}\LINEBREAK}
\newcommand{\TOERT}[0]
  {\LINEBREAK\longrightarrow_{\SET{E}}^{*}\LINEBREAK}
\newcommand{\TOES}[0]
  {\longrightarrow_{\SET{E}}^{\slashed{\parallel}}}

\newcommand{\REDUCES}[0]
  {\twoheadrightarrow_{\SET{E}}}
\newcommand{\DOESNOTREDUCE}[0]
  {\ensuremath{\mathrel{\slashed{\REDUCES}}}}
\newcommand{\TOEFT}[0]
  {\ \longrightarrow_{e,\FAILURE}^{*}\ }
\newcommand{\EVENTUALLYFAILSBECAUSEOF}[1]
  {\Downarrow_{\SET{E}}\FAILURE_{#1}}
%\newcommand{\DOESNTFAILBECAUSEOF}[1]
%  {{\slashed{\Downarrow}}_{\FAILUREONARROW}^{#1}}
\newcommand{\EVENTUALLYFAILSBECAUSEOFDIMENSION}[0]
  {\EVENTUALLYFAILSBECAUSEOF{\#}}
\newcommand{\DOESNTFAILBECAUSEOFDIMENSION}[0]
  {\DOESNTFAILBECAUSEOF{\#}}  
\newcommand{\EVENTUALLYFAILSBECAUSEOFPRIMITIVE}[0]
  {\EVENTUALLYFAILSBECAUSEOF{\SET{P}}}
\newcommand{\DOESNTFAILBECAUSEOFPRIMITIVE}[0]
  {\DOESNTFAILBECAUSEOF{\SET{P}}}  
\newcommand{\EVENTUALLYFAILSBECAUSEOFENVIRONMENTS}[0]
  {\EVENTUALLYFAILSBECAUSEOF{\SET{X}}}

\newcommand{\EVENTUALLYFAILSBECAUSEOFWITHRELATION}[2]
  {\Downarrow_{\SET{E}}^{#2}\FAILURE_{#1}}

\newcommand{\EVENTUALLYFAILSSBECAUSEOF}[1]
  {\Downarrow_{\SET{E}}^{\slashed{\parallel}}\FAILURE_{#1}}
\newcommand{\EVENTUALLYFAILSSBECAUSEOFDIMENSION}[0]
  {\EVENTUALLYFAILSSBECAUSEOF{\#}}
\newcommand{\EVENTUALLYFAILSSBECAUSEOFPRIMITIVE}[0]
  {\EVENTUALLYFAILSSBECAUSEOF{\SET{P}}}
\newcommand{\EVENTUALLYFAILSSBECAUSEOFENVIRONMENTS}[0]
  {\EVENTUALLYFAILSSBECAUSEOF{\SET{X}}}
%  {\EVENTUALLYFAILSBECAUSEOF{\textcolor{red}{\rho}}}
\newcommand{\EVENTUALLYFAILSS}[0]
  {\EVENTUALLYFAILSSBECAUSEOF{}}
\newcommand{\DOESNTFAILBECAUSEOFENVIRONMENTS}[0]
%  {\DOESNTFAILBECAUSEOF{\textcolor{red}{\rho}}}
  {\DOESNTFAILBECAUSEOF{\SET{X}}}
%\newcommand{\EVENTUALLYFAILSBECAUSEOFANOTHERTHREAD}[0]
%  {\EVENTUALLYFAILSBECAUSEOF{\parallel}}
%\newcommand{\DOESNTFAILBECAUSEOFANOTHERTHREAD}[0]
%  {\DOESNTFAILBECAUSEOF{\parallel}}
\newcommand{\EVENTUALLYFAILS}[0]
  {\EVENTUALLYFAILSBECAUSEOF{}}
\newcommand{\DOESNTFAIL}[0]
  {{\slashed{\Downarrow}_{\FAILUREONARROW}}}

\newcommand{\FAILUREBECAUSEOFDIMENSION}[0]
  {\FAILURE^{\#}}
\newcommand{\FAILUREBECAUSEOFPRIMITIVE}[0]
  {\FAILURE^{\Pi}}
\newcommand{\FAILUREBECAUSEOFENVIRONMENTS}[0]
  {\FAILURE^{\textcolor{red}{\rho}}}

\newcommand{\FAILURE}[0]
%  {\textcolor{red}{\divideontimes}}
%  {\textbf{E}}
%  {\maltese} 
%  {\text{$\skull$}}
%  {\text{\small$\skull$}} % It looks the same with and without \small
%  {\text{\tiny$\skull$}} 
%  {\frownie}
%  {\rip}
%  {\textrm{\tiny{\blitza}}}
%  {\frownie}
%  {\textrm{\lightning}}
%  {\textrm{\rm \textinterrobang}}
%  {\textit{\textinterrobang}}
%  {\textrm{\blitza}}
%  {\textrm{\rm \textreferencemark}}
  {\textrm{\rm \large \textreferencemark}}
\newcommand{\FAILUREONARROW}[0]
  {\FAILURE}
%  {\textcolor{red}{\divideontimes}}
%  {\textbf{E}}
%  {\maltese} 
% {\skull}
% {{}^{\text{\tiny$\skull$}}}
% {{}^{\textrm{\blitza}}}
%{^{^{\text{\blitza}}}}

\newcommand{\FAILSBECAUSEOF}[1]
%  {\longrightarrow_{\SET{E}}\FAILURE^{#1}}
  {\longrightarrow_{\SET{E}}\FAILURE_{#1}}
\newcommand{\DOESNTFAILBECAUSEOF}[1]
%  {\longrightarrow_{\SET{E}}\FAILURE^{#1}}
  {{\slashed{\longrightarrow}}_{\SET{E}}\FAILURE_{#1}}
\newcommand{\FAILSBECAUSEOFDIMENSION}[0]
  {\FAILSBECAUSEOF{\#}}
\newcommand{\FAILSBECAUSEOFPRIMITIVE}[0]
  {\FAILSBECAUSEOF{\SET{P}}}
\newcommand{\FAILSBECAUSEOFENVIRONMENTS}[0]
  {\FAILSBECAUSEOF{\SET{X}}}
%\newcommand{\FAILSBECAUSEOFANOTHERTHREAD}[0]
%  {\FAILSBECAUSEOF{\parallel}}
\newcommand{\FAILS}[0]
  {\FAILSBECAUSEOF{}}

\newcommand{\DIVERGESE}[0]
  {\Uparrow_{\SET{E}}}
%\newcommand{\DIVERGESES}[0]
%  {\Uparrow_{\SET{E}}^{\slashed{\parallel}}}
\newcommand{\DOESNTDIVERGE}[0]
  {{\slashed{\uparrow}}}
\newcommand{\CONVERGES}[0]
  {\Downarrow}
\newcommand{\DOESNTCONVERGE}[0]
  {{\slashed{\CONVERGES}}}
\newcommand{\CONVERGESP}[0]
  {\CONVERGES_{\SET{P}}}
\newcommand{\CONVERGEST}[0]
  {\CONVERGES_{\SET{T}}}
\newcommand{\CONVERGESPPROVISIONAL}[0]
  {\CONVERGES_{\SET{P}}^{\thicksim}}
\newcommand{\CONVERGESE}[0]
  {\CONVERGES_{\SET{E}}}
\newcommand{\CONVERGESI}[0]
  {\CONVERGES_{\SET{I}}}
\newcommand{\TOITILDE}[0]
  {\longrightarrow_{\SET{I}}^{_\thicksim}}
\newcommand{\TOPTILDE}[0]
  {\longrightarrow_{\SET{P}}^{_\thicksim}}
\newcommand{\CONVERGESITILDE}[0]
  {\CONVERGES^{\thicksim}_{\SET{I}}}
\newcommand{\CONVERGESES}[0]
  {\CONVERGES_{\SET{E}}^{\slashed{\parallel}}}
\newcommand{\CONVERGESSE}[0]
  {\CONVERGESES}
\newcommand{\DOESNTCONVERGEE}[0]
  {{\slashed{\CONVERGES}}_{\SET{E}}}
\newcommand{\NCONVERGESE}[0]
  {\DOESNTCONVERGEE}
\newcommand{\TOEE}[0]
  {\ \longrightarrow_{ee}\ }
%  {\ \to_{EE}\ }

\newcommand{\TOD}[0]
  {\longrightarrow_{\SET{D}}}
\newcommand{\TOT}[0]
  {\longrightarrow_{\SET{T}}}
\newcommand{\TOP}[0]
  {\longrightarrow_{\SET{P}}}
\newcommand{\TOV}[0]
  {\longrightarrow_{\SET{V}}}
\newcommand{\TOTPROVISIONAL}[0]
  {\xrightarrow{provisional}_{\SET{T}}}


\newcommand{\TOM}[0]
  {\to_M}
\newcommand{\TOBIG}[0]
  {\downarrow}
\newcommand{\TOBIGE}[0]
  {\TOBIG_E}
\newcommand{\TOBIGT}[0]
  {\TOBIG_T}
\newcommand{\TOBIGP}[0]
  {\TOBIG_P}
\newcommand{\TOBIGM}[0]
  {\TOBIG_M}
\newcommand{\PARALLELCONFIGURATION}[2]
  {{#1} \rhd {#2}}
\newcommand{\PCONF}[2]
  {\PARALLELCONFIGURATION{#1}{#2}}
%\newcommand{\REWRITE}[3]
%  {{#1}\ \vdash\ {#2}\ \TOE\ {#3}}
\newcommand{\REWRITEP}[5]
%  {{\PCONF{#1}{#2}}\ \TOE\ {\PCONF{#3}{#4}}}
  {{#1}\ \vdash\ {\PCONF{#2}{#3}}\ \TOE\ {\PCONF{#4}{#5}}}
\newcommand{\REWRITEBIGE}[5]
  {{#1}\ \vdash\ \PCONF{#2}{#3}\ \TOBIGE\ \PCONF{#4}{#5}}
\newcommand{\REWRITEM}[4]
  {\PCONF{#1}{#2}\ \TOM\ \PCONF{#3}{#4}}
%% \newcommand{\STATE}[1]
%%   {\STATEQ_{#1}}
%% \newcommand{\STATEDEFAULT}[0]
%%   {\STATE{\Gamma, \Pi, M}}
%% \newcommand{\STATEQ}[0]
%%   {\Sigma}

\newcommand{\FUTURES}[0]
  {\Phi}

\newcommand{\RULETWO}[5]
  {\begin{prooftree}
      \LeftLabel{$[#1]$}
      \RightLabel{$#5$}
      \AxiomC{$#2$}
      \AxiomC{$#3$}
      \BinaryInfC{$#4$}
  \end{prooftree}}

\newcommand{\RULETHREE}[6]
  {\begin{prooftree}
      \LeftLabel{$[#1]$}
      \RightLabel{$#6$}
      \AxiomC{$#2$}
      \AxiomC{$#3$}
      \AxiomC{$#4$}
      \TrinaryInfC{$#5$}
  \end{prooftree}}

\newcommand{\RULEONE}[4]
  {\begin{prooftree}
      \LeftLabel{$[#1]$}
      \RightLabel{$#4$}
      \AxiomC{$#2$}
      \UnaryInfC{$#3$}
  \end{prooftree}}
  
\newcommand{\RULEZERO}[3]
  {\RULEONE{#1}{\INVISIBLE{hack}}{#2}{#3}}

\newcommand{\SET}[1]
  {\mathbb{#1}}
\newcommand{\VALUES}[0]
  {\SET{V}}
\newcommand{\NATURALS}{\SET{N}}
\newcommand{\INTEGERS}{\SET{Z}}
\newcommand{\RATIONALS}{\SET{Q}}
\newcommand{\REALS}{\SET{R}}
\newcommand{\SYMBOLS}{\SET{S}}

\newcommand{\UNION}{\cup}
\newcommand{\INTERSECTION}{\cap}

\newcommand{\CLOSURE}[3]
  {\mathcal{C}({#1}, {#2}, {#3})}

\newcommand{\NONVALUE}[1]
  {{#1} \notin \VALUES}
%\newcommand{\VALUE}[1] {{#1} \in \VALUES}

\newcommand{\BULLET}[0]
  {\bullet}
\newcommand{\SECTION}[0]
  {§}
\newcommand{\SECTIONS}[0]
  {§§}
\newcommand{\SOMEREF}[0]
  {\textcolor{red}{\textbf{[\SECTION???]}}}
\newcommand{\REFSOMETHING}[0]
  {\SOMEREF}

%\newcommand{\UNFILLEDUNPOSITIONEDQED}{\ensuremath{\Box}}
\newcommand{\UNFILLEDUNPOSITIONEDQED}{\ensuremath{\square}}
\newcommand{\FILLEDUNPOSITIONEDQED}{\ensuremath{\blacksquare}}
%\newcommand{\FILLEDUNPOSITIONEDQED}{\ensuremath{\filledsquare}}

%\newcommand{\qed}{\hfill \ensuremath{\Box}}
\newcommand{\UNFILLEDQED}{\hfill \ensuremath{\UNFILLEDUNPOSITIONEDQED}}
\newcommand{\FILLEDQED}{\hfill \ensuremath{\FILLEDUNPOSITIONEDQED}}

\newcommand{\EXAMPLEQED}{\UNFILLEDQED}
\newcommand{\PROOFQED}{\FILLEDQED}
\newcommand{\NOPROOF}{\PROOFQED}
\newcommand{\NOPROOFQED}{\NOPROOF}

\newcommand{\INJECT}{inject}
\newcommand{\EJECT}{eject} % Christophe didn't like this name. To do: find the name he
                           % liked; I have it on paper

%\newcommand{\CODE}[1]{\verb!{#1}!}
\newcommand{\CODE}[1]{{\rm \texttt{#1}}}
\newcommand{\FILE}[1]{\texttt{#1}}
%\newcommand{\UNIT}[0]{$\square$}
\newcommand{\UNIT}[0]{$\bullet$}
%\newcommand{\UNIT}[0]{$\star$}
\newcommand{\SYMBOL}[1]{\CODE{\#s-}{#1}}
\newcommand{\SYMBOLTEXT}[1]{\CODE{\#s-#1}}
\newcommand{\SYMBOLMETA}[1]{\CODE{\#s-{\ensuremath{#1}}}}

%\newcommand{\HASDIMENSION}{:_d}
\newcommand{\HASDIMENSION}{:_{\#}}

\newcommand{\LINEBREAK}[0]
  {\allowbreak}
%  {\linebreak[1]{\allowbreak{}}}
%{}
%%%%
\newcommand{\DHANDLE}[2]
  {\LINEBREAK{#2}_{#1}}
\newcommand{\DCONSTANT}[2]
  {\LINEBREAK{#2}_{#1}}
\newcommand{\DVARIABLE}[2]
  {\LINEBREAK{#2}_{#1}}
\newcommand{\DBUNDLE}[2]
  {\LINEBREAK[\BUNDLENAME\xspace\ \LINEBREAK{#2}]_{#1}}
\newcommand{\DPRIMITIVE}[3]
  {\LINEBREAK[\PRIMITIVENAME\ \LINEBREAK{{\tt #2}}\ \LINEBREAK{#3}]_{#1}}
\newcommand{\DCALL}[3]
%  {\LINEBREAK[\CALLNAME\ \LINEBREAK{{\tt \CODE{#2}}}\ \xspace\LINEBREAK{#3}]_{#1}}
  {\LINEBREAK[\CALLNAME\ \LINEBREAK{{{#2}}}\ \xspace\LINEBREAK{#3}]_{#1}}
\newcommand{\DCALLINDIRECT}[3]
%  {\LINEBREAK[\CALLINDIRECTNAME\ \LINEBREAK{{\tt \CODE{#2}}}\ \xspace\LINEBREAK{#3}]_{#1}}
  {\LINEBREAK[\CALLINDIRECTNAME\ \LINEBREAK{{{#2}}}\ \xspace\LINEBREAK{#3}]_{#1}}
\newcommand{\DIFIN}[5]
%  {[{#2} \in {#3} \rightarrow {#4},\ {#5}]_{#1}}
  {\LINEBREAK[\IFNAME\ \LINEBREAK{#2} \LINEBREAK\in \LINEBREAK\{{#3}\}\ \LINEBREAK\CODE{then}\ \LINEBREAK{#4}\ \LINEBREAK\CODE{else}\ \LINEBREAK{#5}]_{#1}}
\newcommand{\DLET}[4]
  {\LINEBREAK[\LETNAME\ \LINEBREAK{#2}\ \LINEBREAK{\tt be}\ \LINEBREAK{#3}\ \LINEBREAK{\tt in}\ \LINEBREAK{#4}]_{#1}}
\newcommand{\DFORK}[3]
  {\LINEBREAK[\FORKNAME\ \LINEBREAK{#2}\ \LINEBREAK{#3}]_{#1}}
\newcommand{\DJOIN}[2]
  {\LINEBREAK[\JOINNAME\ \LINEBREAK{#2}]_{#1}}
\newcommand{\DEXTENDED}[3]
  {\LINEBREAK[\EXTENDEDNAME\ \LINEBREAK{#2}\ \LINEBREAK{#3}]_{#1}}
\newcommand{\DDEFINENONPROCEDURE}[3]
%  {\LINEBREAK[\DEFINENONPROCEDURENAME\ \LINEBREAK{{\tt #2}}\ \LINEBREAK{#3}]_{#1}}
  {\LINEBREAK[\DEFINENONPROCEDURENAME\ \LINEBREAK{#2}\ \LINEBREAK{#3}]_{#1}}
\newcommand{\DDEFINEPROCEDURE}[3]
%  {\LINEBREAK[\DEFINEPROCEDURENAME\CODE{\ \LINEBREAK(}{{\tt \CODE{#2}}\ \xspace\LINEBREAK{#3}}\CODE{)\ \LINEBREAK}{#4}]_{#1}}
  {\LINEBREAK[\DEFINEPROCEDURENAME\CODE{\ \LINEBREAK(}{{{#1}}\ \xspace\LINEBREAK{#2}}\CODE{)\ \LINEBREAK}{#3}]}
\newcommand{\DCALLWITHHOLE}[2]
  {\LINEBREAK[\CODE{call}\ {#2}\ \LINEBREAK{\HOLE}]_{#1}}
\newcommand{\DFORKWITHHOLE}[2]
  {\LINEBREAK[\FORKNAME\ \LINEBREAK{#2}\ \LINEBREAK\HOLE]_{#1}}
\newcommand{\DBUNDLEWITHHOLE}[1]
  {\LINEBREAK[\BUNDLENAME\ \LINEBREAK\HOLE]_{#1}}
\newcommand{\DPRIMITIVEWITHHOLE}[2]
  {\LINEBREAK[\PRIMITIVENAME\ \LINEBREAK{#2}\ \LINEBREAK\HOLE]_{#1}}

\newcommand{\AHOLE}[0]
  {\bullet}

\newcommand{\SEQUENCE}[1]
  {\langle {#1} \rangle}
%  {\text{$\langle${}${#1}${}$\rangle$}}
\newcommand{\EMPTYSEQUENCE}[0]
  {\SEQUENCE{}}

\def\DRSH{\ensuremath{\mathrel{\slashed{\drsh}}}}
\def\NDRSH{\ensuremath{\mathrel{\slashed{\drsh}}}}
\newcommand{\ISTAIL}[1]
  {{#1}\DRSH}
%  {{#1}\Rsh}
\newcommand{\ISTAILFOR}[2]
  {\ISTAIL{#1}{#2}}
\newcommand{\ISNOTTAIL}[1]
  {{#1}\NDRSH}

\newcommand{\DIMENSION}[1]
  {\mbox{$d \ldbrack {#1}\rdbrack$}}

\newcommand{\MEET}[0]
  {\wedge}
\newcommand{\JOIN}[0]
  {\vee}
\newcommand{\SQMEET}[0]
  {\sqcap}
\newcommand{\SQJOIN}[0]
  {\sqcup}
\newcommand{\BIGSQJOIN}[0]
  {\bigsqcup}
\newcommand{\BIGSQMEET}[0]
  {\bigsqcap}
%% \newcommand{\BIGMEET}[2]
%%   {\bigwedge{#2}}
%% \newcommand{\BIGJOIN}[2]
%%   {\bigvee{#2}}

\newcommand{\SUCHTHAT}[0]
%  {~|~}
%  {~.~}
  {:}

\newcommand{\SQL}[0]
  {\sqsubset}
\newcommand{\SQLE}[0]
  {\sqsubseteq}
\newcommand{\SQGE}[0]
  {\sqsupseteq}
\newcommand{\NSQL}[0]
  {\nsqsubset}
\newcommand{\NSQLE}[0]
  {\nsqsubseteq}

\newcommand{\NATURALSB}[0]
  {\NATURALS_{\bot}}
\newcommand{\NATURALSTB}[0]
  {\NATURALS_{\bot}^{\top}}
\newcommand{\NATURALSBT}[0]
  {\NATURALSTB}

\newcommand{\EXPRESSIONS}{\SET{E}}
\newcommand{\PROGRAMS}{\SET{P}}
\newcommand{\CONSTANTS}{\SET{C}}
\newcommand{\STATESSET}{\SET{\Gamma}}
%\newcommand{\PROCEDURES}{\ensuremath{Proc}}
\newcommand{\PROCEDURES}{\SET{F}}

\newcommand{\PARTIAL}{\rightharpoonup}

\newcommand{\HOLE}{\Box}

% A way of superimposing two symbols, in Math mode.  I took the general idea
% from ``The LaTeX comprehensive symbol list'' and generalized:
\newcommand*{\SUPERIMPOSE}[2]
            {\mathrel{\mathchoice{\hbox{\hbox to 0pt{$\displaystyle{#1}$\hss}$\displaystyle{#2}$}}
                {\hbox{\hbox to 0pt{$\textstyle{#1}$\hss}$\textstyle{#2}$}}
                {\hbox{\hbox to 0pt{$\scriptstyle{#1}$\hss}$\scriptstyle{#2}$}}
                {\hbox{\hbox to 0pt{$\scriptscriptstyle{#1}$\hss}$\scriptscriptstyle{#2}$}}}}

\newcommand{\OCCURS}{\SUPERIMPOSE{\Subset}{-}}

\newcommand{\COLOREDMINIPAGE}[2]
  {\textcolor{#1}{
     \begin{minipage}{\linewidth}
       #2
     \end{minipage}}}

\newcommand{\PARAMETER}[0]
  {\mathunderscore}

\newcommand{\xqed}[1]{%
  \leavevmode\unskip\penalty9999 \hbox{}\nobreak\hfill
  \quad\hbox{\ensuremath{#1}}}
\def\QEDDEFINITION{\xqed{\UNFILLEDUNPOSITIONEDQED}}
%\def\QEDNOPROOF{\xqed{\blacksquare}} % To do: it should be smaller
\def\QEDNOPROOF{\xqed{\UNFILLEDUNPOSITIONEDQED}}
\def\QEDPROOF{\xqed{\FILLEDUNPOSITIONEDQED}}
\def\QEDAXIOM{\xqed{\UNFILLEDUNPOSITIONEDQED}}
\def\QEDEXAMPLE{\xqed{\Diamond}}
\def\QEDIMPLEMENTATIONNOTE{\QEDDEFINITION}
\def\QEDSYNTACTICCONVENTION{\QEDIMPLEMENTATIONNOTE}

\newcommand{\EMPTYSET}[0]{\varnothing}
%\newcommand{\EMPTYSET}[0]{\emptyset}
%\newcommand{\EMPTYSTACK}[0]{\EMPTYSET}
\newcommand{\EMPTYSTACK}[0]{\EMPTYSEQUENCE}

%\newcommand{\VALUESEPARATOR}[0]{!}
\newcommand{\VALUESEPARATOR}[0]{\text{\ensuremath{\wr}}} % This hack suppresses the spaces around \wr
%\newcommand{\ACTIVATIONSEPARATOR}[0]{?}
%\newcommand{\VALUESEPARATOR}[0]{\ensuremath{\wr}}
%\newcommand{\VALUESEPARATOR}[0]{\ensuremath{\dagger}}
\newcommand{\ACTIVATIONSEPARATOR}[0]{\ensuremath{\ddagger}}

%% \newcommand{\VALUESEPARATOR}[0]{\ensuremath{\lrcorner}}
%% \newcommand{\ACTIVATIONSEPARATOR}[0]{\ensuremath{\urcorner}}

\newcommand{\UNIFY}[0]
  {\equiv}
\newcommand{\UNIFIES}[0]
  {\UNIFY}
\newcommand{\DOESNTUNIFY}[0]
  {\ensuremath{\mathrel{\slashed{\equiv}}}}
  %{\nequiv}
\newcommand{\DOESNOTUNIFY}
  {\DOESNTUNIFY}

\newcommand{\SIMULATES}
  {\thickapprox}
%  {\thicksim}

\newcommand{\REDUCTIONEQUIVALENT}[0]
  {\equiv_{\EXPRESSIONS}}
\newcommand{\FILLBYHAND}[0]
  {\INVISIBLE{a\\\\a}}

\newcommand{\STATE}
  {state\xspace}
%  {context\xspace}
\newcommand{\STATES}
  {states\xspace}
%  {contexts\xspace}
\newcommand{\SSTATE}
  {State\xspace}
%  {Context\xspace}
\newcommand{\SSTATES}
  {States\xspace}
%  {Contexts\xspace}
\newcommand{\STATEKEY}
  {state key\xspace}
%  {context\xspace}
\newcommand{\STATEKEYS}
  {state keys\xspace}
%  {contexts\xspace}
\newcommand{\SSTATEKEY}
  {State key\xspace}
%  {Context\xspace}
\newcommand{\SSTATEKEYS}
  {State keys\xspace}
%  {Contexts\xspace}

\newcommand{\UPDATEENVIRONMENT}[3]
  {{#1}[{#2} \mapsto {#3}]}
\newcommand{\UPDATESTATE}[3]
  {{#1}[{}_{#2}^{#3}]}
\newcommand{\UPDATESTATEIN}[4]
  {\UPDATESTATE{#1}{#2}{{#3}~\mapsto~{#4}}}

%% % I want my unnumbered chapters to appear in the contents:
%% \newcommand{\UNNUMBEREDCHAPTER}[1]{\chapter*{#1}\addcontentsline{toc}{chapter}{\protect\numberline{}#1}}

%% % I want my unnumbered sections to appear in the contents.  They should be
%% % listed as chapters in the ToC, since they are independent top-level units.
%% %\newcommand{\UNNUMBEREDSECTION}[1]{\section*{#1}\addcontentsline{toc}{chapter}{\protect\numberline{}#1}}
%% \newcommand{\UNNUMBEREDSECTION}[1]{\section*{#1}\addcontentsline{toc}{section}{\protect\numberline{}#1}}
%% \newcommand{\UNNUMBEREDSUBSECTION}[1]{\subsection*{#1}\addcontentsline{toc}{subsection}{\protect\numberline{}#1}}
%% \newcommand{\UNNUMBEREDSUBSUBSECTION}[1]{\subsubsection*{#1}\addcontentsline{toc}{subsubsection}{\protect\numberline{}#1}}

% This use of \addstarredchapter is needed not to interfere with minitocs.
% I want my unnumbered chapters to appear in the contents:
%\newcommand{\UNNUMBEREDCHAPTER}[1]{\chapter*{#1}\addstarredchapter{\protect\numberline{}#1}}
\newcommand{\UNNUMBEREDCHAPTER}[1]{\chapter*{#1}\addstarredchapter{#1}}

% This use of \addcontentsline is needed not to interfere with minitocs.
% I want my unnumbered sections to appear in the contents.  They should be
% listed as chapters in the ToC, since they are independent top-level units.
\newcommand{\UNNUMBEREDSECTION}[1]{\section*{#1}\addcontentsline{toc}{section}{\protect\numberline{}#1}}

% I like verbatim environments to always have small fonts:
%% \makeatletter 
%% \g@addto@macro\@verbatim\small
%% \makeatother 
%\fvset{gobble=2, fontsize=\tiny} % For fancyvrb
\newcommand{\EPSILONGC}[0]{\CODE{epsilongc}\xspace}

\newcommand{\FATBRACKETSWITH}[2]
%  {\mbox{${#1}[\![ {#2} ]\!]$}}
%  {E_{\SET{#1}}{[{#2}]}}
%  {E_{\SET{#1}}{({#2})}}
%  {E_{\SET{#1}}{\Lbag{#2}\Rbag}}
%  {E_{\SET{#1}}{\lbag{{#2}}\rbag}}
  {E_{\SET{#1}}\BANANABRACKETS{#2}}
%  {E_{\SET{#1}}{\lcorners{#2}\rcorners}}
%  {E_{\SET{#1}}{\llcorner{#2}\lrcorner}}
%  {E_{\SET{#1}}{\ulcorner{#2}\urcorner}}
%  {E_{\SET{#1}}{\llfloor{#2}\rrfloor}}
\newcommand{\LOTSA}[1]
%  {\overline{#1}}
%  {{#1}^{*}}
  {{#1}s}
\newcommand{\EXPANDE}[1]{\FATBRACKETSWITH{E}{#1}}
\newcommand{\EXPANDX}[1]{\FATBRACKETSWITH{X}{#1}}
\newcommand{\EXPANDCS}[1]{\FATBRACKETSWITH{\LOTSA{C}}{#1}}
\newcommand{\EXPANDXS}[1]{\FATBRACKETSWITH{\LOTSA{X}}{#1}}
\newcommand{\EXPANDES}[1]{\FATBRACKETSWITH{\LOTSA{E}}{#1}}

\newcommand{\BOOTSTRAPPHASE}[1]{{\em ({#1})}}
%\newcommand{\WHATEVER}[1]{\ensuremath{\overline{\CODE{#1}}}}
\newcommand{\WHATEVER}[1]{\CODE{#1}} % The definition above was a bad idea

\newcommand{\BETWEENTINYANDSMALL}{\fontsize{9}{11}\selectfont}
\newcommand{\LARGERTHANSMALL}{\fontsize{12}{14}\selectfont}

\usepackage{censor}

% I don't want page number on ``Part'' pages:
\makeatletter
\renewcommand\part{%
  \if@openright
\cleardoublepage
  \else
\clearpage
  \fi
  \thispagestyle{empty}%
  \if@twocolumn
\onecolumn
\@tempswatrue
  \else
\@tempswafalse
  \fi
  \null\vfil
  \secdef\@part\@spart}
\makeatother

\hypersetup
{
  breaklinks=true,  % so long urls are correctly broken across lines
%  colorlinks=true,
  hidelinks=true, % Even if those squares are useful, I hate to see them --Luca Saiu.
    %%   urlcolor=blue,
    %%   linkcolor=darkorange,
    %%   citecolor=darkgreen,
  %% bookmarksopen=true,
  %pdftitle={\SHORTTITLE},
  %pdftitle={\TTITLE},
  pdftitle={\vcsn \vcsnversion},
  pdfauthor="{Luca SAIU, Akim DEMAILLE, Jacques SAKAROVITCH et al.}", 
  pdfsubject={\vcsn \vcsnversion} %subject of the document
  %% pdfhighlight=/O, %effect of clicking on a link
  %% colorlinks=true, %couleurs sur les liens hypertextes
  %% %% %colorlinks=false, %couleurs sur les liens hypertextes
  %% pdfpagemode=None,
  %% pdfpagelayout=SinglePage,
  %% pdffitwindow=true,
  %% linkcolor=linkcol, %couleur des liens hypertextes internes
  %% citecolor=citecol, %couleur des liens pour les citations
  %% urlcolor=linkcol %couleur des liens pour les url
}
