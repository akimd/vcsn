% -*- mode: latex; fill-column: 79; mode: auto-fill; mode: flyspell; buffer-file-coding-system: utf-8 -*-
\index{documentation}
\index{manual}
\index{manual!adding to}
The sources of this document reside in the \FILE{doc/manual/sources}
subdirectory of Vaucanson archives.

\index{\LaTeX{}}
The main \LaTeX{} source file is
\FILE{doc/manual/sources/vaucanson-manual.tex}, which in its turn
\verb!\input!s other files --- usually, but not necessarily, one per
chapter.

\section{Including IPython notebooks}
The documentation build system includes a facility automatically
converting IPython notebook files into \LaTeX{} documents, which is
particularly useful for chapters or sections containing many examples:
for them to be accessible the user only needs to include notebook
files (or links to them) in the \FILE{doc/manual/sources/notebooks}
subdirectory.  The build system will convert each notebook into a
\LaTeX{} file, on which other \LaTeX{} files can freely use
\verb!\input! or \verb!\include!.

In order to be more friendly to \LaTeX (and Unix), an IPython notebook
file with extension \FILE{.ipynb} is automatically translated into a
\LaTeX{} file with extension \FILE{.tex}, \textit{and with every space
  and underscore in its name replaced by a dash}.

For example \FILE{doc/manual/sources/notebooks/} contains a symbolic
link named \FILE{Graph~operations.ipynb}, making up the source of
§\ref{graph-operations}: that is automatically translated into
\FILE{Graph-operations.tex}; the file happens to be used directly from
the main documentation source \FILE{vaucanson-manual.tex}, which
contains the following line:
\begin{verbatim}
\input{Graph-operations}
\end{verbatim}
As shown in the line above, the user should not specify a directory
when referring a translated notebook file.

\subsection{IPython notebook markup extension}
A notebook to be translated into \LaTeX{} will usually consist in some
combination of Python expressions with their evaluated results, and
\index{Markdown}
text with \textit{Markdown}\footnote{An informal specification is
  available at \url{http://daringfireball.net/projects/markdown/}.
  The Wikipedia page \url{http://en.wikipedia.org/wiki/Markdown}
  contains some further information.}
markup.

We implemented a minor extension to the Markdown language by allowing
the user to add an \textit{index entry} for the manual in Markdown
comments starting with a capital \CODE{I} character, not preceded by
spaces.

For example a notebook could contain the following Markdown text in a cell:
\begin{verbatim}
<!--I determinization -->
<!--I determinization!weighted -->
Determinization of weighted automata is not always possible.
\end{verbatim}
which will generate the \LaTeX{} code:
\begin{verbatim}
\index{determinization}
\index{determinization!weighted}
Determinization of weighted automata is not always possible.
\end{verbatim}

The included index entry is allowed to contain arbitrary \LaTeX{}
text, math and macro calls, as in:
\begin{verbatim}
\input{<!--I $\pi$ -->}
\end{verbatim}

\index{script}
\index{compile-manual}
\index{\CODE{nbconvert}}
\index{\CODE{ipython}!\CODE{nbconvert}}
The bulk of the translation from IPython notebooks to \LaTeX{} is implemented
in the \CODE{ipython nbconvert} tool.  The Vaucanson source code
includes the script \FILE{doc/manual/compile-manual} to automatically
invoke it and then patch its output.  As the script is reasonably simple
and well-commented it can be extended to support more comment
extensions.

\section{Building}
\index{make}
The PDF manual is generated by the \CODE{manual} target: assuming that
Vaucanson has been configured to be built in its source directory as
recommended in §\ref{assume-we-build-in-the-source-directory}, the
following command line from the source directory will regenerate it:\
\begin{verbatim}
$ make manual
\end{verbatim}
The subdirectory \FILE{doc/manual/scratch} is used for temporary
files.

\index{build directory}
\index{source directory}
\index{make}
Documentation is currently generated under the \textit{source}
directory, which will be noticeable to users configuring from a
different \textit{build} directory.
