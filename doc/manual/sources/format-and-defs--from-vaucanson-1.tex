% -*- mode: latex; fill-column: 79; mode: auto-fill; mode: flyspell; buffer-file-coding-system: utf-8 -*-

%% This comes from the Vaucanson 1 documentation, with very few changes.
%% It will help us to share LaTeX fragments.

%%% flag definition
% draft controls  frames, labels showing, footer, comments, etc.
\newif\ifdraft \draftfalse % \drafttrue  %
\newif\ifital \italfalse %\italtrue
\newcommand{\JSLandscape}%
   {\ifital \thispagestyle{empty}\mbox{ }\clearpage%
       \addtocounter{page}{-1}\fi}
\newif\iflong \longfalse   % \longtrue  %
\newcommand{\longonly}[1]{\iflong{#1}\else{}\fi}
\newcommand{\shortlong}[2]{\iflong{#2}\else{#1}\fi}
\newcommand{\longshort}[2]{\iflong{#1}\else{#2}\fi}
\newcommand{\shortonly}[1]{\iflong{}\else{#1}\fi}
\newcommand{\longclear}{\iflong{\clearpage}\else{}\fi}
\newcommand{\shortclear}{\iflong{}\else{\clearpage}\fi}

\usepackage{calc}
\newcounter{hours}\newcounter{minutes}
\newcommand\printtime{\setcounter{hours}{\time/60}%
  \setcounter{minutes}{\time-\value{hours}*60}%
  \thehours\,h\,\theminutes}
\newcommand{\writedate}{}
\newcounter{heure}\setcounter{heure}{\time}%
\makeatletter
\def\@oddfoot {{\footnotesize
%\longshort{\makebox[0pt][l]{\texttt{Version with commments}}}%
%          {\makebox[0pt][l]{\texttt{\vcsnv \tafkit Documentation}}}
% \longshort{\makebox[0pt][l]{\texttt{Version FSMNLP 2011 with commments}}}%
%           {\makebox[0pt][l]{\texttt{FSMNLP 2011 Special Edition}}}
\hfil -- {\normalsize \thepage} -- \hfil
%\longshort{\makebox[0pt][r]{\texttt{for the \vcsn Group}}}%
%          {\makebox[0pt][r]{\texttt{\today}}}%
}}
\def\@evenfoot{{\footnotesize
%\longshort{\makebox[0pt][l]{\texttt{Working document --- Do not circulate}}}%
%          {\makebox[0pt][l]{\texttt{\vcsnv \tafkit Documentation}}}
% \longshort{\makebox[0pt][l]{\texttt{Working document --- Do not circulate}}}%
%           {\makebox[0pt][l]{\texttt{FSMNLP 2011 Special Edition}}}
\hfil -- {\normalsize \thepage} -- \hfil
%\longshort{\makebox[0pt][r]{\texttt{Compiled \today \e at \printtime}}}%\theheure
%          {\makebox[0pt][r]{\texttt{\today}}}%
}}
\makeatother
\addtolength{\footskip}{20pt}
%%%%%%%%%%%%%%%%%%%%%%%%%%%%%%%%%%%%%%%%%%%%
%\usepackage[utf8]{inputenc}
%\input{texinputs/accent_keys}
\usepackage{a4wide}
\usepackage[american]{babel}
\usepackage{amsmath,amsfonts,amscd,amssymb}
\usepackage{xspace}
\usepackage{textcomp}
\usepackage{tabularx}
% \usepackage{url}
\usepackage{graphicx}
\usepackage{vaucanson-g}
\usepackage{subfigure}
% \usepackage{showlabels}
\usepackage{makeidx}
%%%%%%%%%%%%%%%%%%%%%%%%%%%%%%%%%%%%%%%%%%%%
%\input{texinputs/js_symboles3}% symboles macros
%\input{texinputs/js_macros3}% symboles macros
\newcommand{\shuffle}{\mathbin{\between}}
\newcommand{\autplus}{+}
\newcommand{\autprod}{\cdot}%{\curvearrowright}
\newcommand{\autstarsymb}{*}%{\circlearrowleft}
\newcommand{\autstar}[1]{#1^\autstarsymb}
\newcommand{\lgt}[1]{\left|#1\right|}
\newcommand{\lbl}[1]{\ell(#1)}
\newcommand{\wgt}[1]{w(#1)}
%%%%%%%%%%%%%\section{environments}
\newtheorem{theorem}{Theorem}
\newtheorem{property}{Property}

%%%%%%%%%%%%%\section{references}
\usepackage{hyperref}
  \hypersetup{breaklinks, plainpages=false}
  \hypersetup{colorlinks, citecolor=blue, linkcolor=blue, urlcolor=blue}
\usepackage[open,numbered]{bookmark}

%% cleveref must be loaded after hyperref.
\usepackage[nameinlink]{cleveref}
  \crefname{appendix}{Appendix}{Appendice}
  \crefname{chapter}{Chapter}{Chapters}
  \crefname{corollary}{Corollary}{Corollaries}
  \crefname{definition}{Definition}{Definitions}
  \crefname{example}{Example}{Examples}
  \crefname{exercise}{Exercise}{Exercises}
  \crefname{fact}{Fact}{Facts}
  \crefname{figure}{Figure}{Figures}
  \crefname{lemma}{Lemma}{Lemmas}
  \crefname{note}{Note}{Notes}
  \crefname{problem}{Problem}{Problems}
  \crefname{property}{Property}{Properties}
  \crefname{proposition}{Proposition}{Propositions}
  \crefname{remark}{Remark}{Remarks}
  \crefname{section}{Section}{Sections}
  \crefname{table}{Table}{Tables}
  \crefname{theorem}{Theorem}{Theorems}
  \newcommand{\apndx}[1]{\cref{chp:#1}}%
  \newcommand{\appnd}[1]{\cref{app:#1}}%
  \newcommand{\chptr}[1]{\cref{chp:#1}}%
  \newcommand{\corol}[1]{\cref{cor:#1}}%
  \newcommand{\defin}[1]{\cref{def:#1}}%
  \newcommand{\equat}[1]{\cref{equ:#1})}%
  \newcommand{\equnm}[1]{(\ref{equ:#1})\xspace}%
  \newcommand{\examp}[1]{\cref{exa:#1}}%
  \newcommand{\exerc}[1]{\cref{exe:#1}}%
  \newcommand{\factx}[1]{\cref{fac:#1}}%??? do not follow
  \newcommand{\figur}[1]{\cref{fig:#1}}%
  \newcommand{\lemme}[1]{\cref{lem:#1}}%
  \newcommand{\nnote}[1]{\cref{not:#1}}%
  \newcommand{\probl}[1]{\cref{prb:#1}}%
  \newcommand{\prope}[1]{\cref{pty:#1}}%
  \newcommand{\propo}[1]{\cref{pro:#1}}%
  \newcommand{\remar}[1]{\cref{rem:#1}}%
  \newcommand{\sbsct}[1]{\cref{ssc:#1}}%
  \newcommand{\secti}[1]{\cref{sec:#1}}%
  \newcommand{\tabla}[1]{\cref{tab:#1}}%
  \newcommand{\theor}[1]{\cref{the:#1}}%

%%%%%%%%%%%%%\section{Two-column layout}
\newcommand{\miniskip}{\vspace*{0.5ex}}%
%%% Two types of two-column layout
\newlength{\TwClIndent}\setlength{\TwClIndent}{1.8cm}% indent (negative)
\newlength{\TwClSpace}\setlength{\TwClSpace}{1.2em}% space between columns
\newlength{\parindenttemp}% for indentation in minipage
\setlength{\parindenttemp}{\parindent}%
%
\newlength{\TwClTxtWid}% Width of "text" column
\newlength{\TwClFigWid}% Width of "figure" column
\newcommand{\TwClOne}{0.681}%
\newcommand{\TwClTwo}{0.8}%
\newcommand{\TwClThree}{0.5}%
%
\newcommand{\SetTwClPrm}[1]%
   {\setlength{\TwClTxtWid}{#1\textwidth}%
    \setlength{\TwClFigWid}{\textwidth}%
    \addtolength{\TwClFigWid}{-\TwClTxtWid}%
    \addtolength{\TwClTxtWid}{-.5\TwClSpace}%
    \addtolength{\TwClFigWid}{-.5\TwClSpace}%
    \addtolength{\TwClFigWid}{\TwClIndent}}%
\newenvironment{TwClFig}%[1][\TwClOne]
   {\medskip\medskip\noindent%
    \hspace*{-\TwClIndent}%
    \begin{minipage}[c]{\TwClFigWid}%
    \par\vspace*{0mm}% top alignment
    \begin{center}}%
   {\end{center}\end{minipage}}%
\newenvironment{TwClTxt}%
   {\hspace*{\TwClSpace}%
    \begin{minipage}[c]{\TwClTxtWid}%
    \par\vspace*{0mm}% top alignment
    \footnotesize}%
   {\normalsize\end{minipage}\medskip\medskip}%
\newenvironment{SwClCmd}%[1][\TwClOne]
   {\medskip\medskip\noindent%
    \hspace*{-\TwClIndent}%
    \begin{minipage}[c]{\TwClFigWid}%
    \par\vspace*{0mm}% top alignment
    \small}%\footnotesize
   {\normalsize\end{minipage}}%
\newenvironment{SwClTxt}%
   {\hspace*{\TwClSpace}% \setlength{\parindent}{\parindenttemp}%
    \begin{minipage}[c]{\TwClTxtWid}%
    \par\vspace*{0mm}}%% top alignment
   {\end{minipage}\medskip}%\medskip
%
\newlength{\OneClIndent}\setlength{\OneClIndent}{2em}% indent (negative)
\newlength{\OneClTxtWid}% Width of "text" column
\setlength{\OneClTxtWid}{\textwidth}
\addtolength{\OneClTxtWid}{-\OneClIndent}%
\newcommand{\CmdCmt}[2]%[\TwClOne]
   {\noindent%
    #1%
    \par
    \hspace*{\OneClIndent}%
    \begin{minipage}[c]{\OneClTxtWid}%
    \par\vspace*{0mm}% top alignment
    #2%
    \end{minipage}%
    \smallskip\miniskip
    }%
\newcommand{\PushLine}{\hbox{}\hfill\hbox{}}
%%%%%% \section{negative vertical space}
\newcommand{\bigskipneg}{\vspace*{-5ex}} %960728
\newcommand{\medskipneg}{\vspace*{-2ex}} %960728
\newcommand{\smallskipneg}{\vspace*{-1ex}} %960728
\newcommand{\miniskipneg}{\vspace*{-0.25ex}} %960728
%%%%%%%%%%%%%\section{List macros}
\newcommand{\jsListe}[1]%
   {\noindent\makebox[\parindent][r]{\rm(#1)}\hspace*{.8em}\ignorespaces}
%
\newcommand{\thi}{\jsListe{i}}
\newcommand{\thii}{\jsListe{ii}}
\newcommand{\thiii}{\jsListe{iii}}
\newcommand{\thiv}{\jsListe{iv}}
\newcommand{\thv}{\jsListe{v}}
\newcommand{\thvi}{\jsListe{vi}}
\newcommand{\thvii}{\jsListe{vii}}
\newcommand{\thip}{\jsListe{i'}}
\newcommand{\thiip}{\jsListe{ii'}}
\newcommand{\thiiip}{\jsListe{iii'}}
\newcommand{\thivp}{\jsListe{iv'}}
\newcommand{\thvp}{\jsListe{v'}}
\newcommand{\thvip}{\jsListe{vi'}}
\newcommand{\point}{\makebox[2.4em][l]{$\bullet$}}
\newcommand{\thp}{\makebox[2.4em][l]{$\bullet$}}
\newcommand{\pointr}{\makebox[2.4em][r]{$\bullet$ \ }}
\newcommand{\pointn}{\noindent \makebox[1.2em]{$\bullet$}\ignorespaces}
\newcommand{\tiret}{\noindent\makebox[2.4em][r]{-- \msp}}
\newcommand{\thI}{\jsListe{I}}
\newcommand{\thII}{\jsListe{II}}
\newcommand{\thIII}{\jsListe{III}}
%
\newcommand{\tha}{\jsListe{a}}
\newcommand{\thb}{\jsListe{b}}
\newcommand{\thc}{\jsListe{c}}
\newcommand{\thd}{\jsListe{d}}
\newcommand{\thejs}{\jsListe{e}}
\newcommand{\thf}{\jsListe{f}}
\newcommand{\thg}{\jsListe{g}}
%
\newcommand{\thnz}{\jsListe{0}}
\newcommand{\thnu}{\jsListe{1}}
\newcommand{\thnd}{\jsListe{2}}
\newcommand{\thnt}{\jsListe{3}}
\newcommand{\thnq}{\jsListe{4}}
\newcommand{\thnc}{\jsListe{5}}
\newcommand{\thns}{\jsListe{6}}
\newcommand{\thA}{\jsListe{A}}
\newcommand{\thB}{\jsListe{B}}
\newcommand{\thC}{\jsListe{C}}
%
%%%%%
\newcommand{\B}{\Bmbb}
\newcommand{\C}{\Cmbb}
\newcommand{\F}{\Fmbb}
\newcommand{\K}{\Kmbb}
\newcommand{\N}{\Nmbb}
\newcommand{\Q}{\Qmbb}
\newcommand{\R}{\Rmbb}
\newcommand{\T}{\Tmbb}
\newcommand{\Z}{\Zmbb}
\newcommand{\Ed}{\Emsf}
\newcommand{\Fd}{\Fmsf}
\newcommand{\Gd}{\Gmsf}
\newcommand{\kd}{\kmsf}
\newcommand{\zed}{\mathsf{0}}
\newcommand{\und}{\mathsf{1}}
% \newcommand{\zeK}{0_{\K}}
% \newcommand{\unK}{1_{\K}}
%
\newcommand{\Zmin}{\Z\mathrm{min}}
\newcommand{\Zmax}{\Z\mathrm{max}}
%%%%%%%%%%%%%%%%%%%%%%%%%%%%%%%%%%%%%%%%%%%%
\newcommand{\Cpp}{\texttt{C++}\xspace}
\newcommand{\gpp}{\texttt{g++}\xspace}
\newcommand{\tex}{\TeX\xspace}
\newcommand{\latex}{\LaTeX\xspace}
\newcommand{\tafkit}{\textsc{TAF-Kit}\xspace}
%\newcommand{\vcsn}{\textsc{Vaucanson}\xspace}
\def\VcsnVersion{1.4.1}
\def\VcsnVrsnOld{1.4}
% \def\VcsnVersion{1.3.9}
\newcommand{\vcsnv}{\vcsn\VcsnVersion\xspace}
\newcommand{\vcsnvo}{\vcsn\VcsnVrsnOld\xspace}
\newcommand{\tafkitv}{\tafkit\VcsnVersion\xspace}
\newcommand{\Vauc}{\vcsn}% for compatibility
\newcommand{\fsmxml}{\textsc{Fsm\,XML}\xspace}
\newcommand{\xml}{\texttt{XML}\xspace}
\newcommand{\XML}{\xml}% for compatibility
%\newcommand{\vgi}{\textsc{Vgi}\xspace}
\newcommand{\fst}{\textsc{OpenFst}\xspace}
%%% FSMXML
\newcommand{\Prec}{\medskip\noindent\textbf{\textsl{Precondition}:}\e}
\newcommand{\Spec}{\medskip\noindent\textbf{\textsl{Specification}:}%
                   \par\smallskip\noindent}
\newcommand{\Comt}{\medskip\noindent\textbf{\textsl{Comments}:}\xspace}
\newcommand{\Cave}{\medskip\noindent\textbf{\textsl{Caveat}:}\xspace}
\newcommand{\Exam}{\medskip\noindent\textbf{\textsl{Example}:}\xspace}
\newcommand{\vrglst}{,\e}
%%%%%%%%%%%%%%%%%%%%%%%%%%%%%%%%%%%%%%%%%%%%
\makeatletter
\newcommand{\Cxx}{%
  \valign{\vfil\hbox{##}\vfil\cr
    {C\kern-.1em}\cr
    $\hbox{\fontsize\sf@size\z@\textbf{+\kern-0.05em+}}$\cr}%
    \xspace
}
\makeatother
%% ----------------------- %%
%% Texinfo like commands.  %%
%% ----------------------- %%
\newcommand\vcdef[1]{\textsl{#1}}
\newcommand\kbd[1]{\textsl{\texttt{#1}}}
\newcommand\file[1]{`\texttt{#1}'}
\newcommand\command[1]{\texttt{#1}}
\newcommand\var[1]{{\ttfamily\itshape #1}}
\newcommand\code[1]{\texttt{#1}}
\newcommand\typet[1]{\texttt{#1}\texttt{\_t}\xspace}
\newcommand\type[1]{\texttt{#1}\xspace}
\newcommand\cc{\texttt{::}}
\newcommand\samp[1]{`\texttt{#1}'}
\newcommand\option[1]{`\texttt{#1}'}
\newcommand{\taffn}[1]{\code{#1}}

\newcommand{\bs}{\symbol{'134}}%
%%%%%%%%%%%%%%index%%%%%%%%%%%%%
\newcommand{\Indexsc}[1]{\index{#1@\textsc{#1}}}
\newcommand{\Indextt}[1]{\index{#1@\texttt{#1}}}
\newcommand{\IndexFct}[1]{\index{#1@\texttt{#1}}}
\newcommand{\IndexFctIs}[1]{\index{#1@\texttt{is-#1}}}
\newcommand{\IndexCom}[1]{\index{#1@\texttt{\bs#1}}}
\newcommand{\IndexEnv}[1]{\index{#1@\texttt{#1}}}
\newcommand{\IndexOpt}[1]{\index{#1@\texttt{--#1}}}
\newcommand{\IndirInd}[2]{\index{#1@\texttt{#1}|see{#2}}%
                          \index{#2!#1@\texttt{#1}}}
\newcommand{\SubIndtt}[2]{\index{#2!#1@\texttt{#1}}}
%%%%%%%% special print + index %%%%%%%%%%%%%%%
\newcommand{\emphind}[1]{\emph{#1}\index{#1}}
\newcommand{\Option}[1]{\texttt{--#1}\index{#1@\texttt{--#1}}}
\newcommand{\ShortOpt}[1]{\texttt{-#1}\index{#1@\texttt{-#1}}}
\newcommand{\FctInd}[1]{\texttt{#1}\index{#1@\texttt{#1}}}
%%%%%%%% special print %%%%%%%%%%%%%%%%%%%%%%%%
% \newcommand{\Prm}[1]{\textsl{#1}}
\newcommand{\Prm}[1]{{\ttfamily\itshape #1}}
% \newcommand{\Com}[1]{\texttt{\bs#1}}
\newcommand{\Fct}[1]{\texttt{#1}}
\newcommand{\Fctp}[1]{\texttt{#1()}}
\newcommand{\Fctq}[2]{\texttt{#1(#2)}}
\newcommand{\FctPar}[2]{\Fct{#1} $\mathtt{<}$\Prm{#2}$\mathtt{>}$}
\newcommand{\Fctaut}[1]{\FctPar{#1}{aut}}
\newcommand{\Fctexp}[1]{\FctPar{#1}{exp}}
\newcommand{\FctParD}[3]{\Fct{#1} $\mathtt{<}$\Prm{#2}$\mathtt{>}$ %
                                  $\mathtt{<}$\Prm{#3}$\mathtt{>}$}
\newcommand{\FctautD}[1]{\FctParD{#1}{aut1}{aut2}}
\newcommand{\Fctkaut}[1]{\FctParD{#1}{aut}{k}}
\newcommand{\FctexpD}[1]{\FctParD{#1}{exp1}{exp2}}
\newcommand{\Fctkexp}[1]{\FctParD{#1}{exp}{k}}
\newcommand{\Fcttra}[1]{\FctPar{#1}{tdc}}
\newcommand{\FcttraD}[1]{\FctParD{#1}{tdc1}{tdc2}}
\newcommand{\Fcttraaut}[1]{\FctParD{#1}{tdc}{aut}}
%
\newcommand{\fm}{\vcdef{fm}\xspace}
\newcommand{\fmp}{\vcdef{fmp}\xspace}
\newcommand{\fmpt}{\vcdef{fmp-transducer}\xspace}
\newcommand{\fmpts}{\vcdef{fmp-transducers}\xspace}
\newcommand{\rwt}{\vcdef{rw-transducer}\xspace}
\newcommand{\rwts}{\vcdef{rw-transducers}\xspace}
\newcommand{\lal}{\code{lal}\xspace}
\newcommand{\laa}{\code{laa}\xspace}
\newcommand{\law}{\code{law}\xspace}
\newcommand{\lam}{\code{lam}\xspace}
\newcommand{\las}{\code{las}\xspace}
%%%%%%%%%%%%%%%%%%%%%%%%%%%%%%%%%%%%%%%%%%%%%%
\newcommand{\dvjs}{\vcdef{Decided at VJS}\xspace}

% \newcommand{\ComInd}[1]{\texttt{\bs#1}\index{#1@\texttt{\bs#1}}}
% \newcommand{\ComPar}[2]{\texttt{\bs#1\{}}
% \newcommand{\ComParInd}[2]%
%    {\texttt{\bs#1\{}\Prm{#2}\texttt{\}}\index{#1@\texttt{\bs#1}}}
% \newcommand{\ComParPar}[3]%
%    {\texttt{\bs#1\{}\Prm{#2}\texttt{\}\{}\Prm{#3}\texttt{\}}}
% \newcommand{\ComParParInd}[3]%
%    {\texttt{\bs#1\{}\Prm{#2}\texttt{\}\{}\Prm{#3}\texttt{\}}%
%     \index{#1@\texttt{\bs#1}}}
%

%% Display an interactive session.
\usepackage{alltt}
\newenvironment{shell}
{\begin{alltt}}
{\end{alltt}}
% @VcsnVersion@
\newcommand{\bslash}{\texttt{\symbol{92}}}
\newcommand{\nojs}{\texttt{\symbol{176}}}
%% XML Tag.
\newcommand{\xtag}[1]{\texttt{<#1>}}
\newcommand{\xftag}[1]{\texttt{</#1>}}
\newcommand{\xtagf}[1]{\texttt{<#1/>}}
\newcommand{\xval}[1]{\textsl{\texttt{#1}}}
\newcommand{\xattr}[1]{\texttt{#1}}
\newcommand{\BTRE}{\texttt{\{Body::typedRegExp\}}}
\newcommand{\tagindent}{\hspace*{\tagindentl}}
\newcommand{\tagsp}{\hspace*{\tagspl}}
\newcommand{\token}{token,\xspace}
\newcommand{\strng}{string,\xspace}
\newcommand{\req}{req.\xspace}
\newcommand{\opt}{opt.\xspace}
\newcommand{\occ}{occ.\xspace}
\newcommand{\oneocc}{}
\newcommand{\ptn}{$\cdot$\xspace}
\newcommand{\trt}{{\normalfont\bfseries\textendash}\xspace}

\newenvironment{Com}
{%
% \renewcommand{\theenumi}{\alph{enumi}}
\medskip
\noindent
\textit{Comment}:
% \par
\itshape}%
{%
% \renewcommand{\theenumi}{\arabic{enumi}}%
% \renewcommand{\theenumii}{\theenumi.\arabic{enumii}}%
\medskip}
 \newenvironment{ComV}
 {\renewcommand{\theenumi}{\alph{enumi}}
 \medskip
 \noindent
 \textit{Comment for the \vcsn Group}:
 % \par
 \itshape}%
 {\renewcommand{\theenumi}{\arabic{enumi}}%
  \renewcommand{\theenumii}{\theenumi.\arabic{enumii}}%
  \medskip}
\newenvironment{ComVd}[1]
{\renewcommand{\theenumi}{\alph{enumi}}
\medskip
\noindent
\textit{Comment for the \vcsn Group} (#1):
% \par
\itshape}%
{\renewcommand{\theenumi}{\arabic{enumi}}%
 \renewcommand{\theenumii}{\theenumi.\arabic{enumii}}%
 \medskip}
\newenvironment{QstV}
{\medskip
\noindent
\textit{Question for the \vcsn Group}:
\itshape}%
{\medskip}
%%% for chA
\newcounter{enumitemp}
%%%
\SetTwClPrm{\TwClOne}%
\SetVCDirectory{figures/}
\newcommand{\VGIFig}{figures/VGI/}
% \ShowFrame
%========== hyphenation===========================
% add words to TeX's hyphenation exception list
\hyphenation{semi-ring}
\hyphenation{simu-la-tion}
