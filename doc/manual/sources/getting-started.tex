% -*- mode: latex; fill-column: 79; mode: auto-fill; mode: flyspell; buffer-file-coding-system: utf-8 -*-
\FILL


\CODE{PYTHONPATH=/opt /usr/local/bin/vcsn notebook -m vcsn-site --no-browser --ip='*' \&}

Sources:

\FIXME{it could be as simple as typing the following line from the source directory:}\\
\CODE{\$ ./tests/bin/vcsn notebook}

\FIXME{but we strongly recommend using a \TDEF{site file}, in this example
  named \FILE{/opt/vcsn-site.py}:}\\
\CODE{\$ PYTHONPATH=/opt ./tests/bin/vcsn notebook -m vcsn-site}

A site file contains Python commands to be automatically run at the beginning
of every IPython Notebook session.  The one used in the virtual machine image
simply contains the two lines:
\begin{lstlisting}
from vcsn import *
from vcsn.functional import *
\end{lstlisting}
Thanks to these global declarations the user can refer to identifiers bound in
the \CODE{vcsn} and \CODE{vcsn.functional} modules without prepending long module
names every time: for example she will be able to simply type
``\CODE{context('a-z',  'z')}''
in lieu of the more verbose
``\CODE{vcsn.functional.context('a-z',  'z')}''.

We will omit explicit module names in the following, assuming the user to
either use a site file or manually type the two \CODE{import} lines above at
the beginning of every notebook.



\section{Using the Virtual Machine image}
\FIXME{recycle: text written by Jacques with Alfred; Alfred has a
  copy; I've never seen it.}

\begin{lstlisting}
To install the Vaucanson virtual machine, please follow these procedures,
 - Install VirtualBox https://www.virtualbox.org/wiki/Downloads.
 - Install Vagrant from http://www.vagrantup.com/downloads.html
 - Download the Vaucanson Virtual Machine from http://vaucanson-project.org/releases/Vaucanson-2.0.0f-VM.tar
 - In a terminal, issue the following commands (in the same directory
   where the Vaucanson VM was downloaded):
   $ tar -xf Vaucanson-2.0.0f-VM.tar
   $ cd Vaucanson-2.0.0f-VM
   $ vagrant up
 - You can remove the Vaucanson-2.0.0f-VM.tar if you wish to save space.
 - Connect your web browser to http://localhost:8888
  - Have fun!
 - To turn of the virtual machine,
   $ vagrant halt
\end{lstlisting}

\section{Compiling and installing from sources}
Since the software does dynamic compilation of complex C++ files, we strongly
recommend ccache to speed it up.

\subsection{Requirements}

\subsection{GNU/Linux}
\subsection{Mac OS X}
\FIXME{The ``ports'' system is recommended.  If you don't use ports then pay attention to prerequisites.}

\label{assume-we-build-in-the-source-directory}
\FIXME{source and build directories: we assume they are the same in the following.}
