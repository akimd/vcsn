% -*- mode: latex; fill-column: 79; mode: auto-fill; mode: flyspell; buffer-file-coding-system: utf-8 -*-
The Vaucanson system is free software, and as such comes with full source code.
However, particularly as a convenience to prospective users who want to quickly
start exploring the software, we also provide a virtual machine image.

\index{virtual machine}
The virtual machine image is particularly easy to use and requires no set up;
we recommend it as a first choice to all beginners.

\section{Using the Virtual Machine image}
\index{Vagrant}
\index{VirtualBox}
The Virtual Machine image works on x86 and x86\_64 machines, using any
operating system supported by VirtualBox.  It relies on VirtualBox and Vagrant,
which can be found at \url{https://www.virtualbox.org/wiki/Downloads} and
\url{http://www.vagrantup.com/downloads.html}.  The following assumes that the
user already installed the two dependencies, using packages or any appropriate
method.

\subsection{Virtual machine download and startup}
The virtual machine image itself is a single file taking about 800MB.
The current version \vcsnversion{} can be found at
\url{http://vaucanson-project.org/releases/Vaucanson-\vcsnversion-VM.tar}.  The
user should download it, open a terminal and enter the directory containing the
file \FILE{Vaucanson-\vcsnversion-VM.tar} with the appropriate \CODE{cd} command.
\\\\
At this point the user should issue the following command to extract the
archive, at the shell prompt --- here indicated as ``\CODE{\$ }'':
\begin{alltt}
$ tar -xfv Vaucanson-\vcsnversion-VM.tar
\end{alltt}
This will create a \FILE{Vaucanson-\vcsnversion-VM} subdirectory; the user
should enter it and then start \CODE{vagrant}, which will use the configuration
file contained there:
\begin{alltt}
$ cd Vaucanson-\vcsnversion-VM
$ vagrant up
\end{alltt}
\index{tunnel}
\index{port}
\index{URL}
\index{port}%\index{port!TCP/IP}
\index{share}
\index{directory}
\index{filesystem}
The last command will boot the virtual machine, consisting of a full GNU/Linux
system running a Vaucanson server on a port accessible as port 8888 on the host
machine.  In order to connect with it the user should open the
URL \url{http://localhost:8888} with a web browser, thus connecting to the
local machine on port 8888, transparently \textit{tunneled} to the Vaucanson server on the
virtual machine.  The directory \FILE{Vaucanson-\vcsnversion-VM}
is \textit{shared} between the virtual machine and the host machine, which can be
convenient to exchange files such as notebooks.

To turn off the virtual machine the user should issue the command:
\begin{alltt}
$ vagrant halt
\end{alltt}
The file \FILE{Vaucanson-\vcsnversion-VM.tar} can be safely removed to save
space.
\\
\\
The rest of this chapter deals with the way to run the system starting from its
source code.  A user who is only interested in running Vaucanson from the
virtual machine may safely skip to §\ref{chapter-right-after-getting-started}.


\section{Compiling and installing from a source tarball}
\label{compiling-from-a-source-tarball}
\index{build system}
\index{make}
\index{Autoconf}
\index{Automake}
\index{GNU!build system}
\index{\FILE{configure}}
\index{\FILE{local.mk}}
\index{configure}
The Vaucanson build system follows the standard GNU conventions described in the Autoconf
and Automake documentation, and is comparatively simple.  The \FILE{configure}
script generates a single \FILE{Makefile}, even if its Automake sources are
divided, for conceptual clarity, into several \FILE{local.mk} files throughout
several source subdirectories.

In the following we are going to show how a simple example of the configuration
and build process.
\label{assume-we-build-in-the-source-directory}
We assume for simplicity that the software is built in its
source directory, which is the convenient choice for end-users.  Building in a
separate directory is however fully supported, and actually encouraged for
developers who may wish to experiment with different configurations.

\subsection{Requirements}
\label{requirements-tarball}
\index{requirements}
Vaucanson is written in C++ and Python and developed on GNU/Linux and Mac
OS X systems.  Support for additional platforms in the Unix family is 
likely easy to add, as long as a suitable C++ compiler and a IPython
implementation exist on the platform.

\index{C++11}
\index{GCC}
\index{Clang}
The implementation uses modern features of the C++ language only available in
the comparatively recent C++11 language revision.  The authors routinely test
the system with recent versions of GCC and Clang; both are supported.

\index{requirements}
\index{IPython notebook}
\index{C++}
\index{GMP}
\index{Boost}
The system depends on IPython notebook, the Boost general-purpose C++ library,
and the GMP library.  In every case header files should be installed as well
along with libraries.
\index{runtime compilation}
\index{compilation!runtime}
\index{\FILE{ccache}}
Since the software performs runtime compilation of complex C++ code, we strongly
recommend the use of \FILE{ccache} to speed it up.

The unusual nature of a system compiling code while running entails one
restriction: compile-time dependencies are also run-time dependencies, and the
user may not (in general) safely uninstall the compilers and library headers
which were used to build Vaucanson.

\subsection{Downloading}

\index{tarball}
\index{download}
The user should download a tarball file from some URL such as
\url{http://vaucanson-project.org/releases/Vaucanson-\vcsnversion.tar.gz}
into some convenient directory she has write access to, and then
\FILE{cd} to the directory.

The next step is uncompressing the archive:
\begin{alltt}
$ tar -xfvz Vaucanson-\vcsnversion.tar.gz
\end{alltt}
The command above will uncompress the tarball and print each extracted file
name to the terminal as long as it is written to the disk.

As soon as the prompt reappears the user can enter the new \FILE{vaucanson}
subdirectory:
\begin{alltt}
$ cd vaucanson
\end{alltt}

\subsection{Configuring}
\label{compiling-from-a-source-tarball-configuring}
\FILL
\begin{alltt}
$ cd vaucanson
\end{alltt}

?????
\subsection{Compiling}
\subsection{Installing}

\subsection{GNU/Linux notes}
\subsection{Mac OS X notes}
\FIXME{The ``ports'' system is recommended.  If you don't use ports then pay attention to prerequisites.}


\section{Compiling and installing from git}
\index{\FILE{git}}
Compiling from git amounts to downloading a source snapshot, generating the
configuration script and then following the instructions for compiling
from a tarball, as explained in
§\ref{compiling-from-a-source-tarball}.

\index{requirements}
\index{\FILE{git}}
\index{\FILE{Autoconf}}
\index{\FILE{Automake}}
\index{\FILE{Libtool}}
\index{\TeX{}Live}
There are a few more requirements in addition to the ones in
§\ref{requirements-tarball}: of course \FILE{git}, then Autoconf, Automake,
Libtool, and a recent \LaTeX{} system to compile the manual --- the entire
\TeX{}Live distribution is a safe choice, if somewhat overkill.

\index{git!public repository}
\index{git!repository}
\index{source repository}
\index{public repository}
The Vaucanson git repository can be accessed (read-only) by anybody at the address
% Using \url here is probably a bad idea: there's nothing interesting for users
% to see unless they have an account on the LRDE system.  Am I wrong?
\texttt{https://gitlab.lrde.epita.fr/vcsn/vaucanson.git}
with no need for authentication; the address
\texttt{git@gitlab.lrde.epita.fr:vcsn/vaucanson.git} provides write access, but
is only available to registered project members.

In order to download a copy from the \CODE{master} branch the user should enter
some appropriate directory she has write access to, and type:
\begin{alltt}
$ git clone https://gitlab.lrde.epita.fr/vcsn/vaucanson.git
\end{alltt}
The command above creates a \FILE{vaucanson} subdirectory containing the sources.  The user
should enter it and generate configuration files by invoking the provided script:
\begin{alltt}
$ cd vaucanson
$ ./bootstrap
\end{alltt}
At this point a \FILE{configure} script will be available.  The user may
configure, compile and install exactly as shown above, following the
instructions starting at §\ref{compiling-from-a-source-tarball-configuring}.

\section{Starting up after compiling from sources}

%\CODE{PYTHONPATH=/opt /usr/local/bin/vcsn notebook -m vcsn-site --no-browser --ip='*'\&}
\CODE{PYTHONPATH=/opt /usr/local/bin/vcsn notebook -m vcsn-site \&}

Sources:

\FIXME{it could be as simple as typing the following line from the source directory:}\\
\CODE{\$ ./tests/bin/vcsn notebook}

\FIXME{but we strongly recommend using a \TDEF{site file}, in this example
  named \FILE{/opt/vcsn-site.py}:}\\
\CODE{\$ PYTHONPATH=/opt ./tests/bin/vcsn notebook -m vcsn-site}

A site file contains Python commands to be automatically run at the beginning
of every IPython Notebook session.  The one used in the virtual machine image
simply contains the two lines:
\begin{lstlisting}
from vcsn import *
from vcsn.functional import *
\end{lstlisting}
Thanks to these global declarations the user can refer to identifiers bound in
the \CODE{vcsn} and \CODE{vcsn.functional} modules without prepending long module
names every time: for example she will be able to simply type
``\CODE{context('a-z',  'z')}''
in lieu of the more verbose
``\CODE{vcsn.functional.context('a-z',  'z')}''.

We will omit explicit module names in the following, assuming the user to
either use a site file or manually type the two \CODE{import} lines above at
the beginning of every notebook.
