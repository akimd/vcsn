% -*- mode: latex; fill-column: 79; mode: auto-fill; mode: flyspell; buffer-file-coding-system: utf-8 -*-
The Vaucanson system is free software, and as such comes with full source code.
However, particularly as a convenience to prospective users who want to quickly
start exploring the software, we also provide a virtual machine image.

The virtual machine image is particularly easy to use and requires no set up;
we recommend it as a first choice to all beginners.

\section{Using the Virtual Machine image}
The Virtual Machine image works on x86 and x86\_64 machines, using any
operating system supported by VirtualBox.  It relies on VirtualBox and Vagrant,
which can be found at \url{https://www.virtualbox.org/wiki/Downloads} and
\url{http://www.vagrantup.com/downloads.html}.  The following assumes that the
user already installed the two dependencies, using packages or any appropriate
method.

\subsection{Virtual machine download and startup}
The virtual machine image itself is a single file taking about \FIXME{1GB}.
The current version \vcsnversion{} can be found at
\url{http://vaucanson-project.org/releases/Vaucanson-\vcsnversion-VM.tar}.  The
user should download it, open a terminal and enter the directory containing the
file \FILE{Vaucanson-\vcsnversion-VM.tar} with the appropriate \CODE{cd} command.
\\\\
At this point the user should issue the following command to extract the
archive, at the shell prompt --- here indicated as ``\CODE{\$ }'':
\begin{alltt}
$ tar -xfv Vaucanson-\vcsnversion-VM.tar
\end{alltt}
This will create a \FILE{Vaucanson-\vcsnversion-VM} subdirectory; the user
should enter it and then start \CODE{vagrant}, which will use the configuration
file contained there:
\begin{alltt}
$ cd Vaucanson-\vcsnversion-VM
$ vagrant up
\end{alltt}
The last command will boot the virtual machine, consisting of a full GNU/Linux
system running a Vaucanson server on a port accessible as port 8888 on the host
machine.  In order to connect with it the user should open a web browser and
open the \url{http://localhost:8888} URL, connecting to the local machine on
port 8888.  The directory \FILE{Vaucanson-\vcsnversion-VM} is \textit{shared}
between the virtual machine and the host machine, which can be convenient to
exchange files, such as notebooks.

To turn off the virtual machine the user should issue the command:
\begin{alltt}
$ vagrant halt
\end{alltt}

The file \FILE{Vaucanson-\vcsnversion-VM.tar} can be safely removed to save
space.



\CODE{PYTHONPATH=/opt /usr/local/bin/vcsn notebook -m vcsn-site --no-browser --ip='*' \&}

Sources:

\FIXME{it could be as simple as typing the following line from the source directory:}\\
\CODE{\$ ./tests/bin/vcsn notebook}

\FIXME{but we strongly recommend using a \TDEF{site file}, in this example
  named \FILE{/opt/vcsn-site.py}:}\\
\CODE{\$ PYTHONPATH=/opt ./tests/bin/vcsn notebook -m vcsn-site}

A site file contains Python commands to be automatically run at the beginning
of every IPython Notebook session.  The one used in the virtual machine image
simply contains the two lines:
\begin{lstlisting}
from vcsn import *
from vcsn.functional import *
\end{lstlisting}
Thanks to these global declarations the user can refer to identifiers bound in
the \CODE{vcsn} and \CODE{vcsn.functional} modules without prepending long module
names every time: for example she will be able to simply type
``\CODE{context('a-z',  'z')}''
in lieu of the more verbose
``\CODE{vcsn.functional.context('a-z',  'z')}''.

We will omit explicit module names in the following, assuming the user to
either use a site file or manually type the two \CODE{import} lines above at
the beginning of every notebook.


\section{Compiling and installing from sources}
Since the software does dynamic compilation of complex C++ files, we strongly
recommend ccache to speed it up.

\subsection{Requirements}

\subsection{GNU/Linux}
\subsection{Mac OS X}
\FIXME{The ``ports'' system is recommended.  If you don't use ports then pay attention to prerequisites.}

\label{assume-we-build-in-the-source-directory}
\FIXME{source and build directories: we assume they are the same in the following.}
