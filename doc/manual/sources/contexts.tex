% -*- mode: latex; fill-column: 79; mode: auto-fill; mode: flyspell; buffer-file-coding-system: utf-8 -*-
\chapter{An extraordinary and funny chapter}

The Vaucanson system relies on a central concept: a \textit{context}
denotes type information about an automaton or a regular expression.
\\
\\
A context is made of two parts:
\begin{itemize}
\item a \textit{labelset} $L$, denoting the type of labels, members of
  a \textit{monoid};
\item a \textit{weightset} $W$, denoting the type of weights, members
  of a \textit{semiring}.
\end{itemize}
When writing a context $C$ composed of a labelset $L$ and a weightset
$W$ we use the notation $C = L \to W$, intuitively hinting at the
type of an automaton transition ``consuming'' a label in $L$ and
``producing'' a weight in $W$.
\\
\\
{{\rm%\small
\begin{tabular}{ll}
\textit{context} $::=$ & \SPACERF\textit{labelset} $\to$ \textit{weightset}
\\\\
\textit{labelset} $::=$ & \SPACERF\textit{alphabet}
\SPACERP \textit{alphabet}$^?$
\SPACERP \textit{alphabet}$^*$
\SPACERP \textit{labelset} $\times$ \textit{labelset}
\SPACERP \textsf{RatE[}\textit{context}\textsf{]}
\\\\
\textit{weightset} $::=$ & \SPACERF$\SET{B}$
\SPACERP $\SET{Z}_{min}$
\SPACERP $\SET{Q}$
\SPACERP $\SET{R}$
\SPACERP \textsf{Series[}\textit{context}\textsf{]}
%% \\\\
%% \textit{s-expression} $::=$ & \SPACERF\textit{atom} & \{ \textit{atom} \}
%% \SPACERP\CODE{(}\ \textit{s-expression}\ \textit{rest} & \{ s-cons(\textit{s-expression}, \textit{rest}) \}
%% \SPACERP\textit{prefix}\ \textit{s-expression} & \{
%% s-cons(lookup(\textit{prefix}), s-cons(\textit{s-expression}, \CODE{()})) \}
%% \\\\ \textit{rest} $::=$ & \SPACERF\CODE{)} & \{ \CODE{()} \}
%% \SPACERP\CODE{.\ }\textit{s-expression}\ \CODE{)} & \{ \textit{s-expression} \}
%% \SPACERP\textit{s-expression}\ \textit{rest} & \{ s-cons(\textit{s-expression},
%% \textit{rest}) \}
\end{tabular}
}}
