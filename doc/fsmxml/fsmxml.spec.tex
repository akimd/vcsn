\documentclass[a4paper]{article}
\usepackage{ifthen}
\usepackage{tabularx}
\usepackage{graphics}
\usepackage{geometry}
\usepackage{lscape}
\usepackage{hyperref}
\usepackage{vaucanson-g}
\usepackage{moreverb}

%%%%%%%%%%%%%%%%%%%%%%%%%%%%%%%%%%%%%%%%%%%%%%%%%%%%%%%%%%%%%%%%%%%%%
%% Give a numbering to paragraphs and subparagrahs and
%% display them into the table of content.
\setcounter{tocdepth}{5}
\setcounter{secnumdepth}{5}

%% Display paragraphs as sections.
\makeatletter
\renewcommand\paragraph{\@startsection{paragraph}{4}{\z@}%
  {-3.25ex\@plus -1ex \@minus -.2ex}%
  {1.5ex \@plus .2ex}%
  {\normalfont\normalsize\bfseries}}
\renewcommand\subparagraph{\@startsection{subparagraph}{5}{\z@}%
  {-3.25ex\@plus -1ex \@minus -.2ex}%
  {1.5ex \@plus .2ex}%
  {\normalfont\normalsize\bfseries}}
\makeatother

%% Use to replace 'see subsection x.x.x' with 'see Section x.x.x'
\renewcommand{\subsectionautorefname}{Section}
\renewcommand{\subsubsectionautorefname}{Section}

%%%%%%%%%%%%%%%%%%%%%%%%%%%%%%%%%%%%%%%%%%%%%%%%%%%%%%%%%%%%%%%%%%%%%
%% Macro used :
%%  - to organize attributes, tag and values into tables.
%%  - to highlight attributes, tag and values.

%% XML Tag.
\newcommand{\xtag}[1]{\texttt{<#1>}}
%% XML Attribute.
\newcommand{\xattr}[1]{\texttt{#1}}
%% Attributes values
\newcommand{\xval}[1]{\texttt{#1}}

%% Attributes description
\newcommand{\attr}[6]{
  \begin{tabular}{| p{0.175\textwidth} | p{0.175\textwidth} | p{0.175\textwidth} | p{0.175\textwidth} | p{0.175\textwidth} |}
    \hline
    \xattr{#1} & #2 & #3 & #4 & #5\\\hline
    \multicolumn{5}{| l |}{#6}\\\hline
  \end{tabular}\\
}
\newcommand{\attrs}[1]{
  \attr{\emph{Name}}
       {\emph{Info}}
       {\emph{Type}}
       {\emph{Default}}
       {\emph{Values}}
       {\emph{Description}}
  #1
}

%% Child description
\newcommand{\child}[3]{
  \item \xtag{#1}, #2
  See \autoref{#3} for more details\xspace
}
\newcommand{\children}[1]{
  \begin{itemize}
    #1
  \end{itemize}
}

%% Ref to examples
\newcommand{\example}[2]{See \hyperref[#1]{\nameref{#1}}, lines #2 for a complete example\xspace}

\newcommand{\version}{1.1}

%%%%%%%%%%%%%%%%%%%%%%%%%%%%%%%%%%%%%%%%%%%%%%%%%%%%%%%%%%%%%%%%%%%%%
\begin{document}

%% Title
\title{Finite State Machine XML (FSMXML)\\
  Specifications}
\author{The {\sc Vaucanson} Group}
\date{\today}
\maketitle

%% Abstract
\begin{abstract}
This document describes FSMXML. FSMXML provides generic finite state
machines and regular expressions' XML description.
\end{abstract}

%% Table of content
\tableofcontents{}

\newpage
%%%%%%%%%%%%%%%%%%%%%%%%%%%%%%%%%%%%%%%%%%%%%%%%%%%%%%%%%%%
\section{Terminology}

\subsection{General}

\begin{itemize}
\item \emph{MUST}

  This word, or the terms \emph{REQUIRED} or \emph{SHALL}, mean that
  the definition is an absolute requirement of the specification.

\item \emph{MUST NOT}

  This phrase, or the phrase \emph{SHALL NOT}, mean that the
  definition is an absolute prohibition of the specification.

\item \emph{SHOULD}

  This word, or the adjective \emph{RECOMMENDED}, mean that there may
  exist valid reasons in particular circumstances to ignore a
  particular item, but the full implications must be understood and
  carefully weighed before choosing a different course.

\item \emph{SHOULD NOT}

  This phrase, or the phrase \emph{NOT RECOMMENDED} mean that there
  may exist valid reasons in particular circumstances when the
  particular behavior is acceptable or even useful, but the full
  implications should be understood and the case carefully weighed
  before implementing any behavior described with this label.

\item \emph{MAY}

  This word, or the adjective \emph{OPTIONAL}, mean that an item is
  truly optional.  One vendor may choose to include the item because a
  particular marketplace requires it or because the vendor feels that
  it enhances the product while another vendor may omit the same item.
  An implementation which does not include a particular option MUST be
  prepared to interoperate with another implementation which does
  include the option, though perhaps with reduced functionality. In
  the same vein an implementation which does include a particular
  option MUST be prepared to interoperate with another implementation
  which does not include the option (except, of course, for the
  feature the option provides).

\end{itemize}

\subsection{FSMXML}

Each element of the FSMXML format is described in a section with the
following 2 parts:

\subsubsection{Children}

Describes the required/allowed children of the element, their number
of occurrences and a brief description of what they stand for.

\subsubsection{Attributes}

The attributes of the element are listed in a table like the following:
\\\attrs{}
\\in which:

\paragraph*{Name}
Gives the name of the attribute.

\paragraph*{Info}
Gives informations about the attribute:
\begin{itemize}
\item \emph{Required}

The attribute is mandatory.

\item \emph{Pivot}

  Depending on the attribute's value, the form of the children should
  differ.

\item \emph{Unique}

  When the same element can be used more than once, it means that one
  shall not declare twice the same value in each of those elements.

\item \emph{Valid}

  Implies that the value must have already been defined previously in
  the XML document.
\end{itemize}

\paragraph*{Type}
Gives the type of the value:
\begin{itemize}
\item \emph{token}

  Is used for a string which takes only a couple of already defined
  values, listed in the \emph{Values} column.

\item \emph{ID}

  Is used for a string which represents an identifier, that can be
  used many times in the document to refer to the unique same
  ``concept''.

\item \emph{generator}

  Is used for a string which only takes a couple of values
  (implicitly) defined in the XML document.

\item \emph{weight}

  Is used for a string which only takes a couple of values
  (implicitly) defined in the XML document.

\item \emph{URI}

  A Uniform Resource Identifier (URI), is a compact string of
  characters used to identify or name a resource.

\item \emph{integer}, \emph{float}, \emph{string}.
\end{itemize}

\paragraph*{Default}
Since some attributes are always required and would usually take the
same value, a default value is possible. Omitting the attribute will
implicitly stand for the default value.

\paragraph*{Values}
When \emph{Type} is set to \xval{token}, lists all the possible values
that can take the attribute.

\paragraph*{Description}
Gives a complete description of the attribute.

\newpage
%%%%%%%%%%%%%%%%%%%%%%%%%%%%%%%%%%%%%%%%%%%%%%%%%%%%%%%%%%%
\section{FSMXML}

\subsection{\xtag{fsmxml}}
\label{fsmxml}

The top-level \emph{root} element, which carries version information,
etc. \example{automatonA1}{1 and 72}.

\subsubsection*{Attributes}
\attrs{
  \attr{xmlns}{}{URI}{none}{}{Sets the namespace.}
  \attr{version}{}{float}{none}{}{Sets the version of FSMXML used.
    This one is \version}
}

\subsubsection*{Children}
\children{

  \child{expression}{A regular expression. Occurs 0 or more
    times.}{expression}.

  \child{automaton}{An automaton. Occurs 0 or more times.}{automaton}.

}

%%%%%%%%%%%%%%%%%%%%%%%%%%%%%%%%%%%%%%%%%%%%%%%%%%%%%%%%%%%
\section{Type}

Regular expressions and automata are described over a type, which can
be described in the same way for both objects. In FSMXML this type is
described by the following elements:

\subsection{\xtag{valueType}}
\label{valueType}

The top-level element of the type part. \example{automatonA1}{5-16}.

\subsubsection*{Attributes}
None.

\subsubsection*{Children}
\children{

  \child{weightset}{The weight set. Occurs 1 time, is required.}
  {weightset}.

  \child{labelSet}{The alphabet. Occurs 1 time, is required.}
  {labelSet}.

}

\subsection{\xtag{labelSet}}
\label{labelSet}

Holds the description of a labelSet.

\subsubsection*{Attributes}
\attrs{
  \attr{type}{Required, Pivot}{token}{none}{``unit'', ``free'',
    ``product''} {Type of the labelSet.}
}

\subsubsection{Unit Label Set}

When \xattr{type} is set to ``unit'', it is equivalent to the lack of
label sets. It enables the possibility to describe valued graphs
within the same format.

\subsubsection{Free Label Set}
\label{freeLabelSet}

When \xattr{type} is set to ``free'', the \xtag{labelSet} describes a
free labelSet and inherits the following new attributes and children.


\subsubsection*{Attributes}
\attrs{
  \attr{genKind}{Required, Pivot}{token}{none}{``simple'', ``tuple''}
  {Kind of the generators.}
  \attr{genDescript}{Required, Pivot}{token}{``enum''}{``enum'',
    ``range'', ``set''} {How are described the generators.}
}

\subsubsection*{Children}
None.

\paragraph{Free Label Set with ``simple'' generators}

When \xattr{type} is set to ``free'', and \xattr{genKind} to
``simple'', the \xtag{labelSet} inherits the following new attributes
and children. \example{automatonA1}{9-15}.

\paragraph*{Attributes}

\attrs{
  \attr{genSort}{Requied}{token}{none}{``letter'', ``digit'',
    ``integer'', ``alphanum''} {Sort of the generators.}
}
\paragraph*{Children}
\children{
  \child{labelGen}{A labelSet generator. Occurs 1 or more
    times.}{labelGen}.
}

\paragraph{Free LabelSet with ``tuple'' generators}

When \xattr{type} is set to ``free'', and \xattr{genKind} to
``tuple'', the \xtag{labelSet} inherits the following new attributes
and children. \example{automatonPdp1}{9-33}.

  \attr{genDim}{Required}{integer}{none}{Greater than 1} {Dimension of
    the tuple.}

\paragraph*{Children}
\children{
  \child{genSort}{List of sort of generator for each ``free'' label
    set. Occurs 1 time, is required.}{genSort}.
  \child{labelGen}{A labelSet generator. Occurs 1 or more
    times.}{labelGen}.
}

\subsubsection{Product LabelSet}

When \xattr{type} is set to ``product'', the \xtag{labelSet} describes
a product of free label sets and inherits the following new attributes
and children. \example{automatonPhim1}{9-26}.

\subsubsection*{Attributes}
\attrs{
  \attr{prodDim}{Required}{integer}{none}{Greater than 1} {Dimension
    of the product.}
}
\subsubsection*{Children}
\children{
  \child{labelSet}{A \texttt{free} labelSet. Occurs \xval{prodDim}
    times.}{freeLabelSet}.
}

\subsection{\xtag{genSort}}
\label{genSort}

Describes the sort of the generator of each item of the tuple in a
``free'' label set with ``tuple'' generators.
\example{automatonPdp1}{13-16}.

\subsubsection*{Attributes}
None.
\subsubsection*{Children}
\children{
  \child{genCompSort}{Sort of an item within the generator. Occurs
    \xval{genDim} times.}{genCompSort}.
}

\subsection{\xtag{genCompSort}}
\label{genCompSort}

Describes the sort of the \xval{k}th coordinate/component in a
``tuple'' generator, \xval{k} being the position of the element in the
list within \xtag{genSort}. \example{automatonPdp1}{18-19}.

\subsubsection*{Attributes}
\attrs{
  \attr{value}{Required}{token}{none}{``letter'', ``digit'',
    ``integer''} {Sort of a coordinate/component of the ``tuple''
    generator.}
}
\subsubsection*{Children}
None.

\subsection{\xtag{labelGen}}
\label{labelGen}

\subsubsection{``enum'' generators}
When \xattr{genDescript} is set to ``enum'', the \xtag{labelGen}
inherits the following new attributes:

\paragraph{``enum'' generator of ``simple'' sort}

Describes a generator of label set when its \xattr{genDescript} is set
to ``enum'' and \xattr{genKind} to ``simple''.
\example{automatonA1}{13-14}.

\paragraph*{Attributes}
\attrs{
  \attr{value}{Required}{generator}{none}{}{ Gives the value of the
    generator. Should fit \xattr{genSort} restriction.  }
}

If used within a \xtag{labelSet}, should also be \emph{Unique}.  If
used within a \xtag{label}, should also be \emph{Valid}.
\paragraph*{Children}
None.

\paragraph{``enum'' generator of ``tuple'' sort}

Describes a generator of labelSet when its \xattr{genDescript} is set
to ``enum'' and \xattr{genKind} to ``tuple''.
\example{automatonPhim1}{17-32}.

\paragraph*{Attributes}
None.
\paragraph*{Children}
\children{
  \child{labelCompGen}{A ``tuple'' label set generator. Occurs
    \xval{genDim} times.}{labelCompGen}.
}

\subsubsection{``range'' and ``set'' generators}

Even if these \texttt{Tokens} are present in FSMXML, no further
investigations were done. Since {\sc Vaucanson} only use enumerations,
we let these investigations to more expert people.
% FIXME : reformulation ?

\subsection{\xtag{labelCompGen}}
\label{labelCompGen}

Gives the \xval{k}th coordinate/component in a ``tuple'' generator,
\xval{k} being the position of the element in the list within
\xtag{labelGen}.
\example{automatonPhim1}{18-19}.

\subsubsection*{Attributes}
\attrs{
  \attr{value}{Required}{generator}{none}{}{ Gives the value of the
    coordinate/component. Should fit the associated
    \xval{genCompSort}.
  }
}

\subsubsection*{Children}
None.

\subsection{\xtag{weightSet}}
\label{weightset}

Holds a weight set description.

\subsubsection*{Attributes}
\attrs{
  \attr{type}{Required, Pivot}{token}{none}{``numerical'', ``series''}
  {Type of the weight set.}
}

\subsubsection*{Children}
None.

\subsubsection{Numerical Weight Set}

When \xattr{type} is set to ``numerical'', \xtag{weightSet} describes
a numerical weight set and inherits the following new attributes and
children. \example{automatonA1}{6-8}.

\subsubsection*{Attributes}
\attrs{
  \attr{set}{Required}{token}{none}{``b'', ``n'', ``z'', ``q'', ``r'',
    ``c''\ldots} {The set on which is described the weight set.}
  \attr{operation}{Required}{token, string}{none}{``classical'',
    ``minPlus'', ``maxPlus''\ldots}{Set the operations to work with into the
    weight set. This is not an exhaustive list.}
}
\subsubsection*{Children}
  None.

\subsubsection{Series Weight Set}

When \xattr{type} is set to ``series'', the \xtag{weightSet} describes
a series weight set and inherits the following new attributes and
children. \example{automatonD2rw}{6-19}.

\subsubsection*{Attributes}
None.
\subsubsection*{Children}
\children{
  \child{weightSet}{A weight set. Occurs 1 time.}{weightset}.
  \child{labelSet}{A label set. Occurs 1 time.}{labelSet}.
}

%%%%%%%%%%%%%%%%%%%%%%%%%%%%%%%%%%%%%%%%%%%%%%%%%%%%%%%%%%%

\newcommand{\RegExpBody}[1]{
  \item #1
  See \autoref{RegExpBody} for more details\xspace
}


\section{Regular Expressions}

\subsection{\xtag{expressions}}
\label{expression}

Holds the complete representation of a regular expression.
\example{regExpC1}{4-52}.

\subsubsection*{Attributes}
None.
\subsubsection*{Children}
\children{
  \child{valueType}{The regular expression's type. Occurs 1
    time.}{valueType}.
  \child{typedExpression}{The regular expression's body. Occurs 1
    time.}{typedExpression}.
}

\subsection{\xtag{typedExpression}}
\label{typedExpression}

Holds the typed regular expression. \example{regExpC1}{17-51}.

\subsubsection*{Attributes}
None.
\subsubsection*{Children}
\children{
  \RegExpBody{Typed regular expression. Occurs 1 time}.
}

\subsection{Regular Expression's Body}
\label{RegExpBody}

A regular expression's body is represented by a recursive tree of
elements, listed below.  Any of these elements is seen as a ``Typed
regular expression''. \example{regExpC1}{18-50}.

\subsubsection{\xtag{sum}}
\label{sum}

Sum of two expressions.

\subsubsection*{Attributes}
None.
\subsubsection*{Children}
\children{
  \RegExpBody{Left Typed regular expression. Occurs 1 time}.
  \RegExpBody{Right Typed regular expression. Occurs 1 time}.
}

\subsubsection{\xtag{product}}
\label{product}

Product of two expressions.

\subsubsection*{Attributes}
None.
\subsubsection*{Children}
\children{
  \RegExpBody{Left Typed regular expression. Occurs 1 time}.
  \RegExpBody{Right Typed regular expression. Occurs 1 time}.
}

\subsubsection{\xtag{star}}
\label{star}

Star of an expression.

\subsubsection*{Attributes}
None.
\subsubsection*{Children}
\children{
  \RegExpBody{Typed regular expression to starify. Occurs 1 time}.
}

\subsubsection{\xtag{rightExtMul} and \xtag{leftExtMul}}
\label{rightExtMulleftExtMul}

Represents the right and left scalar multiplication of an expression.

\subsubsection*{Attributes}
None.
\subsubsection*{Children}
\children{
  \child{weight}{Weight for the multiplication. Occurs 1
    time.}{weight}.
  \RegExpBody{Typed regular expression to multiply. Occurs 1 time}.
}

\subsubsection{\xtag{label}}
\label{label}

Represents a label set element, which is a concatenation of
label set generators.

\paragraph{On a free Label Set}
\label{labelFree}

Represents a label set element on a free label set.
\example{automatonA1}{26-28}.

\paragraph*{Attributes}
None.

\paragraph*{Children}
\children{
  \child{labelGen}{A label set Generator. Occurs as many time as
    wanted.}{labelGen}.
}

\paragraph{On a product Label Set.}

Represents a label element on a product label set.
\example{automatonPhim1}{35-42}.

\paragraph*{Attributes}
None.

\paragraph*{Children}
\children{
\item \xtag{label} or \xtag{one} A label set element or the identity
  of a \texttt{free} label set. Occurs \xval{prodDim} times.  See
  \autoref{labelFree} and \autoref{one} for more details.
}

\subsubsection{\xtag{zero}}
\label{zero}

Represents the null series.

\subsubsection*{Attributes}
None.
\subsubsection*{Children}
None.

\subsubsection{\xtag{one}}
\label{one}

Represents the identity series or the identity symbol of a
\texttt{free} label set.

\subsubsection*{Attributes}
None.
\subsubsection*{Children}
None.

\subsection{\xtag{weight}}
\label{weight}

Represents the weight of an expression.

\subsubsection{On a Numerical Weight Set}

Represents the weight of an expression with a numerical weight set.
\example{automatonC1}{47-47}.

\subsubsection*{Attributes}
\attrs{
  \attr{value}{Required}{weight}{none}{}{Weight's value.}
}
\subsubsection*{Children}
None.

\subsubsection{On a Series Weight Set}

Represents the weight of an expression with a series weight set.
\example{automatonD2rw}{39-43}.

\subsubsection*{Attributes}
None.
\subsubsection*{Children}
\children{
  \RegExpBody{Typed regular expression taken into the
    weight set. Occurs 1 time.}
}

%%%%%%%%%%%%%%%%%%%%%%%%%%%%%%%%%%%%%%%%%%%%%%%%%%%%%%%%%%%
\section{Automata}

\subsection{\xtag{automaton}}
\label{automaton}

Holds the complete representation of an automaton.
\example{automatonA1}{4-70}.

\subsubsection*{Attributes}
\attrs{
  \attr{name}{}{string}{}{}{Name of the automaton.}
  \attr{readingDir}{}{token}{left}{``left'', ``right''}{Reading
    direction of the automaton.}
}

\subsubsection*{Children}
\children{
  \child{valueType}{The automaton's type. Occurs 1
    time.}{valueType}.
  \child{automatonStruct}{The automaton's content. Occurs 1
    time.}{automatonStruct}.
}

\subsection{\xtag{automatonStruct}}
\label{automatonStruct}

Holds the automaton's content. \example{automatonA1}{17-69}.

\subsubsection*{Attributes}
None.
\subsubsection*{Children}
\children{
  \child{states}{Lists the automaton's states. Occurs 1
    time.}{states}.
  \child{transitions}{Lists the automaton's transitions. Occurs 1
    time.}{transitions}.
}

\subsection{\xtag{states}}
\label{states}

Holds the automaton's states. \example{automatonA1}{18-22}.

\subsubsection*{Attributes}
None.
\subsubsection*{Children}
\children{
  \child{state}{Adds a state. Occurs 0 or more times.}{state}.
}

\subsection{\xtag{state}}
\label{state}

Describes a state.
\xattr{key} is usually used to define in which order should be
processed the states.
\xattr{name} allows you to give a more explicit name.
\example{automatonA1}{19-21}.

\subsubsection*{Attributes}
\attrs{
  \attr{id}{Required, Unique}{ID}{none}{} {Id of the state.}
}
\subsubsection*{Children}
None.

\subsection{\xtag{transitions}}
\label{transitions}

Holds the automaton's transitions and initial/final properties of
states. \example{automatonA1}{23-68}.

\subsubsection*{Attributes}
None.
\subsubsection*{Children}
\children{
  \child{transition}{Adds a transition. Occurs 0 or more
    times.}{transition}.
  \child{initial}{Adds the initial property to a state. Occurs 0 or
    more times.}{initialfinal}.
  \child{final}{Adds the final property to a state. Occurs 0 or
    more times.}{initialfinal}.
}

\subsection{\xtag{transition}}
\label{transition}

Describes a transition. \example{automatonA1}{24-30}.

\subsubsection*{Attributes}
\attrs{
  \attr{source}{Required}{ID}{none}{Valid ID} {Source of the
    transition.}
  \attr{target}{Required}{ID}{none}{Valid ID} {Target of the
    transition.}
}
\subsubsection*{Children}
\children{
  \child{label}{Label of the transition. Occurs 1 time}{label}.
}

\subsection{\xtag{initial} and \xtag{final}}
\label{initialfinal}

Adds the initial/final property to a state.
\example{automatonA1}{66-67}.

\subsubsection*{Attributes}
\attrs{
  \attr{state}{Required}{ID}{none}{Valid ID} {State to give the
    initial/final property.}
}
\subsubsection*{Children}
\children{
  \child{label}{Label of the transition. Occurs 1 time}{label}.
}

\subsection{\xtag{label}}
\label{label}

Holds the label of a transition. \example{automatonA1}{25-29}.

\subsubsection*{Attributes}
None.
\subsubsection*{Children}
\children{
  \RegExpBody{Typed regular expression. Occurs 1 time}.
}

%%%%%%%%%%%%%%%%%%%%%%%%%%%%%%%%%%%%%%%%%%%%%%%%%%%%%%%%%%%
\newpage
\appendix

%\setcounter{tocdepth}{5}
%\setcounter{secnumdepth}{1}

%%%%%%%%%%%%%%%%%%%%%%%%%%%%%%%%%%%%%%%%%%%%%%%%%%%%%%%%%%%%%%%%%%%%%
%% Macro for figures
%%%%%%%%%%%%%%%%%%%%%%%%%%%%%%%%%%%%%%%%%%%%%%
%%%%%% \section{Twocolumns macros}%%% colonne + figure %%%%%%%%
%%% variables
\newlength{\ColoText}% largeur de la colonne "de texte"
\newlength{\ColoFigu}% largeur de la colonne "de figure"
\newlength{\TextFiguSpace}% intervalle entre les deux colonnes
\newlength{\parindenttemp} % for indentation in minipage
\newlength{\parskiptemp} % for alinea spacing in minioage
\newlength{\fboxseptemp} % pour memoriser \fboxsep
\newcommand{\TFBoxing}{}
\newcommand{\TFVertAlig}{}
% values
\setlength{\fboxseptemp}{\fboxsep}% parce qu'on va l'annuler en draft
\setlength{\parindenttemp}{\parindent}
\setlength{\parskiptemp}{\parskip}
\setlength{\TextFiguSpace}{1.2em}% intervalle entre les deux colonnes
% \ifdraft\renewcommand{\TFBoxing}{\fbox}\fi
%%%%%%%%%%%%%%%%%%%%%%%%%%%%%%%%%%%%%%
%%% 3 paramèes
%%% 1) largeur de la colonne de gauche, en % de \textwidth
%%%    valeur implicite : 2/3
%%% 2) texte de la colonne de gauche
%%% 3) texte de la colonne de droite (souvent une figure)
%%% Boites aligné sur le haut,
%%%        encadré en mode draft
\newcommand{\TxtFg}[3]%
{%
  \setlength{\ColoText}{#1\textwidth}%
    \setlength{\ColoFigu}{\textwidth}%
    \addtolength{\ColoFigu}{-\ColoText}%
    \addtolength{\ColoText}{-.5\TextFiguSpace}%
    \addtolength{\ColoFigu}{-.5\TextFiguSpace}%
    \ifdraft\setlength{\fboxsep}{0pt}\fi% % mod 000912, 050822
    %
    \noi
    \TFBoxing{%
      \begin{minipage}[\TFVertAlig]{\ColoText}%
	%           \RstBLS% 050822
	\setlength{\parindent}{\parindenttemp}%
	\setlength{\parskip}{\parskiptemp}%
	\par\vspace*{0mm}% pour l'alignement sur le haut
#2
	\end{minipage}%
    }%
  \hspace*{\TextFiguSpace}%
    \TFBoxing{%
      \begin{minipage}[\TFVertAlig]{\ColoFigu}%
	\par\vspace*{0mm}%
#3%
	\end{minipage}%
    }%
  %
    %     \ifdraft\setlength{\fboxsep}{\fboxseptemp}\fi%050822
}%
%%%
\newcommand{\TextFigu}[3][.66]%
{\renewcommand{\TFVertAlig}{t}\TxtFg{#1}{#2}{#3}}
\newcommand{\TextFiguC}[3][.66]%
{\renewcommand{\TFVertAlig}{c}\TxtFg{#1}{#2}{#3}}
%%%%%%%%% Figures vers l'exterieur avec dérdement
%%% ie la colonne "de droite" est du cotée la marge
\newcommand{\TextFiguX}[3][.66]
{%
  \setlength{\ColoText}{#1\textwidth}%
    \setlength{\ColoFigu}{\textwidth}%
    \addtolength{\ColoFigu}{-\ColoText}%
    \addtolength{\ColoText}{-.5\TextFiguSpace}%
    \addtolength{\ColoFigu}{-.5\TextFiguSpace}%
    \addtolength{\ColoFigu}{\ETAExtendedLineWidth}% mod 000912,050822
    \ifdraft\setlength{\fboxsep}{0pt}\fi% % mod 000912, 050822
    %
    \noi
    \ifodd\value{page}%
    \TFBoxing{%
      \begin{minipage}[t]{\ColoText}%
	%              \RstBLS% 050822
	\setlength{\parindent}{\parindenttemp}%
	\setlength{\parskip}{\parskiptemp}%
	\par\vspace*{0mm}% pour l'alignement sur le haut
#2
	\end{minipage}%
    }%
  \hspace*{\TextFiguSpace}%
    \TFBoxing{%
      \begin{minipage}[t]{\ColoFigu}%
	\par\vspace*{0mm}%
#3%
	\end{minipage}%
    }%
  \else
    \hspace*{-\ETAExtendedLineWidth}% mod 000912
    \TFBoxing{%
      \begin{minipage}[t]{\ColoFigu}%
	\par\vspace*{0mm}%
#3%
	\end{minipage}%
    }%
  \hspace*{\TextFiguSpace}%
    \TFBoxing{%
      \begin{minipage}[t]{\ColoText}%
	%              \RstBLS% 050822
	\setlength{\parindent}{\parindenttemp}%
	\setlength{\parskip}{\parskiptemp}%
	\par\vspace*{0mm}% pour l'alignement sur le haut
#2
	\end{minipage}%
    }%
  \fi%
    %     \ifdraft\setlength{\fboxsep}{\fboxseptemp}\fi%050822
} 


%%%%%%%%%%%%%%%%%%%%%%%%%%%%%%%%%%%%%%%%%%%%%%%%%%%%%%%%%%%%%%%%%%%%%
%% Figures
\section{Examples}

\subsection{Automaton $\mathcal{A}_1$}
\label{automatonA1}

\begin{figure}[h]
  \begin{minipage}[c]{.66\textwidth}
    Ref: ETA p. 58 Fig. I.1.1
  \end{minipage}
  \begin{minipage}[c]{.34\textwidth}

\VCDraw{%
  \begin{VCPicture}{(-1.4,-1.4cm)(7.4,1.4cm)}
  % etats
  \State[p]{(0,0)}{A}\State[q]{(3,0)}{B}
  \State[r]{(6,0)}{C}
  %
  \Initial{A}\Final{C}
  % transitions 
  \EdgeL{A}{B}{a}\EdgeL{B}{C}{b}
  %
  \LoopN{A}{a}\LoopS{A}{b}
  \LoopN{C}{a}\LoopS{C}{b}
  %
  \end{VCPicture}
}
  \end{minipage}
\end{figure}

%% Automaton Type
%% FIXME http://vaucanson-project.org
{\footnotesize 
\begin{listing}[5]{1}
<fsmxml  xmlns   = "http://vaucanson-project.org" 
         version = "1.0" > 

<automaton name = "a1" >
  <valueType>
    <semiring  type        = 'numerical'
               set         = 'B'
               operations  = 'classical' /> 
    <monoid    type        = 'free'
               genSort     = 'simple' 
               genKind     = 'letter' 
               genDescript = 'enum' >
      <monGen value="a"/>
      <monGen value="b"/>
    </monoid>
  </valueType>
\end{listing}
}

\newpage

%% Automaton Content
{\footnotesize 
\begin{listingcont}
  <automatonStruct>
    <states>
      <state  id="0"  name="p"/>
      <state  id="1"  name="q"/>
      <state  id="2"  name="r"/>
    </states>
    <transitions>
      <transition  source="0"  target="0" >
        <label>
          <monElmt>
            <monGen value="a"/>
          </monElmt>
        </label>
      </transition>
      <transition  source="0"  target="0" >
        <label>
          <monElmt>
            <monGen value="b"/>
          </monElmt>
        </label>
      </transition>
      <transition  source="0"  target="1" >
        <label>
          <monElmt>
            <monGen value="a"/>
          </monElmt>
        </label>
      </transition>
      <transition  source="1"  target="2" >
        <label>
          <monElmt>
            <monGen value="b"/>
          </monElmt>
        </label>
      </transition>
      <transition  source="2"  target="2" >
        <label>
          <monElmt>
            <monGen value="a"/>
          </monElmt>
        </label>
      </transition>
      <transition  source="2"  target="2" >
        <label>
          <monElmt>
            <monGen value="b"/>
          </monElmt>
        </label>
      </transition>
      <initial  state="0"/>
      <final    state="2"/>
    </transitions>
  </automatonStruct>
</automaton>

</fsmxml>
\end{listingcont}
}

\clearpage

\subsection{Automaton $\mathcal{P}''_1$}
\label{automatonPdp1}

\begin{figure}[h]
  \begin{minipage}[c]{.66\textwidth}
    Ref: HECCA p. 14 Fig. 7(a)
  \end{minipage}
  \begin{minipage}[c]{.34\textwidth}

\VCDraw{%
  \begin{VCPicture}{(-1.4,-2)(7.4,2)}
  % etats
  \MediumState
  \State[4]{(0,0)}{A}\State[1]{(3,1.5)}{B}\State[2]{(3,-1.5)}{Bp}\State[3]{(6,0)}{C}
  \Initial[n]{A}\Final[s]{A}%
  % transitions
  \ArcL{A}{B}{b}\ArcL{B}{A}{\overline{b}}
  \EdgeL{Bp}{A}{b}
  \ArcL{Bp}{C}{a}\ArcL{C}{Bp}{a}
  \EdgeL{B}{C}{\overline{a}}
  \LoopW{A}{a}\LoopE{C}{b}
  %
  \end{VCPicture}
}
  \end{minipage}
\end{figure}

%% Automaton Type
%% FIXME http://vaucanson-project.org
{\footnotesize 
\begin{listing}[5]{1}
<fsmxml  xmlns   = "http://vaucanson-project.org" 
         version = "1.0" > 

<automaton name = "Pdp1" >
  <valueType>
    <semiring  type       = 'numerical'
               set        = 'B'
               operations = 'classical' /> 
    <monoid    type       = 'free'
               genSort    = 'tuple' 
               genDescript = 'enum' 
	       genDim	  = "2" >
      <genSort>
         <genCompSort value = 'letter' />
         <genCompSort value = 'digit'  />
      </genSort>
      <monGen>
         <monCompGen value = "a" />
         <monCompGen value = "0" />
      </monGen>
      <monGen>
         <monCompGen value = "a" />
         <monCompGen value = "1" />
      </monGen>
      <monGen>
         <monCompGen value = "b" />
         <monCompGen value = "0" />
      </monGen>
      <monGen>
         <monCompGen value = "b" />
         <monCompGen value = "1" />
      </monGen>
    </monoid>
  </valueType>
\end{listing}
}

\newpage

%% Automaton Content
{\footnotesize 
\begin{listingcont}
  <automatonStruct>
    <states>
    <state  id="0"  name="1"/>
    <state  id="1"  name="2"/>
    <state  id="2"  name="3"/>
    <state  id="3"  name="4"/>
    </states>
    <transitions>
      <transition  source="3"  target="3" >
        <label>
	  <monElmt>
	    <monGen>
	      <monCompGen value = "a" />
	      <monCompGen value = "0" />
	    </monGen>
	  </monElmt>
        </label>
      </transition>
      <transition  source="3"  target="0" >
        <label>
	  <monElmt>
	    <monGen>
	      <monCompGen value = "b" />
	      <monCompGen value = "0" />
	    </monGen>
	  </monElmt>
        </label>
      </transition>
      <transition  source="0"  target="3" >
        <label>
	  <monElmt>
	    <monGen>
	      <monCompGen value = "b" />
	      <monCompGen value = "1" />
	    </monGen>
	  </monElmt>
        </label>
      </transition>
      <transition  source="0"  target="2" >
        <label>
	  <monElmt>
	    <monGen>
	      <monCompGen value = "a" />
	      <monCompGen value = "1" />
	    </monGen>
	  </monElmt>
        </label>
      </transition>
      ...
      <initial state="3"/>
      <final   state="3"/>
    </transitions>
  </automatonStruct>
</automaton>

</fsmxml>
\end{listingcont}
}

\clearpage

\subsection{Automaton $\mathcal{C}_1$}
\label{automatonC1}

\begin{figure}[h]
  \begin{minipage}[c]{.66\textwidth}
    Ref: ETA p. 437 Fig. III.2.6
  \end{minipage}
  \begin{minipage}[c]{.34\textwidth}

\VCDraw{%
  \begin{VCPicture}{(-1.4,-1.4cm)(7.4,1.4cm)}
  % etats
  \State[p]{(0,0)}{A}\State[q]{(6,0)}{B}
  %
  \Initial{A}\Final{B}
  % transitions 
  \EdgeL{A}{B}{b}
  %
  \LoopN{A}{a}\LoopS{A}{b}
  \LoopN{B}{2\,a}\LoopS{B}{2\,b}
  %
  \end{VCPicture}
}
  \end{minipage}
\end{figure}

%% Automaton Type
{\footnotesize 
\begin{listing}[5]{1}
<fsmxml  xmlns   = "http://vaucanson-project.org" 
         version = "1.0" > 

<automaton name = "c1" >
  <valueType>
    <semiring  type       = 'numerical'
               set        = 'N'
               operations = 'classical' /> 
    <monoid    type       = 'free'
               genSort    = 'simple' 
               genKind    = 'digit' 
               genDescript = 'enum' > 
      <monGen value="0"/>
      <monGen value="1"/>
    </monoid>
  </valueType>
\end{listing}
}

\newpage 

%% Automaton Content
{\footnotesize 
\begin{listingcont}
  <automatonStruct>
    <states>
      <state  id="0"  name="p"/>
      <state  id="1"  name="q"/>
    </states>
    <transitions>
      <transition  source="0"  target="0" >
        <label>
          <monElmt>
            <monGen value="0"/>
          </monElmt>
        </label>
      </transition>
      <transition  source="0"  target="0" >
        <label>
          <monElmt>
            <monGen value="1"/>
          </monElmt>
        </label>
      </transition>
      <transition  source="0"  target="1" >
        <label>
          <monElmt>
            <monGen value="1"/>
          </monElmt>
        </label>
      </transition>
      <transition  source="1"  target="1" >
        <label>
	  <leftExtMul>
            <weight value="2"/>
	    <monElmt>
	      <monGen value="0"/>
	    </monElmt>
	  </leftExtMul>
        </label>
      </transition>
      <transition  source="1"  target="1" >
        <label>
	  <leftExtMul>
            <weight value="2"/>
	    <monElmt>
	      <monGen value="1"/>
	    </monElmt>
	  </leftExtMul>
        </label>
      </transition>
      <initial  state="0"/>
      <final    state="1"/>
    </transitions>
  </automatonStruct>
</automaton>

</fsmxml>
\end{listingcont}
}

\newpage

\subsection{A rational expression for 
$|\mathcal{C}_{1}| = (a+b)^{*}(b\cdot(2\,a + 2\,b)^{*})$}
\label{regExpC1}

{\footnotesize 
\begin{listing}[5]{1} 
<fsmxml  xmlns   = "http://vaucanson-project.org" 
         version = "1.0" > 

<regExp name = "c1-behaviour" >
  <valueType>
    <semiring  type        = 'numerical'
               set         = 'N'
               operations  = 'classical' /> 
    <monoid    type        = 'free'
               genSort     = 'simple' 
               genKind     = 'digit' 
               genDescript = 'enum' > 
      <monGen value="0"/>
      <monGen value="1"/>
    </monoid>
  </valueType>
  <typedRegExp>
    <product>
      <star>
        <sum>
          <monElmt>
            <monGen value="0"/>
          </monElmt>
          <monElmt>
            <monGen value="1"/>
          </monElmt>
        </sum>
      </star>
      <product>
        <monElmt>
          <monGen value="1"/>
        </monElmt>
        <star>
          <sum>
	    <leftExtMul>
              <weight value="2"/>
	      <monElmt>
		<monGen value="0"/>
	      </monElmt>
	    </leftExtMul>
	    <leftExtMul>
              <weight value="2"/>
	      <monElmt>
		<monGen value="1"/>
	      </monElmt>
	    </leftExtMul>
          </sum>
        </star>
      </product>
    </product>
  </typedRegExp>
</regExp>

</fsmxml>
\end{listing}
}

\clearpage

~
\clearpage
\newpage

\subsection{Automaton $\varphi_{1}^{-1}$}
\label{automatonPhim1}

\begin{figure}[h]
  \begin{minipage}[c]{.66\textwidth}
    Ref: ETA p. 582  (not quite)
    The one state automaton that realises the inverse of the morphism~$\varphi_{1}$.
  \end{minipage}
  \begin{minipage}[c]{.34\textwidth}

\VCDraw{%
  \begin{VCPicture}{(-1.4,-2)(7.4,2)}
  % etats
  \MediumState
  \State[ ]{(5,0)}{A}
  \Initial{A}\Final{A}%
  % transitions
  \LoopN[.5]{A}{(x,a)}
  \LoopSW[.5]{A}{(y\,x,b)}
  \LoopSE[.5]{A}{(x\,y,c)}
  %
  \end{VCPicture}
}

  \end{minipage}
\end{figure}

{\footnotesize
\begin{listing}[5]{1}
<fsmxml  xmlns   = "http://vaucanson-project.org"
         version = "1.0" >

<automaton name = "phim1" >
  <valueType>
    <semiring  type       = 'numerical'
               set        = 'B'
               operations = 'classical' />
    <monoid    type       = 'product'
               prodDim    = "2" >
      <monoid    type        = 'free'
                 genSort     = 'simple'
                 genKind     = 'letter'
                 genDescript = 'enum' >
        <monGen value="x"/>
        <monGen value="y"/>
      </monoid>
      <monoid    type       = 'free'
                 genSort    = 'simple'
                 genKind    = 'letter'
                 genDescrip = 'enum' >
        <monGen value="a"/>
        <monGen value="b"/>
        <monGen value="c"/>
      </monoid>
    </monoid>
  </valueType>
\end{listing}
}

\newpage

{\footnotesize
\begin{listingcont}
  <automatonStruct>
    <states>
      <state  id="0" />
    </states>
    <transitions>
      <transition  source="0"  target="0" >
        <label>
	  <monElmt>
	    <monElmt>
	      <monGen  value = "x" />
	    </monElmt>
	    <monElmt>
	      <monGen  value = "a" />
	    </monElmt>
	  </monElmt>
        </label>
      </transition>
      <transition  source="0"  target="0" >
        <label>
	  <monElmt>
	    <monElmt>
	      <monGen  value = "y" />
	      <monGen  value = "x" />
	    </monElmt>
	    <monElmt>
	      <monGen  value = "b" />
	    </monElmt>
	  </monElmt>
        </label>
      </transition>
      <transition  source="0"  target="0" >
        <label>
	  <monElmt>
	    <monElmt>
	      <monGen  value = "x" />
	      <monGen  value = "y" />
	    </monElmt>
	    <monElmt>
	      <monGen  value = "c" />
	    </monElmt>
	  </monElmt>
        </label>
      </transition>
      <initial state="0"/>
      <final   state="0"/>
    </transitions>
  </automatonStruct>
</automaton>

</fsmxml>
\end{listingcont}
}

\clearpage

\subsection{Automaton $\mathcal{D}_{2}$}
\label{automatonD2fmp}

\begin{figure}[h]
  \begin{minipage}[c]{.60\textwidth}
    Ref: ETA p. 699  Fig. V.1.4

    The transducer~$\mathcal{D}_{2}$ that realises the Fibonacci reduction
    $a\, b\, b \rightarrow b\, a\, a$, viewed as an automaton on
    $\left\{ a,b \right\}* \times \left\{ a,b \right\}*$.

  \end{minipage}
  \begin{minipage}[c]{.40\textwidth}

\VCDraw{%
\begin{VCPicture}{(3,-1.4)(13,2)}
% param etats
\SmallState
% etats
% \State{(0,0)}{A}{p}
\State{(5,0)}{B} %{d}
\State{(9,0)}{C} %{q}
\State{(13,0)}{D} %{r}
%
\ChgEdgeLabelScale{0.9}
\Initial{B}
\Final[s]{B}
\FinalR[1]{s}{C}{\IOL{1}{a}}
\FinalL[1]{s}{D}{\IOL{1}{ab}}
% transitions [.75]
\EdgeL{B}{C}{\IOL{a}{1}}
\ArcR{C}{D}{\IOL{b}{1}}
\ArcR{D}{C}{\IOL{a}{ab}}
%
\LoopN[.5]{B}{\IOL{b}{b}}
\LoopN[.5]{C}{\IOL{a}{a}}
\VCurveR[.2]{ncurv=1.2,angleA=90,angleB=40}%
    {D}{C}{\IOL{b}{ba}}
%
\end{VCPicture}
}

  \end{minipage}
\end{figure}

{\footnotesize
\begin{listing}[5]{1}
<fsmxml  xmlns   = "http://vaucanson-project.org"
         version = "1.0" >

<automaton name = "d2" >
  <valueType>
    <semiring  type       = 'numerical'
               set        = 'B'
               operations = 'classical' />
    <monoid    type       = 'product'
               prodDim    = "2" >
      <monoid    type        = 'free'
                 genSort     = 'simple'
                 genKind     = 'letter'
                 genDescript = 'enum' >
        <monGen value="a"/>
        <monGen value="b"/>
      </monoid>
      <monoid    type        = 'free'
                 genSort     = 'simple'
                 genKind     = 'letter'
                 genDescript = 'enum' >
        <monGen value="a"/>
        <monGen value="b"/>
      </monoid>
    </monoid>
  </valueType>

  <automatonStruct>
    <states>
      <state  id="0"  name = "1" />
      <state  id="1"  name = "a" />
      <state  id="2"  name = "ab" />
    </states>

\end{listing}
}

\newpage

{\footnotesize
\begin{listingcont}
    <transitions>
      <transition  source="0"  target="0" >
	<label>
	  <monElmt>
	    <monElmt>
	      <monGen  value = "b" />
	    </monElmt>
	    <monElmt>
	      <monGen  value = "b" />
	    </monElmt>
	  </monElmt>
        </label>
      </transition>
      <transition  source="0"  target="1" >
        <label>
	  <monElmt>
	    <monElmt>
	      <monGen  value = "a" />
	    </monElmt>
	    <one/>
	  </monElmt>
        </label>
      </transition>
      <transition  source="1"  target="1" >
        <label>
	  <monElmt>
	    <monElmt>
	      <monGen  value = "a" />
	    </monElmt>
	    <monElmt>
	      <monGen  value = "a" />
	    </monElmt>
	  </monElmt>
        </label>
      </transition>
      <transition  source="1"  target="2" >
        <label>
	  <monElmt>
	    <monElmt>
	      <monGen  value = "b" />
	    </monElmt>
	    <one/>
	  </monElmt>
        </label>
      </transition>
\end{listingcont}
}

\newpage
\enlargethispage*{2ex}
{\footnotesize
\begin{listingcont}
      <transition  source="2"  target="1" >
        <label>
	  <monElmt>
	    <monElmt>
	      <monGen  value = "a" />
	    </monElmt>
	    <monElmt>
	      <monGen  value = "a" />
	      <monGen  value = "b" />
	    </monElmt>
	  </monElmt>
        </label>
      </transition>
      <transition  source="2"  target="1" >
        <label>
	  <monElmt>
	    <monElmt>
	      <monGen  value = "b" />
	    </monElmt>
	    <monElmt>
	      <monGen  value = "b" />
	      <monGen  value = "a" />
	    </monElmt>
	  </monElmt>
        </label>
      </transition>
      <initial state="0"/>
      <final   state="0"/>
      <final   state="1"/>
        <label>
	  <monElmt>
	    <one/>
	    <monElmt>
	      <monGen  value = "a" />
	    </monElmt>
	  </monElmt>
        </label>
      </final>
      <final   state="2"/>
        <label>
	  <monElmt>
	    <one/>
	    <monElmt>
	      <monGen  value = "a" />
	      <monGen  value = "b" />
	    </monElmt>
	  </monElmt>
        </label>
      </final>
    </transitions>
  </automatonStruct>
</automaton>

</fsmxml>
\end{listingcont}
}

\clearpage

\subsection{Automaton $\mathcal{D}_{2}$ (bis)}
\label{automatonD2rw}
\begin{figure}[h]
  \begin{minipage}[c]{.60\textwidth}
    Ref: ETA p. 699  Fig. V.1.4

    The transducer~$\mathcal{D}_{2}$ that realises the Fibonacci reduction 
    $a\, b\, b \rightarrow b\, a\, a$, viewed as an automaton on 
    $\left\{ a,b \right\}*$ with multiplicities in $\mathrm{Rat}\left\{ a,b \right\}*$.
  \end{minipage}
  \begin{minipage}[c]{.40\textwidth}

  \VCDraw{%
\begin{VCPicture}{(3,-1.4)(13,2)}
% param etats
\SmallState
% etats
% \State{(0,0)}{A}{p}
\State{(5,0)}{B} %{d}
\State{(9,0)}{C} %{q}
\State{(13,0)}{D} %{r}
%
\ChgEdgeLabelScale{0.9}
\Initial{B}
\Final[s]{B}
\FinalR[1]{s}{C}{\IOL{1}{a}}
\FinalL[1]{s}{D}{\IOL{1}{ab}}
% transitions [.75] 
\EdgeL{B}{C}{\IOL{a}{1}}
\ArcR{C}{D}{\IOL{b}{1}}
\ArcR{D}{C}{\IOL{a}{ab}}
%
\LoopN[.5]{B}{\IOL{b}{b}}
\LoopN[.5]{C}{\IOL{a}{a}}
\VCurveR[.2]{ncurv=1.2,angleA=90,angleB=40}%
    {D}{C}{\IOL{b}{ba}}
%
\end{VCPicture}
}

  \end{minipage}
\end{figure}

{\footnotesize 
\begin{listing}[5]{1} 
<fsmxml  xmlns   = "http://vaucanson-project.org" 
         version = "1.0" > 

<automaton name = "d2-rw" >
  <valueType>
    <semiring  type       = 'series'
               zeroSymbol     = "0"
               identitySymbol = "1" >
      <semiring  type       = 'numerical'
                 set        = 'B'
                 operations = 'classical' /> 
      <monoid    type       = 'free'
                 genSort    = 'simple' 
                 genKind    = 'letter' 
                 genDescript = 'enum' > 
        <monGen value="a"/>
        <monGen value="b"/>
      </monoid>
    </semiring>
    <monoid    type       = 'free'
               genSort    = 'simple' 
               genKind    = 'letter' 
               genDescript = 'enum'> 
      <monGen value="a"/>
      <monGen value="b"/>
    </monoid>
  </valueType>

  <automatonStruct>
    <states>
      <state  id="0"  name = "1" />
      <state  id="1"  name = "a" />
      <state  id="2"  name = "ab" />
    </states>
\end{listing}
}

\newpage 

{\footnotesize 
\begin{listingcont}
    <transitions>
      <transition  source="0"  target="0" >
        <label>
          <leftExtMul>
            <weight>
	      <monElmt>
		<monGen  value = "b" />
	      </monElmt>
            </weight>
            <monElmt>
              <monGen  value = "b" />
            </monElmt>
          </leftExtMul>
        </label>
      </transition>
      <transition  source="0"  target="1" >
        <label>
	  <monElmt>
	    <monGen  value = "a" />
	  </monElmt>
        </label>
      </transition>
      <transition  source="1"  target="1" >
        <label>
          <leftExtMul>
            <weight>
	      <monElmt>
		<monGen  value = "a" />
	      </monElmt>
            </weight>
            <monElmt>
              <monGen  value = "a" />
            </monElmt>
          </leftExtMul>
        </label>
      </transition>
      <transition  source="1"  target="2" >
        <label>
          <monElmt>
            <monGen  value = "b" />
          </monElmt>
        </label>
      </transition>
\end{listingcont}
}

\newpage 
\enlargethispage*{2ex}
{\footnotesize 
\begin{listingcont}
      <transition  source="2"  target="1" >
        <label>
          <leftExtMul>
            <weight>
	      <monElmt>
		<monGen  value = "a" />
		<monGen  value = "b" />
	      </monElmt>
            </weight>
            <monElmt>
              <monGen  value = "a" />
            </monElmt>
          </leftExtMul>
        </label>
      </transition>
      <transition  source="2"  target="1" >
        <label>
          <leftExtMul>
            <weight>
	      <monElmt>
		<monGen  value = "b" />
		<monGen  value = "a" />
	      </monElmt>
            </weight>
            <monElmt>
              <monGen  value = "b" />
            </monElmt>
          </leftExtMul>
        </label>
      </transition>
      <initial state="0"/>
      <final   state="0"/>
      <final   state="1"/>
	<label>
	  <leftExtMul>
	    <weight>
	      <monElmt>
		<monGen  value = "a" />
	      </monElmt>
	    </weight>
	    <one/>
	  </leftExtMul>
	<label>
      </final>
      <final   state="2"/>
	<label>
	  <leftExtMul>
	    <weight>
	      <monElmt>
		<monGen  value = "a" />
		<monGen  value = "b" />
	      </monElmt>
	    </weight>
	    <one/>
	  </leftExtMul>
	</label>
      </final>
    </transitions>
  </automatonStruct>
</automaton>

</fsmxml>
\end{listingcont}
}

\clearpage


%%%%%%%%%%%%%%%%%%%%%%%%%%%%%%%%%%%%%%%%%%%%%%%%%%%%%%%%%%%
\end{document}

%%% Local Variables:
%%% mode: latex
%%% TeX-master: t
%%% ispell-local-dictionary: "american"
%%% End:
